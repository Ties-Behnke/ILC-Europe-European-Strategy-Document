A key advantage of linear colliders is the possibility to upgrade the center-of-mass energy.
After finalizing the program discussed in Sections~\ref{sec:physics}, \ref{sec:higgs} and \ref{sec:searches} the linear collider can be expanded to explore energies well beyond 250~\gev{}. In this section, the potential of higher-energy operation is reviewed.

{\bf Upgradability:} While the energy reach of circular electron-positron colliders of a given circumference is limited by synchrotron radiation, a linear collider is very well suited to an upgrade of the center-of-mass energy after the 250~\gev{} program. This provides great flexibility to adapt or extend the program in response to new discoveries. The most obvious energy upgrade path is an extension of the linear accelerator (LINAC) sections of the colliders, which provides an increase in center-of-mass energy that is proportional to the length of the LINACs. The design of the ILC presented in the Technical Design Report~\cite{} thus reaches a center-of-mass energy of 500~\gev{} in a facility with a total length of 31~km. An even larger increase in center-of-mass energy may be achieved by exploiting advances in accelerator technology. The development of cavities with higher accelerating gradient can drive a significant increase in the energy while maintaining a compact infrastructure. The ILC TDR documents a possible extention to 1 TeV based on current superconducting RF technology~\cite{}. On a longer time scale, the advent of novel acceleration schemes such as plasma wakefield acceleration may open up the energy regime beyond several \tev. Thus, the linear collider facility can continue to contribute to particle physics over many decades. 



{\bf Measurement of the top-quark mass}

{\bf Top-quark electro-weak couplings}

{\bf Vector-boson fusion production of the Higgs boson}

{\bf Measurement of the Higgs self-coupling}


C. Adolphsen
et al.
, “The International Linear Collider Technical Design Report
-  Volume  3.I:  Accelerator  &  in  the  Technical  Design  Phase,”  arXiv:1306.6353
[physics.acc-ph].

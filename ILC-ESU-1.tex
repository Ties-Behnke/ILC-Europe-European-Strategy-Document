% ****** Start of file apssamp.tex ******
%
%   This file is part of the APS files in the REVTeX 4.1 distribution.
%   Version 4.1r of REVTeX, August 2010
%
%   Copyright (c) 2009, 2010 The American Physical Society.
%
%   See the REVTeX 4 README file for restrictions and more information.
%
% TeX'ing this file requires that you have AMS-LaTeX 2.0 installed
% as well as the rest of the prerequisites for REVTeX 4.1
%
% See the REVTeX 4 README file
% It also requires running BibTeX. The commands are as follows:
%
%  1)  latex apssamp.tex
%  2)  bibtex apssamp
%  3)  latex apssamp.tex
%  4)  latex apssamp.tex
%
\documentclass[%
 reprint,
%superscriptaddress,
%groupedaddress,
%unsortedaddress,
%runinaddress,
%frontmatterverbose, 
%preprint,
%showpacs,preprintnumbers,
%nofootinbib,
%nobibnotes,
%bibnotes,
 amsmath,amssymb,
 aps,
%pra,
%prb,
%rmp,
%prstab,
%prstper,
%floatfix,
]{revtex4-1}

\usepackage{graphicx}% Include figure files
\usepackage{dcolumn}% Align table columns on decimal point
\usepackage{bm}% bold math
%\usepackage{hyperref}% add hypertext capabilities
%\usepackage[mathlines]{lineno}% Enable numbering of text and display math
%\linenumbers\relax % Commence numbering lines

%\usepackage[showframe,%Uncomment any one of the following lines to test 
%%scale=0.7, marginratio={1:1, 2:3}, ignoreall,% default settings
%%text={7in,10in},centering,
%%margin=1.5in,
%%total={6.5in,8.75in}, top=1.2in, left=0.9in, includefoot,
%%height=10in,a5paper,hmargin={3cm,0.8in},
%]{geometry}

%%%%%%%%%%%%%%%%%%%%%%%%%%%%%%%%%%%%%%%%%%%%%%%%%%%%%%%%%%%%%%%%%%%%%
%  needed macros:

%text macros:

\def\etal{{\it et al.}}
\def\etc{{\it etc.}}
\def\eg{{\it e.g.}}

% math mode macros:

\def\ee{e^+e^-} 

%%%%%%%%%%%%%%%%%%%%%%%%%%%%%%%%%%%%%%%%%%%%%%%%%%%%%%%%%%%%%%%%%%%%%%


\begin{document}

\preprint{LCCPEB---}

\title{The International Linear Collider \\ A Global Project}% Force line breaks with \\
\thanks{Version 1.3}%

\author{Jim Brau}
% \altaffiliation[Also at ]{Physics Department, XYZ University.}%Lines break automatically or can be forced with \\
\author{et.al.}%
 \email{Second.Author@institution.edu}
\affiliation{%
 Authors' institution and/or address\\
% This line break forced with \textbackslash\textbackslash
}%

\collaboration{Linear Collider Collaboration}%\noaffiliation

\date{\today}% It is always \today, today,
             %  but any date may be explicitly specified

\begin{abstract}
Input from the International Linear Collider community for the European Strategy Update 

\end{abstract}

\pacs{Valid PACS appear here}% PACS, the Physics and Astronomy
                             % Classification Scheme.
%\keywords{Suggested keywords}%Use showkeys class option if keyword
                              %display desired
\maketitle

%\tableofcontents

\section{\label{sec:intro}Introduction}

1 page - Jim and Juan

    Introduce the ILC250, brief overview of status (technical maturity and TDR, staging, cost analysis, status of political situation)

The central issue in particle physics today is the search for new particles and 
forces needed to address shortcomings of the highly successful Standard Model.  
With the discovery of the Higgs boson in 2012 at the Large Hadron Collider,
the Standard Model was completed.  
While theoretically self-consistent, with
no evidence of physics beyond the Standard Model having appeared, a number of issues remain
unaddressed, leaving the Standard Model as an incomplete theory of the fundamental
interactions.  New particles or forces could advance our understanding
of how the physics of the Standard Model fits into a more complete
picture of the nature of the universe.

Among the outstanding questions that are the focus of energy frontier
efforts are the explanation of the apparent mismatch of the scale of
electroweak physics with the Planck scale, or the hierarchy problem.
Why is the Higgs mass as light as it is?
The Standard Model does not offer any explanation for the evidence for dark matter.
Likewise, gravitation is not included.  These and other issues
motivate intense efforts to investigate the tests of the Standard Model
for small deviations that would provide clues toward answering such
open questions.

For more than twenty years the worldwide community has been engaged in a research
program developing the technology required to realize a high energy linear collider
to precisely measure electron-positron collisions, contributing to 
answering the critical questions at the energy frontier: such as the hierarchy problem,
the nature of dark matter and even the relationship of gravity.
The effort to design and establish the technology for the linear collider 
culminated in the publication of the Technical Design Report (TDR)
for the International Linear Collider (ILC) in 2013~\cite{Behnke:2013xla}.
The ILC presented in the TDR is a 200-500 GeV (extendable to 1 TeV) centre-of-mass 
high luminosity
linear electron-positron collider, based on 1.3 GHz superconducting radio-frequency (SCRF)
accelerating technology. 
The parameters were set by considerations of the physics goals,
with an energy reach designed to likely provide access to the mechanism of 
electroweak symmetry breaking (Higgs or no-Higgs).

Once the mass of the Higgs boson was known, it was established that the
linear collider could begin to address these questions in unique fashion
with an initial center-of-mass energy of 250 GeV.
A revised design of the ILC (the ILC250) has been presented~\cite{Evans:2017rvt}
based on the ILC TDR.  This design retains the final focus and beam dump
 capability to extend the
centre-of-mass energy to 500 GeV and 1 TeV.  The cost estimate for ILC250 has 
been developed and presented as well.

Additionally, advances in the theoretical understanding of the impact of precision
measurements of the Higgs boson couplings by ILC250 have increased the understanding
of their sensitivity to physics beyond the Standard Model.  It is significant sensitivity.

The experimental community has developed the designs for two complementary detectors,
ILD and SiD.  These detectors are designed to optimally address the
ILC physics goals.  They are based on a detector R\&D program that has
contributed a number of advances in detector capabilities.
An additional key need for the experimental program is the software and computing.

This report summarizes the current status of this effort.



\section{\label{sec:phys}Physics}

[2 pages - Michael, Christophe, Keisuke, Jenny, Junping]

The physics case for the construction of the ILC is very strong.   The
most important item in this case is the ability to study the couplings
of the Higgs boson with high precision.  The ILC at 250~GeV also
presents many opportunities to discover new particles that go beyond
the capabilities of the LHC.  Finally, the ILC at 250~GeV opens the
door to further exploration of $\ee$ reactions at higher energies. 
The ILC physics case is reviewed at greater length in the reference
document~\cite{ILCforESS}. 

The Higgs boson is a necessary yet also mysterious part of the
Standard Model of Particle Physics (SM).    In the SM, the Higgs field
couples to every elementary particle and provides the mass for all
quarks, leptons, and heavy vector bosons.   The LHC has now discovered
the Higgs particle and confirmed the couplings responsible for the
masses of the $W$, $Z$, $t$, $b$, and $\tau$ at the qualitative
level~\cite{LHCHiggssummary}.  However, many mysteries could still be
buried here.   The Higgs couplings are not universal, as the gauge
couplings are, and their pattern (which is also the pattern of lepton
and quark masses) is not explained by the SM.  The basic phenomenon that provides
mass for elementary particles---the spontaneous breaking of the gauge
symmetry $SU(2)\times U(1)$---is not explained, and actually cannot be
explained, by the SM.   The Higgs boson could also couple to new
particles and fields that have no SM gauge interactions and are
otherwise completely inaccessible to observation.  Thus, detailed
examination of the Higgs boson properties should be the next major
goal for particle physics experiements.

Within the SM, the couplings of the Higgs boson are specified now that
the parameters of the model, including the Higgs boson mass, are
known.  Expected improvements in the SM parameters in the 2020's will
allow these couplings to be predicted to the part-per-mille level~\cite{Lepage:2014fla}.
Models of new physics correct  these predictions.   These corrections
are predicted to be small, at the 10\% level or below, but they can
be visible to precision experiments.   Most importantly, different
classes of models affect the various Higgs couplings differently, so that
systematic measurement of the Higgs couplings can reveal clues to the
nature of the new intereractions.   The precision study of the Higgs
boson interactions then provides a new method both to {\it discover}  the
presence of physics beyond the SM and to {\it learn}  about its nature.

The couplings of the Higgs boson are now being studied at the LHC, but
it is unlikely that the LHC experiments will be able to reach the
level of precision required for sensitivity to new physics models. 
It is important to remember that the goal of
precision Higgs measurement is not to confirm the SM at a given level
of accuracy but rather to discover robust deviations from the SM
pattern that are signals of new physics.   For all Higgs decay modes in which the
Higgs boson does not appear as a resonance (that is, for all decays
except those to 
$\gamma\gamma$ and $4\ell$), Higgs boson event samples at the LHC are dominated
by SM background processes.  Complex selections are needed to make the
Higgs boson component visible. A few percent modelling uncertainty in
the estimation of the SM backgrounds
 destroys the ability to measure Higgs couplings at the 5\%
level.  Further, since this uncertainty is a systematic error, it will
always leave the question of whether deviations from the SM have truly
been observed.

What is needed for a precision Higgs boson measurement program
is a new experimental method in which all individual Higgs boson decay events
are manifest
and can be studied in detail.   This is provided by the reaction
$\ee\to Zh$ at 250~GeV in the center of mass.
  With small and precisely calculable SM backgrounds, any $Z$ boson
  observed with a lab energy of 110~GeV is recoiling against a Higgs
  boson.   Selecting such events gives the profile of Higgs boson
  decays, in SM leptonic and hadron modes and even in invisible or
  partially visible exotic modes. 

Further, since the cross section for Higgs production can be measured
without measuring any property of the Higgs boson, the scale of Higgs
couplings can be determined and the individual couplings can be
absolutely normalized.  Each individual coupling can be compared to
its SM prediction.

In the description of new physics by an  $SU(2)\times U(1)$-invariant
effective field theory (EFT), new physics effects on the  Higgs boson
couplings to $W$ and $Z$ are related to new physics effects on
precision electroweak observables and in $\ee\to W^+W^-$.   This
latter reaction can also be studied to high precision at an $\ee$
collider.  Beam polarization at the ILC is an especially powerful tool
to separate the contributions of different EFT
coefficients.    In \cite{Barklow:2017suo}, it is shown that {\it all}
relevant EFT
coefficients can be fit {\it simultaneously} from the multiple
observables available at the ILC, giving a 
determination of Higgs boson couplings that is as
model-independent as the EFT description itself. 
 This analysis is reviewed in detail in
\cite{ILCforESS}. 
For the nominal ILC program at 250~GeV, we predict that the Higgs
coupling to $b$ quarks will be measured to 1\% accuracy and the
couplings to $W$ and $Z$ to 0.7\% accuracy.  The spectrum of  expected
measurements is shown in Fig.~\ref{fig:Higgssummary}.  Note that the 
discovery of any anomaly at 250~GeV can be confirmed in running at
500~GeV 
using additional reactions  such as $WW$ fusion production of the
Higgs boson.   Measurements at this
level can discover---and distinguish---models of new physics over a
wide space of possibilities, even for models in which the predicted new
particles are too heavy to be discovered at the LHC~\cite{Barklow:2017suo}.

%%%%%%%%%%%%%%%%
\begin{figure}
\begin{center}
\includegraphics[width=0.95\hsize]{figures/DeltaH_EFT.pdf}
\end{center}
\caption{Projected Higgs boson coupling uncertainties for the ILC
  program at 250~GeV and an energy upgrade to 500~GeV, from \cite{Fujii:2017vwa}, 
 These projections are compared to
  the results of model-dependent estimates for HL-LHC uncertainties 
presented by the ATLAS 
collaboration~\cite{H2aaLHC}. [Note: the LHC projections will be
updated in the fall of 2018.]}
\label{fig:Higgssummary}
\end{figure}
%%%%%%%%%%%%%%%%%%%%%%%%%%%%%%%%%%%%%%%%%%%%%%%%%%%%%%%%%%%%%%

In addition to decays predicted in the SM, the Higgs boson could decay
to particles with no SM gauge interactions.    These decays may
include invisible decays (\eg, to a pair of dark matter particles $\chi$)  or
partially invisible decays (\eg, to $b\bar b \chi \chi$).   The ILC
can robustly seach for all types of exotic decays  to the part per
mille level of branching ratios~\cite{Liu:2016zki}.

The ILC can also search for particles produced through electroweak
interactions, closing gaps that are left by searches at the LHC.  An
important example is the Higgsino of supersymmetric models.   If the
mass  differences among Higgsinos is smaller than a few GeV---which is
actually the prediction of currently allowed models---then Higgsinos
of 100~GeV mass would be produced copiously at the LHC, but this
production would not be registered by LHC triggers.  In the clean
environment 
of the ILC, even such fragile signatures as this 
would be discovered and the new particles 
studied with percent-level precision~\cite{Higgsino}.

 ILC precision
measurements of $\ee\to f\bar f$ processes at 250~GeV have a sensitivity to new
electroweak gauge bosons comparable to (and complementary with) 
direct searches at the LHC.  The reaction $\ee\to b\bar b$ is
particularly interesting, since models of the top quark mass with
composite Higgs bosons can give significant corrections in this
reaction~\cite{eetobb}.

The ILC at 250~GeV can be the first step to the study of $\ee$
reactions at higher energy.   A linear $\ee$ collider is extendable in
energy by making the accelerator longer or by improving the
acceleration gradient. Extensions to 500~GeV and 1~TeV were envisioned
in ILC Technical Design Report~\cite{Behnke:2013xla}.    These would offer a
measurement of the top quark mass to 40~MeV, measurements of the top
quark electroweak couplings to the per-mille level, measurement of the
Higgs coupling to the top quark to 2\% accuracy, and measurement of
the triple Higgs boson coupling to 10\%  accuracy. They  also would be
the setting for much 
deeper searches for new particles with electroweak interactions.
A higher-gradient accelerator in the ILC tunnel could reach even
higher energies.  The
ILC promises a long future beyond its initial 250~GeV stage.




\section{\label{sec:collider}Collider}

%2 pages - Benno, Shin

%    Summary of the ILC250 design (important to note the elements that are retained in first stage to accommodate future energy extensions)

{\it Note (BL): The following 3 paragraphs in italics will probably be absorbed in the general introduction.}

{\it 
The International Linear Accelerator (ILC) is a $250\,{\mathrm{GeV}}$ (extendable up to $1\,{\mathrm{TeV}}$) linear $e^+e^-$ collider, based on $1.3\,{\mathrm{GHz}}$ superconducting radio-frequency (SCRF) cavities.
It is designed to  achieve a luminosity of $1.5\cdot 10^{34}~{\mathrm{cm}}^{-1}{\mathrm{s}}^{-1}$ and provide an integrated luminosity of $350\,{\mathrm{fb}}^{-1}$ in the first four years of running.
The electron beam will be polarised to $80\,\%$, and positrons with $30\,\%$ polarization will be provided if the undulator based positron source concept is employed. 
}

{\it 
Its parameters have been set by physics requirements first outlined in 2003,
updated in 2006, and thoroughly discussed over many years with the physics user community. 
After the discovery of the Higgs boson it was decided that an initial energy of $250\,{\mathrm{GeV}}$ provides the opportunity for a precision physics programme at a reasonable initial cost~\cite{Evans:2017rvt}.
Some relevant parameters are given in Tab.~\ref{tab:ilc-params}.
}

\begin{table}
\begin{tabular}{lccccc}
Quantity & Symbol & Unit & Initial &  \multicolumn{2}{c}{Upgrades} \\
\hline
Centre of mass energy & $\sqrt{s}$ & ${\mathrm{GeV}}$ & $250$ & $500$ & $1000$ \\
Luminosity & \multicolumn{2}{c}{${\mathcal{L}}$ $10^{34}{\mathrm{cm^{-2}s^{-1}}}$} & $1.35$ & $1.8$ & $4.9$ \\
Repetition frequency &$f_{\mathrm{rep}}$ & ${\mathrm{Hz}}$  & $5$ & $5$ & $4$ \\
Bunches per pulse  &$n_{\mathrm{bunch}}$ & 1  & $1312$ & $1312$ & $2450$ \\
Bunch population  &$N_{\mathrm{e}}$ & $10^{10}$ &$2$ & $2$ & $1.74$ \\
Linac bunch interval & $\Delta t_{\mathrm{b}}$ & ${\mathrm{ns}}$ & $554$ & $554$ & $366$ \\
Beam current in pulse & $I_{\mathrm{pulse}}$ & ${\mathrm{mA}}$& $5.8$ & $5.8$ & $7.6$  \\
Beam pulse duration  & $t_{\mathrm{pulse}}$ & ${\mathrm{\mu s}}$ &$727$ & $727$ & $897$ \\
Average beam power  & $P_{\mathrm{ave}}$   & ${\mathrm{MW}}$ & $5.3$   &$10.5$  & $27.2$ \\  
Norm. hor. emitt. at IP & $\gamma\epsilon_{\mathrm{x}}$ & ${\mathrm{\mu m}}$& $5$ & $10$ & $10$  \\ 
Norm. vert. emitt. at IP & $\gamma\epsilon_{\mathrm{y}}$ & ${\mathrm{nm}}$ & $35$ & $35$ & $35$ \\ 
RMS hor. beam size at IP  & $\sigma^*_{\mathrm{x}}$ & ${\mathrm{nm}}$  & $516$ & $474$ & $335$ \\
RMS vert. beam size at IP &$\sigma^*_{\mathrm{y}}$ & ${\mathrm{nm}}$ & $7.7$  & $5.9$ & $2.7$ \\
Site AC power  & $P_{\mathrm{site}}$ &  ${\mathrm{MW}}$ & $129$ & $163$ & $300$ \\
Site length & $L_{\mathrm{site}}$ &  ${\mathrm{km}}$ & $20.5$ & $31$ & $40$ \\
\end{tabular}
\caption{Summary table of the ILC accelerator parameters in the initial $250\,{\mathrm{GeV}}$ staged configuration
and possible upgrades.
\label{tab:ilc-params}}
\end{table}

{\it
The collider design is the result of nearly twenty years of R\&D. 
The heart of the ILC, the superconducting cavities, is based on over a decade of pioneering work by the TESLA collaboration in the 1990s. 
Some other aspects were based on the R\&D carried out for the JLC/GLC and NLC projects, which were based on room-temperature accelerating structures. 
From 2005 to the publication of the Technical Design Report (TDR)~\cite{Adolphsen:2013kya} in 2013, the design of the ILC accelerator was conducted as a worldwide international collaboration coordinated by the Global Design Effort (GDE) under a mandate from the International Committee for Future Accelerators (ICFA).
Since then, the Linear Collider Collaboration (LCC) has been coordinating the international activities for both, the ILC and CLIC projects, again mandated by ICFA.
}

The fundamental goal of the design of the ILC is a high energy efficiency, which limits the overall power consumption of the accelerator complex during operation to $129\,{\mathrm{MW}}$ at  $250\,{\mathrm{GeV}}$ and $300\,{\mathrm{MV}}$ at  $1\,{\mathrm{TeV}}$, which is comparable to the power consumption of CERN.
% Stapnes at ALCW2018: 1.35TWh in 2012 -> 154MW on average in 2012
This is achieved by the use of SCRF technology for the main accelerator, which offers a high RF-to-beam efficiency through the use of superconducting cavities, operating at $1.3\,{\mathrm{GHz}}$, where high-efficiency klystrons are commercially available.
At accelerating gradients of $31.5$ to $35\,{\mathrm{MV/m}}$ this technology offers high overall efficiency and reasonable investment costs, even considering the cryogenic infrastructure needed for the operation at $2\,{\mathrm{K}}$.

The underlying TESLA technology is mature, with a broad industrial base throughout the world, and is in use at a number of free electron laser facilities that are in operation (European XFEL at DESY, Hamburg), under construction (LCLS-II at SLAC, Stanford) or in preparation (SCLF in Shanghai) in the three regions Asia, Americas, and Europe that have contributed to the ILC design.
In preparation for the ILC, Japan and the U.S.\ have founded a collaboration for further cost optimisation of the TESLA technology.
In recent years, new surface treatments during the cavity preparation process, such as the so-called nitrogen infusion, have been developed at Fermilab and elsewhere, with the prospect to achieve higher gradients and lower loss rates with a less expensive surface preparation scheme than assumed in the TDR.

When the Higgs discovery was imminent in 2012, the Japan Association of High Energy Physicists (JAHEP) made a proposal to host the ILC in Japan. 
Subsequently, the Japanese ILC Strategy Council conducted a survey of possible sites for the ILC in Japan, looking for  suitable geological conditions for a tunnel up to $50\,{\mathrm{km}}$ in length (as required for a $1\,{\mathrm{TeV}}$  machine), and the possibility to establish a laboratory where several thousand international scientists can work and live. 
As a result, the candidate site in the Kitakami region in northern Japan, close to the larger cities of Sendai and Morioka, was found to be the best option. 
The site offers a large, uniform granite formation with no currently active faults that is well suited for tunneling.
Even at the great Tohoku earthquake in 2011 underground installations in this rock formation were essentially unaffected, which underlines the suitability of this candidate site. 

Fig.~\ref{fig_ilc-schematic} shows a schematic overview over the accelerator with its main subsystems.
The accelerator extends over $20.5\,{\mathrm{km}}$, with two main arms that are dominated by the electron and positron main linacs, respectively, at an $14\,{\mathrm{mrad}}$ crossing angle.

 \begin{figure}[tb]
 %\epsfysize=9.0cm
 \begin{center}
 \includegraphics[width=\hsize]{figures/TDR-machine-layout-cartoon-staged.pdf}
\caption{Schematic layout of the ILC in the $250\,{\mathrm{GeV}}$ staged configuration.
\label{fig_ilc-schematic}}
 \end{center}
 \end{figure}

Electrons are produced by a polarised electron gun located in the tunnel of the positron beam delivery system. A Ti:sapphire laser impinges on a photocathode with a strained GaAs/GaAsP superlattice structure, which will provide  $90\,\%$, electron polarisation at the source, resulting in $80\,\%$ polarisation at the interaction point. The design is based on the electron source of the SLAC accelerator. 

Two concepts for positron production are considered:

The baseline solution employs superconducting helical undulators at the end of the electron main linac producing circularly polarised photons, which are converted in rotating titanium target to  positrons with a $30\,\%$ longitudinal polarisation.
This positron production scheme requires a fully commissioned electron linac delivering a beam close to $125\,{\mathrm{GeV}}$, which is a concern for commissioning and operation. 
%The main technical challenges of this concept are the target and the photon dump. 
%In addition, the undulator photon flux rapidly falls for electron energies below $125\,{\mathrm{GeV}}$, which is a concern for commissioning and operation.  
An alternative design, the electron driven source, utilises a dedicated S-band electron accelerator to provide a $3\,{\mathrm{GeV}}$ beam that is used to produce electrons.
% on a rotating titanium target. 
%Electrons are produced over a timespan of $66\,{\mathrm{ms}}$, reducing the target heat load and the necessary rotation speed.
%Technical challenges for this concept are radiation in the capture device and beam loading in the first accelerating cavities. 
This source would not provide positron polarisation,
but offer advantages for operation at lower electron beam energies and commissioning.
% as its operation is independent  from the main linac electron beam, operation at lower electron beam energies is possible and commissioning can begin in parallel to electron source commissioning.

Electrons and positrons are injected at $5\,{\mathrm{GeV}}$ into the centrally placed $3.2\,{\mathrm{km}}$ long damping ring complex, where their emittance is reduced to $2\,{\mathrm{pm}}$ ($5\,{\mathrm{\mu m}}$) in the horizontal (vertical) plane within $200\,{\mathrm{ms}}$. 
These emittance numbers are well in line with the performance of today's storage rings for advanced light sources.
To achieve the necessary damping time constant, the damping ring is equipped with $54$ superconducting wigglers. 

The damped beams are transported to the beginning of the main accelerator by two low emittance beam transport lines. Two bunch compressor stages at $5$ and $15\,{\mathrm{GeV}}$ reduce the longitudinal bunch length to $300\,{\mathrm{\mu m}}$ before the beams are accelerated to $125\,{\mathrm{GeV}}$ in the two main linacs.

The main linacs accelerate the beams in superconducting cavities made of niobium, operating at $1.3\,{\mathrm{GHz}}$ frequency and a temperature of $2.0\,{\mathrm{K}}$. 
Each cavity has $9$ cells and is $1.25\,{\mathrm{m}}$ long. The mean accelerating gradient will be $31.5$ to $35\,{\mathrm{MV/m}}$.
Cavities are mounted in $12\,{\mathrm{m}}$ long cryomodules that house $9$ cavities or $8$ cavities plus a quadrupole unit for beam focusing. 
The cryomodules provide cooling and thermal shielding to the cavities and cryomodules, and contain all necessary pipes for fluid and gaseous helium at various temperatures, so that no separate helium transport line is necessary.
Cryomodules of this type have been in continuous operation since 2000 in the TESLA Test Facility (TTF, now FLASH) at DESY,  proving their long-term stability. 
Since September 2017, the European X-FEL at DESY has been in operation, utilizing 97 of these cryo modules. 
Cost and performance estimates for the ILC cryomodules are based on the experience from these facilities, and thus can be regarded with high confidence. 

The radiofrequency (RF) power for the cavities is generated by commercially available $10\,{\mathrm{MW}}$ klystrons with an efficiency of $65\,\%$. 
The pulse modulators will be use a new, modular and cost-effective semiconductor design develped at SLAC, the MARX modulator.

The cryogenic system design foresees six cryo plants for the main linacs with a size similar to the plants for one LHC octant; 
two smaller plants would supply the central region including the preaccelerators of the sources and the damping rings. 

Finally, the beam delivery system focuses the beams to the required size of $516\,{\mathrm{\mu m}} \times 7.7\,{\mathrm{nm}} $. 
A feedback system, which profits from the relatively long inter bunch separation of $554\,{\mathrm{ns}}$, ensures the necessary beam stability. 
The necessary nano-beam technology has been tested at the Accelerator Test Facility 2 (ATF-2) at KEK.


The TDR baseline design was for a centre-of-mass energy of $\sqrt{s}=500\,{\mathrm{GeV}}$, upgradeable to a final energy of $1\,{\mathrm{TeV}}$.
After the discovery of the Higgs boson in 2013 interest grew for an accelerator operating as a ``Higgs factory'' at $\sqrt{s}=250\,{\mathrm{GeV}}$, slightly above the maximum for $Zh$ production. 
The design for a $250\,{\mathrm{GeV}}$ version of the ILC has recently been presented in a  staging report by the LCC directorate~\cite{Evans:2017rvt} and was endorsed by ICFA.

The staged ILC version of the ILC  
% (in the preferred ``option A'') 
would have two main linac tunnels about half the length ($6$ instead of $11\, {\mathrm{km}}$) of the $500\,{\mathrm{GeV}}$ TDR design. 
Other systems, in particular the beam delivery system and the main dumps, would retain the dimensions of the TDR design.
Thus, the ILC-250 could be upgraded to energies of $500\,{\mathrm{GeV}}$ or even $1\,{\mathrm{TeV}}$ with a reasonable effort, without extensive modifications to the central region. 
Recent studies of rock vibrations from tunnel excavation in granite rock indicate that the necessary additional main linac tunnels could be largely constructed with the ILC running, so that an energy upgrade could be realised with an interruption in data taking of a couple of years only, compatible with a smooth continuation of the physics programme.

Another upgrade option, which could come before or after an energy upgrade, is a luminosity upgrade. 
Doubling the luminosity by doubling number of bunches per pulse to $2625$ at a reduced bunch separation of $366\,{\mathrm{ns}}$ would require $50\,\%$ more klystrons and modulators and an increased cryogenic capacity. 
The damping rings would also permit an increase of the pulse repetition rate from $5$ to $10\,{\mathrm{Hz}}$. 
This would require a significant increase in cryogenic capacity, or running at a reduced gradient after an energy upgrade.
% Both paths to a luminosity increase would pose large  stress for the positron source target and probably require its redesign, which would be a significant engineering challenge, but not a cost driver.
The projections for the physics potential of the ILC at $250\,{\mathrm{GeV}}$ are based on a total integrated luminosity of $2\,{\mathrm{ab}}^{-1}$, which assumes at least one luminosity upgrade.

The total project cost for the construction of the $250\,{\mathrm{GeV}}$ accelerator is estimated to be $5,260\,{\mathrm{MILCU}}$, where an ILCU (ILC Currency Unit) is defined as $1\,{\mathrm{US\$}}$ in 2012 terms, plus $17\,{\mathrm{Mh}}$ of work. 
These numbers include the cost for civil engineering and the laboratory, while costs for land acquisition are exempt, as are R\&D costs before the construction start, and costs for the detectors.
The estimate is an estimate of the median project cost, without contingency or management reserve.
The cost premium to cover the project cost with $85\,\%$ instead of $50\,\%$ confidence level (loosely speaking, the $1\,\sigma$ uncertainty of the cost estimate) has been estimated to be $23\,\%$ of the estimated cost.

The construction of the accelerator will require nine years after ground breaking, preceded by a four year preparation phase.

\section{\label{sec:detect}Detectors and R\&D}

2 pages - Ties, Andy


The science at the ILC drives the requirements on detectors. The main factors are:
\begin{itemize}
 \item The detector has to have an excellent track momentum
   resolution. The benchmark reaction here is the analysis 
of the di-lepton mass in the process $HZ \to H \ell^+
\ell^-$. This reaction allows the reconstruction of the 
Higgs mass independent of its decay mode via the 
reconstruction of the lepton recoil spectrum. In order that 
the momentum resolution of the detector does not limit 
the mass resolution achievable for the recoiling lepton 
system, stringent momentum resolution requirements have to be met. 
\item The reconstruction of the flavour of the final state can 
often be done best with the help of lifetime information of the 
decaying particles. For this, very powerful vertex detectors 
are needed. This is particularly important 
in the Higgs sector, where -- at least for light Higgs bosons -- 
a large fraction of the Higgs decays has bottom 
quarks in the final state. Many other physics signatures will 
produce complex final states with bottom or charm quarks as well. 
A supreme vertex detector therefore is needed to reconstruct these 
long lived particles with excellent resolution. 
\item The overall event is best reconstructed with the 
particle flow measurement. The particle flow technique combines 
the information from the tracking systems and from the 
calorimetric systems in an attempt to reconstruct the 
energy and the direction of all charged and 
neutral particles in the event. To minimize overlaps between 
neighboring particles, and to maximize the probability to 
correctly combine tracking and calorimeter information, 
excellent calorimeters are needed with very high granularity. 
\item Many physics signatures predict some undetectable particles, 
which escape from the detector. They can only be reconstructed by 
measuring the missing energy in the event. This requires 
that the detector is as hermetic as possible, to 
minimize the amount of energy that can escape detection. 
Particular care has to be given to the region surrounding the 
beampipe in the forward direction. 
\end{itemize}

Compared to the last large scale detector project in particle physics, the construction and upgrade of the LHC detectors, the emphasis for linear collider detector is shifted towards ultimate precision. This requires detcetor technologues which are driven towards ultimate precision, and this requires a minimisation of dead material in the detector, at an unprecedented level. This also requires a management and control of services and in particular a thermal management of the detector concept. Significant technological R\& D was needed to demonstrate the feasibility, and is, in fact, still ongoing, as will be discussed in the next section.  

Over the last decade two detector concepts have emerged from the discussions in the community. Both are based on the assumption that particle flow reconstruction plays a central role in the event reconstruction. Both therefore have highly granular calorimeters, placed inside the coil which is providing the central magnetic field. Both have excellent trackers and vertexing systems. The two approaches differ in the choice of tracker technology, and in the approach taken to maximize the overall precision of the event reconstruction. ILD has chosen a gaseous central tracker, a time projection chamber, combined with silicon detectors inside and outside the TPC. SiD relies on an all Silicon solution, similar to the LHC detectors. ILD tries to optimize the particle flow resolution by making the detector large, thus separating charged and neutral particles. SiD keeps the detector more compact, and compensates the reduced particle separation at the position of the calorimeters by using a higher central magnetic field. Both approaches have demonstrated excellent performance, meeting or even exceeding the performance requirements. 

For both detector concepts, communities have found themselves and pre-collaborations have formed. These organizations have over the last ten years or so pushed both concepts to a remarkable level of maturity, and have, in close interaction with the different groups performing detector R\&D from around the world, demonstrated the feasibility to build and operate such high precision detectors. 

European groups have played a central role in these efforts. The ILD concept group is formed from some 70 groups from around the world, with more than half coming from Europe. The SiD collaboration has a strong basis in the Americas, but also relies on significant participation from European groups. Major contributions to the development of all sub-systems have come from Europe. Significant technological breakthroughs for example in the area of highly granular calorimeter are strongly driven by European groups. 

An important aspect of the detector concept work has been in addition to the development and demonstration of the technology the integration of the detector into the collider and into the proposed site. The location of the experiment in an earth-quake prone area poses challenges which have been addressed through R\& D on detector stability, support and service. The scheme to operate two detectors in one interaction region, the so called push-pull scheme, has no example and needed significant engineering work to demonstrate its feasibility. With strong support from particle physics laboratories in Europe, in  particular DESY and CERN, many of the most relevant questions could be answered and the feasibility of the approach could be demonstrated at least in principle. 



\section{\label{sec:soft}Software and Computing}

[1 page - Frank G. and Akiya

  Description of ILC software and computing requirements]

The excellent and unprecedented detector resolution of the two proposed ILC detector concepts described
in the previous section is of course crucial for meeting the ambitious physics goals
defined for the ILC programme. However achieving the ultimate physics performance is only possible if it
is complemented with powerful software tools at all levels of the data processing. This starts with the
detailed detector description for simulation, followed by sophisticated event reconstruction algorithms
and is completed with high-performance and easy to use tools for physics analyses.
For over a decade the ILC community has developed and improved its software ecosystem, called \emph{iLCSoft}.
At the heart of \emph{iLCSoft} are the common event data model LCIO, the Marlin application framework and the
DD4hep detector description toolkit. These tools provide the basis for almost all linear collider related
studies carried over the years, including those for CLIC and also recently for CEPC.
From the start a strong focus has been put on developing flexible and generic tools that can easily be applied
to other experiments or new detector concepts. This approach of developing common tools wherever possible
has helped a lot in leveraging the limited manpower and putting the focus on the algorithm development that
is crucial for the physics performance. Recent developments evolving around the Hep Software Foundation are
now following a similar strategy, also for the LHC experiments.
For both detectors there are simulation models with realistic descriptions of the detector technologies,
the dead material, gaps, imperfections, as well as detector services. These models have been used for large
scale Monte Carlo productions and physics analyses for the TDR and more recent detector optimization
campaigns. Based on these studies, we already now have a rather good understanding of the expected
detector performance and physics reach of the ILC.
The development of \emph{iLCSoft} is a truly international activity, where European groups, in particular DESY and
CERN have played a leading role and should continue to do so, if the ILC will be approved. In this case
a strong focus will be on adapting the software tools for modern hardware architectures and continue to
further improve the computing and physics performance of the algorithms.


  
An initial computing concept for the ILC, including a first estimate of the required resources, has been developed by the LCC Software and Computing Group.
This concepts follows in general terms that of other ongoing experiments at the LHC and Belle-2 with a strong on-site computing center, complemented with large
Grid-based computing resources distributed around the world. However, due to he much lower event rates at the ILC, compared to the LHC, we will be
able to run in an un-triggered mode where collision data from every bunch crossing will be recorded. At the experiment site we require only limited computing
resources for online monitoring, QA and data buffering capabilities for a few days. Prompt reconstruction, event building and filtering of the interesting collisions
will be performed at the main ILC campus, resulting in a data reduction of around 97\%. The reminder of the data will be distributed to major participating Grid sites
in the world for further skimming and final redistributed for physics analysis.
Based on detailed physics and background simulations we estimate the total raw data rate of the ILC to be $\sim$1.5GB/s, resulting in about 10-15 PB/y storage needs.
The estimated computing power that will be needed for simulation, reconstruction and analysis will be around 200-300 kHepSpec06.
Given that these numbers are already smaller than what is needed by the LHC experiments today and an expected annual increase of 15\% and 25\% for storage and CPU
respectively at flat budget, we expect the overall computing costs for the ILC to be more than an order of magnitude small than those for the LHC.

\section{\label{sec:discuss}Discussion}

1 page - Keisuke, Jim and Juan

    Discussion of HEP community interest and support, political progress, plan for international realization

\section{\label{sec:sum}Summary \& Conclusion} 

1 page


\begin{thebibliography}{99}
\bibitem{Behnke:2013xla} 
  T.~Behnke {\it et al.},
%  ``The International Linear Collider Technical Design Report - Volume 1: Executive Summary,''
  arXiv:1306.6327 [physics.acc-ph].
  %%CITATION = ARXIV:1306.6327;%%
  %224 citations counted in INSPIRE as of 24 Sep 2017


%\cite{Evans:2017rvt}
\bibitem{Evans:2017rvt}
  L.~Evans {\it et al.} [Linear Collider Collaboration],
  %``The International Linear Collider Machine Staging Report 2017,'' 
  arXiv:1711.00568 [physics.acc-ph].
  %%CITATION = ARXIV:1711.00568;%%

\bibitem{ILCforESS}

Our long report in preparation  (2018). 

\bibitem{LHCHiggssummary}

LHC Higgs summary

\bibitem{Lepage:2014fla} 
  G.~P.~Lepage, P.~B.~Mackenzie and M.~E.~Peskin,
 % ``Expected Precision of Higgs Boson Partial Widths within the Standard Model,''
  arXiv:1404.0319 [hep-ph].

\bibitem{Barklow:2017suo} 
  T.~Barklow, K.~Fujii, S.~Jung, R.~Karl, J.~List, T.~Ogawa, M.~E.~Peskin and J.~Tian,
  %``Improved Formalism for Precision Higgs Coupling Fits,''
  Phys.\ Rev.\ D {\bf 97},  053003 (2018)
 % doi:10.1103/PhysRevD.97.053003
  [arXiv:1708.08912 [hep-ph]].
 
\bibitem{Fujii:2017vwa} 
  K.~Fujii {\it et al.},
  %``Physics Case for the 250 GeV Stage of the International Linear Collider,''
  arXiv:1710.07621 [hep-ex].


\bibitem{H2aaLHC}
ATLAS Collaboration, ATL-PHYS-PUB-2014-016 (2014).

\bibitem{Liu:2016zki} 
 Z.~Liu, L.~T.~Wang and H.~Zhang,
  %``Exotic decays of the 125 GeV Higgs boson at future $e^+e^-$ lepton colliders,''
  Chin.\ Phys.\ C {\bf 41}, no. 6, 063102 (2017)
 % doi:10.1088/1674-1137/41/6/063102
  [arXiv:1612.09284 [hep-ph]].


\bibitem{Higgsino}

Higgsino at ILC

\bibitem{eetobb}

paper on $\ee\to b\bar b$



%\cite{Adolphsen:2013kya}
\bibitem{Adolphsen:2013kya}
  C.~Adolphsen {\it et al.},
  %``The International Linear Collider Technical Design Report - Volume 3.II: Accelerator Baseline Design,''
  arXiv:1306.6328 [physics.acc-ph].
  %%CITATION = ARXIV:1306.6328;%%



\end{thebibliography}



\end{document}
%
% ****** End of file apssamp.tex ******

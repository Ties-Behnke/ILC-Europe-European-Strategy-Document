% ****** Start of file apssamp.tex ******
%
%   This file is part of the APS files in the REVTeX 4.1 distribution.
%   Version 4.1r of REVTeX, August 2010
%
%   Copyright (c) 2009, 2010 The American Physical Society.
%
%   See the REVTeX 4 README file for restrictions and more information.
%
% TeX'ing this file requires that you have AMS-LaTeX 2.0 installed
% as well as the rest of the prerequisites for REVTeX 4.1
%
% See the REVTeX 4 README file
% It also requires running BibTeX. The commands are as follows:
%
%  1)  latex apssamp.tex
%  2)  bibtex apssamp
%  3)  latex apssamp.tex
%  4)  latex apssamp.tex
%
\documentclass[%
 reprint,
%superscriptaddress,
%groupedaddress,
%unsortedaddress,
%runinaddress,
%frontmatterverbose, 
%preprint,
%showpacs,preprintnumbers,
%nofootinbib,
%nobibnotes,
%bibnotes,
 amsmath,amssymb,
 aps,
%pra,
%prb,
%rmp,
%prstab,
%prstper,
%floatfix,
]{revtex4-1}

\usepackage{graphicx}% Include figure files
\usepackage{dcolumn}% Align table columns on decimal point
\usepackage{bm}% bold math
%\usepackage{hyperref}% add hypertext capabilities
%\usepackage[mathlines]{lineno}% Enable numbering of text and display math
%\linenumbers\relax % Commence numbering lines

%\usepackage[showframe,%Uncomment any one of the following lines to test 
%%scale=0.7, marginratio={1:1, 2:3}, ignoreall,% default settings
%%text={7in,10in},centering,
%%margin=1.5in,
%%total={6.5in,8.75in}, top=1.2in, left=0.9in, includefoot,
%%height=10in,a5paper,hmargin={3cm,0.8in},
%]{geometry}

\begin{document}

\preprint{LCCPEB---}

\title{The International Linear Collider}% Force line breaks with \\
\thanks{Version 1}%

\author{Juan Fuster}
% \altaffiliation[Also at ]{Physics Department, XYZ University.}%Lines break automatically or can be forced with \\
\author{etal}%
 \email{Second.Author@institution.edu}
\affiliation{%
 Authors' institution and/or address\\
% This line break forced with \textbackslash\textbackslash
}%

\collaboration{LCC Collaboration}%\noaffiliation

\date{\today}% It is always \today, today,
             %  but any date may be explicitly specified

\begin{abstract}
Input from the International Linear Collider community for the European Strategy Update 

\end{abstract}

\pacs{Valid PACS appear here}% PACS, the Physics and Astronomy
                             % Classification Scheme.
%\keywords{Suggested keywords}%Use showkeys class option if keyword
                              %display desired
\maketitle

%\tableofcontents

\section{\label{sec:intro}Introduction}

Steinar, Jim, Juan, 1.5 pages

\section{\label{sec:acc}Accelerator}

Phil, Marcel, Steinar, 2 pages

\section{\label{sec:det}Detector}
Marcel, Ties, 2 pages

The science at the ILC drives the requirements on detectors. The main factors are:
\begin{itemize}
 \item The detector has to have an excellent track momentum
   resolution. The benchmark reaction here is the analysis 
of the di-lepton mass in the process $HZ \to H \ell^+
\ell^-$. This reaction allows the reconstruction of the 
Higgs mass independent of its decay mode via the 
reconstruction of the lepton recoil spectrum. In order that 
the momentum resolution of the detector does not limit 
the mass resolution achievable for the recoiling lepton 
system, stringent momentum resolution requirements have to be met. 
\item The reconstruction of the flavour of the final state can 
often be done best with the help of lifetime information of the 
decaying particles. For this, very powerful vertex detectors 
are needed. This is particularly important 
in the Higgs sector, where -- at least for light Higgs bosons -- 
a large fraction of the Higgs decays has bottom 
quarks in the final state. Many other physics signatures will 
produce complex final states with bottom or charm quarks as well. 
A supreme vertex detector therefore is needed to reconstruct these 
long lived particles with excellent resolution. 
\item The overall event is best reconstructed with the 
particle flow measurement. The particle flow technique combines 
the information from the tracking systems and from the 
calorimetric systems in an attempt to reconstruct the 
energy and the direction of all charged and 
neutral particles in the event. To minimize overlaps between 
neighboring particles, and to maximize the probability to 
correctly combine tracking and calorimeter information, 
excellent calorimeters are needed with very high granularity. 
\item Many physics signatures predict some undetectable particles, 
which escape from the detector. They can only be reconstructed by 
measuring the missing energy in the event. This requires 
that the detector is as hermetic as possible, to 
minimize the amount of energy that can escape detection. 
Particular care has to be given to the region surrounding the 
beampipe in the forward direction. 
\end{itemize}

Compared to the last large scale detector project in particle physics, the construction and upgrade of the LHC detectors, the emphasis for linear collider detector is shifted towards ultimate precision. This requires detcetor technologues which are driven towards ultimate precision, and this requires a minimisation of dead material in the detector, at an unprecedented level. This also requires a management and control of services and in particular a thermal management of the detector concept. Significant technological R\& D was needed to demonstrate the feasibility, and is, in fact, still ongoing, as will be discussed in the next section.  

Over the last decade two detector concepts have emerged from the discussions in the community. Both are based on the assumption that particle flow reconstruction plays a central role in the even reconstruction. Both therefore have highly granular calorimeters, placed inside the coil which is providing the central magnetic field. Both have excellent trackers and vertexing systems. The two approaches differ in the choice of tracker technology, and in the approach taken to maximize the overall precision of the event reconstruction. ILD has chosen a gaseous central tracker, a time projection chamber, combined with silicon detectors inside and outside the TPC. SiD relies on an all Silicon solution, similar to the LHC detectors. ILD tries to optimize the particle flow resolution by making the detector large, thus separating charged and neutral particles. SiD keeps the detector more compact, and compensates the reduced particle separation by using a higher central magnetic field. Both approaches have demonstrated excellent performance, meeting or even exceeding the performance requirements. 

For both detector concepts communities have found themselves and pre-collaborations have formed. These organizations have over the last ten years or so pushed both concepts to a remarkable level of maturity, and have, in close interaction with the different groups performing detector R\&D from around the world, demonstrated the feasibility to build and operate such high precision detectors. 

European groups have played a central role in these efforts. The ILD concept group is formed from some 70 groups from around the world, with more than half coming from Europe. Major contributions to the development of all sub-systems have come from Europe. Significant technological breakthroughs for example in the area of highly granular calorimeter are strongly driven by European groups. 

An important aspect of the detector concept work has been in addition to the development and demonstration of the technology the integration of the detector into the collider and into the proposed site. The location of the experiment in an earth-quake prone area poses challenges which have been addressed through R\&D on detector stability, support and service. The scheme to operate two detectors in one interaction region, the so called push-pull scheme, has no example and needed significant engineering work to demonstrate its feasiblity. With strong support from particle physics laboboratories in Europe, in  particular DESY and CERN, many of the most relevant questions could be answered and the feasibility of the approach could be demonstrated at least in principle. 


\section{\label{sec:RandD}R and D}

Marc 1 page

\section{\label{sec:discussion}Discussion}

Marc, Ties, Phil, 2.5 pages


\section{This is a test}

\end{document}
%
% ****** End of file apssamp.tex ******

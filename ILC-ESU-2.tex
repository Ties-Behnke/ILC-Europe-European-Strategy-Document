% ****** Start of file apssamp.tex ******
%
%   This file is part of the APS files in the REVTeX 4.1 distribution.
%   Version 4.1r of REVTeX, August 2010
%
%   Copyright (c) 2009, 2010 The American Physical Society.
%
%   See the REVTeX 4 README file for restrictions and more information.
%
% TeX'ing this file requires that you have AMS-LaTeX 2.0 installed
% as well as the rest of the prerequisites for REVTeX 4.1
%
% See the REVTeX 4 README file
% It also requires running BibTeX. The commands are as follows:
%
%  1)  latex apssamp.tex
%  2)  bibtex apssamp
%  3)  latex apssamp.tex
%  4)  latex apssamp.tex
%
\documentclass[%
 reprint,
%superscriptaddress,
%groupedaddress,
%unsortedaddress,
%runinaddress,
%frontmatterverbose, 
%preprint,
%showpacs,preprintnumbers,
%nofootinbib,
%nobibnotes,
%bibnotes,
 amsmath,amssymb,
 aps,
%pra,
%prb,
%rmp,
%prstab,
%prstper,
%floatfix,
]{revtex4-1}

\usepackage{graphicx}% Include figure files
\usepackage{dcolumn}% Align table columns on decimal point
\usepackage{bm}% bold math
%\usepackage{hyperref}% add hypertext capabilities
%\usepackage[mathlines]{lineno}%  Enable numbering of text and display math
%\linenumbers\relax % Commence numbering lines

%\usepackage[showframe,%Uncomment any one of the following lines to test 
%%scale=0.7, marginratio={1:1, 2:3}, ignoreall,% default settings
%%text={7in,10in},centering,
%%margin=1.5in,
%%total={6.5in,8.75in}, top=1.2in, left=0.9in, includefoot,
%%height=10in,a5paper,hmargin={3cm,0.8in},
%]{geometry}
\usepackage{acronym}

\begin{document}

\preprint{LCCPEB---}

\title{The International Linear Collider\\ An European Perspective}% Force line breaks with \\
\thanks{Version 1.3}%

\author{Juan Fuster}
% \altaffiliation[Also at ]{Physics Department, XYZ University.}%Lines break automatically or can be forced with \\
\author{etal}%
 \email{Second.Author@institution.edu}
\affiliation{%
 Authors' institution and/or address\\
% This line break forced with \textbackslash\textbackslash
}%

\collaboration{LCC Collaboration}%\noaffiliation

\date{\today}% It is always \today, today,
             %  but any date may be explicitly specified

\begin{abstract}
Input from the International Linear Collider community for the European Strategy Update

\end{abstract}

\pacs{Valid PACS appear here}% PACS, the Physics and Astronomy
                             % Classification Scheme.
%\keywords{Suggested keywords}%Use showkeys class option if keyword
                              %display desired
\maketitle

%\tableofcontents

\section{\label{sec:intro}Introduction}

Steinar, Jim, Juan, 1.5 pages

\section{\label{sec:acc}Accelerator}

Europe has a very strong scientific, technological and industrial basis
to make significant contributions to the construction of virtually any part of the ILC accelerator. 
(list activities ... )

\subsection{ILC accelerator competence in Europe~\label{sec:competence:accelerator}}

During 2007-2012 the ILC GDE (Global Design Effort) was responsible for the coordination of the worldwide ILC accelerator design. R\&D and cost-estimate development during this period culminated in the publication of the ILC TDR (ref to abbrev) in 2013~\cite{ILC-TDR}. After the publication of the ILC TDR, the GDE was replaced by the Linear Collider Collaboration (LCC). 

The estimated European FTE contributions (728 person years in total) during 2007-2012 is summarised in figure~\ref{fig:PrePrep:ilcgde4}, 
divided into accelerator domain issues and SCRF (ref to abbrev) technology development (excluding management and documentation support). Note that the SCRF numbers represent ILC-specific resources and do not include the extensive synergetic contributions from the European XFEL.


\begin{figure}[htbp]
\includegraphics[width=0.45\textwidth]{figures/EU-GDE-FTE-columns-per-country.pdf}
\caption{\label{fig:PrePrep:ilcgde4} The contribution to the ILC GDE  (2007-2012) in staff years per European country, separately for SCRF and accelerator domain studies.}
\end{figure}

After the delivery of the ILC TDR in 2013, the ILC activitites have continued and the present European activities under the LCC umbrella that are carried out in close collaboration with Japan are summarised in Table~\ref{fig:PrePrep:gdelccadi}.

\begin{table}[htbp]
\includegraphics[width=0.45\textwidth]{figures/ILCEAP-Matrices-LCCADI.pdf}
\caption{\label{fig:PrePrep:gdelccadi} Current common studies between European institutions and Japan relevant for the ILC.}
\end{table}

Of highest relevance to the ILC is the 17.5~GeV superconducting linac at the European XFEL at DESY, comprising 100 superconducting ILC-like cryomodules (800 1.3~GHz TESLA cavities) and driven by 25 10~MW multibeam klystrons. The XFEL 
linac configuration is very similar to that foreseen for the ILC and can be 
seen as a 7\% (in energy) ILC prototype. The cryomodules were produced by a 
consortium of six countries together with predominantly European industries. The consortium members and the various responsibilities across the linac-relevant XFEL work packages (including testing) are given in Table~\ref{tab:PrePrep:XFELResponsibilities}.

\begin{table}[htbp]
\begin{center}
\includegraphics[width=0.45\textwidth]{figures/ILCEAP-Matrices-XFEL.pdf}
\caption{\label{tab:PrePrep:XFELResponsibilities} Responsibility matrix for cryomodule production and testing for the European XFEL.} 
% Edited 20.07.2017 by NJW. Removed this comment as it is now covered in the ESS section/table
%(Note that only cryomodule-relevant work packages are shown. Note also that including projects such as \acs{ESS} will bring additional European 
%institutes into play for \acs{SCRF}; see Section~\ref{sec:pastpresent:ESS
\end{center}
\end{table}

Construction of the European XFEL is now complete, and the SCRF linac has been brought into operation. The XFEL can directly benefit the ILC in the following key ways:
\begin{itemize}
\item The experience and knowledge gained during the unprecedented industrial 
production of 100 cryomodules over a three-year period can provide 
invaluable input to any future large-scale production for the ILC, including 
identifying directions where further R\&D could be beneficial for cost reduction 
and performance enhancement (e.g.\ more cost-effective approaches to mass 
production). The detailed cost breakdown of the XFEL cryomodules provides a 
solid basis for any future projection of a possible European in-kind contribution 
to the ILC.
\item The currently ongoing commissioning of the XFEL and its future operation will 
provide invaluable ``system testing'' for the ILC, including understanding the 
ultimate performance of the modules with beam loading, beam control (LLRF 
development), software tools, and more general operational experience. 
Furthermore, the XFEL could provide a test-bed for ILC-relevant accelerator 
experiments, although the time available for this will be limited once user 
operation is in full swing.
\item The infrastructure that was constructed for XFEL cavity and module 
testing, high-power coupler conditioning and module assembly will continue to be 
maintained, and (in the case of the testing infrastructure) will provide a 
significant support for SCRF R\&D.
\end{itemize}

Add a few sentences about ESS ... 


\subsection{ILC accelerator preparation phase activities in Europe ~\label{sec:prepphase:accelerator}}

The overall resources needed during the four-year preparation phase are estimated to be 5\% of the material and 10\% of the personnel foreseen for the initial 250 GeV accelerator project construction. Irrespective of the final level of European investment into the ILC, and to allow the preparation phase to begin after an official expression of interest from Japan and support by the European strategy, it would be appropriate that, given its expertise and previous involvement, Europe invests 1/3 of the overall effort required in the preparation phase. The remainder of this section assumes this to be the case. This would then amount to a total European material budget of 85 M\Euro and 240 FTE-years, integrated over the period.
Distributed over four years, an average yearly budget of around 30 M\Euro{} (covering material and personnel) would be needed with an increasing profile from 2019 towards 2022. The resources required for the activities in the preparation phase are hence similar in scope to those used by existing project studies such as CLIC and FCC. One important difference will be that the ILC preparation requires a strong engineering team, such that the profile of the ILC personnel will be towards more engineering and technical design and gradually less focused on R&D studies.

The preparation phase will be used for three main purposes:

\begin{description}

\item{\bfseries Technical developments}
Technical preparation of the major European deliverables foreseen for the construction phase. This covers final technical specifications, final prototypes, the preparation of pre- series orders and the preparation of local facilities. A particularly important point for Europe is the transfer of European XFEL know-how and the preparation of the relevant facilities for ILC construction.
\item{\bfseries European design office}
The other key technical activity will be the organisation of a strong European design office for ILC that will liaise with other such offices: there will certainly be a host-lab office in Japan, but additional international design offices will be required. In Europe, the installation of a central European Design Office at CERN with satellite offices in other European laboratories is considered the most viable model.
\item{\bfseries Prepare construction responsibilities and governance}
• The third key activity in the preparation phase will be negotiations about the final European ILC contributions, about the organisation of the project in the construction and operation phase, and about a future governance model for the ILC. These issues are discussed in further chapters of this document.
\end{description}

The key activities are summarised in table~\ref{fig:prep-phase-summary} - unify format.


\begin{table}[htbp]
\includegraphics[width=0.45\textwidth]{figures/prep-phase-summary.pdf}
\caption{\label{fig:prep-phase-summary} Key activities during the preparation phase}
\end{table}


\subsection{ILC accelerator in-kind contributions from Europe ~\label{sec:constrphase:accelerator}}

Discuss 1/3 of non CFS (use figure included) 

\begin{figure}[htbp]
\begin{center}
\includegraphics[width=0.45\textwidth]{figures/eap-chp3-ilccostdrivers.pdf}
 \caption{\label{fig:constructionmodel:ILCPrimaryCostDrivers} Primary cost drivers for the ILC (breakdown based on ILCU).}
\end{center}
\end{figure}
Discuss European matrix (needs work on a table or two - WE DON'T HAVE SUITABLE TABLE) 

Discuss profile (use figure included - needs re-check)

As discussed in Section II B, an assumed 5\% of the total value and 10\% of the personnel are assumed
to ramp up during the four-year preparatory phase. The profiles also includes a fraction
of the expected lab services personnel, which will almost certainly be required during this
phase. 

\begin{figure}[htbp]
\includegraphics[width=0.45\textwidth]{figures/profile-250GeV-MEUR-norm.pdf}
\includegraphics[width=0.45\textwidth]{figures/profile-250GeV-FTE-norm.pdf}
\caption{\label{fig:costprofile:costprofile} An estimation of the cost and personnel profile cover the preparatory phase (years -3 to 0) 
and the construction phase (years 1 to 8) for the 1/3 model of a 250 GeV machine, as a fraction of the
totals. In this timeline, 
year 1 corresponds to the first year of construction currently foreseen in 2023 
Left: capital costs. Right: explicit personnel in FTE.
}
\end{figure}

Add something about organization (maybe in conclusion).

\section{\label{sec:det}Detector}
Marcel, Ties, 2 pages

Detectors at colliding beam facilities are international efforts, realised by strong collaborations. Typically many countries join forces, and contribute mostly in-kind to the building of the detector systems. 

Within the ILC project, the community has come together behind two complementary detector concepts, the ILD and the SID detector concept. After a call for ``letters of intent'' in 2009, the two detector concept groups submitted their proposals which were validated and then presented in Volume 4 of the ILC TDR ~\cite{ILC-TDR}.

Europe has participated strongly in the work on the ILC concepts and related technological developments. Initiated by the TESLA project at the start of the century, an international effort was put into place to develop the needed detector technologies. This is augmented by strong national support, but also by European programs like the EUDET initiative (2006-2010), the AIDA program (2011-2015) or the AIDA 2020 program (2016-2019). Programs in Asia, in particular in Japan, and through to a lesser extend, in the Americas were put into place. 

The technological developments will be described in more detail in the next section. The detectors at the ILC are very different from the LHC ones in that they can focus much more strongly on precision measurements. This is possible since the environment at an electron positron collider is much more benign than at a hadron collider. Radiation hardness hardly plays a role in designing the detectors. Multiple interactions, which are a major challenge at the LHC, are essentially not present. These different boundary conditions allow a detector optimisation complementary to the one at the LHC. A focus on detailed event reconstruction, on high precision vertexing and high precision high efficiency tracking and highly detailed imaging calorimetry is possible. As a guiding principle Particle Flow has been widely accepted and is used by both SiD and ILD as a major indicator of the performance of the complete systems. 

European groups have played strong roles in nearly all areas of detector design and development for the ILC. A core element of ILC detectors, particle flow calorimeters, has been strongly pushed by European groups. Many developments on high precision vertexing, in particular in the area of MAPS technology, have come from European groups. The development of high precision gaseous detectors based on micro-pattern gaseous detectors has been dominated by European groups. Last but not least Europe has played a central role in the integration of one of the two detector concepts, the ILD collaboration.  

\subsection{Detector preparations~\label{sec:prepphase:detectors}}
The community is well prepared to move forward quickly once the ILC turns into a real project. We anticipate that the detector construction will follow a path similar to that of the LHC detectors. Once the ILC laboratory is formed, detector collaboration will formally start, and develop based on the existing work concrete proposals for detectors at the ILC. To govern and organise this, strong central laboratory support will be essential. The host country would have to play a special role in this, but strong regional centers will also be very important.  In Europe CERN would be a natural candidate for such a regional center, but national laboratories like DESY or others could also play this role for parts of the detector. Examples for such more regional centers within the LHC are the Detector Assembly Facility at DESY, used for the construction of parts of the upgrades to the LHC detectors, the CMS center at Fermilab in the United States, or CERNs neutrino platform towards a European role in Dune. 

Different from the accelerator, the collaborations themselves will define and institute the needed structures to design and build the detectors. Strong support from the host organisation on all issues of interfaces between the detectors and the accelerator is however essential. 

During the preparation phase, there are four major milestones for the
detectors.
\begin{itemize}

\item{\bfseries Optimisation} After the approval of the ILC the design of the existing detector concepts will need to be reviewed and refined. We expect that the community supporting these detectors will grow substantially. The design will also profit from the recent experiences of the LHC upgrade program, and other major detector contruction projects. 

\item{\bfseries Integration into the ILC}
With the choice of the final location of the interaction region of the ILC, the detector designs need to be adapted to the
conditions of the site in terms of hall size, transport capabilities and
assembly space. Even though sigificant preparatory work on these questions is already happening, much more will be required. In particular the experience from Europe and from CERN will be invaluable. 

 
\item{\bfseries Prototyping and Validation}
The ILC detectors have already reached an impressive level of maturity - more so probably than at any other large scale HEP project at a similar stage in the past. Nevertheless the final designs and decisions will need a very vigorous and high quality testing and prototyping program, to demonstrate the technical readiness, and to provide the basis for final technology decisions. This will require access to test beam and testing infrastructures in Europe and beyond, and will put particular demands on the test beam installations at CERN and at DESY. This process will continue well beyond the stage of the technical design report and approval of detectors, into the construction phase.

\item{\bfseries Technical design report}
At the end of the preparatory phase the ILC detector concepts will present detailed technical design reports to the community and the funding bodies. Based on these reports the final decisions on which detectors will be build will be taken. 

\end{itemize}

\subsection{\label{Section:constructionmodel:ILCDet} Estimation of a
contribution to the ILC detectors}

In high-energy physics, the financial contribution to the detector
construction and operation
is typically assumed to be proportional to the number of authors of
the detector collaboration. In the LHC experiments, Europe accounts for about 50\% of the members, in non-European experiments like Belle-II European groups contribute about a 1/3 share. Most likely an experiment at the ILC would be close to a Belle-like model, though, given the large interest in the community, possibly somewhere in between Belle and the LHC. 

Taking
the cost numbers from
the ILC TDR~\cite{ILC-TDR}, the European share for the detector
construction is approximately 270 M\Euro{}, which is comparable to the cost for the ongoing upgrade of the LHC detectors. At the moment some 45 institutions from 14 European countries have expressed an interest to participate in ILC related detector work, a number, which is likely to increase after the ILC approval.

\section{\label{sec:RandD}R\&D strategy and achievements}

The capacity to reconstruct with extreme precision the characteristics of the final states produced at the ILC is mandatory to take advantage of the inherently accurate knowledge of the scattering conditions of the beam particles. Moreover, the relatively mild running conditions allow alleviating the constraints on radiation tolerance and read-out speed. The R\&D related to each sub-system was therefore driven by a challenging and innovative trade-off to be found between very demanding resolution (granularity) and material budget requirements on the one hand, and an acceptable speed and power consumption on the other hand. Moreover, the detector steering and read-out architectures obey specific conditions: they should be operated triggerless and designed to exploit the machine duty cycle and bunch spacing to mitigate the power consumption.
In several cases, individual performances targeted by the R\&D could have been considered as nearly achieved outside of the ILC programme, but intensive R\&D was needed to realise their combination at a level well beyond previous achievements, including detailed system integration aspects. In most cases, the R\&D was aiming at an order of magnitude improvement w.r.t. the state-of-the-art. The prototyping undertaken for each experiment sub-system had as a final goal to provide a realistic design with reliable performance and cost evaluation of the complete detector.

Overall, the R\&D on highly segmented sub-systems composing the experimental set-up was guided by the need to reconstruct each particle composing the final states using the so-called Particle Flow Approach (PFA). The performances demonstrated by the R\&D were transferred to the Monte-Carlo description of the experiment used to predict the achievable experimental performances for each component of the physics programme.

The R\&D was conducted by a large number of groups over many years. A wide variety of technological and design options could therefore be extensively explored. Technological alternatives were investigated for all sub-systems and were compared in terms of detection performances as well as for their cost. In most cases, these detection performances were evaluated on particle beams with system aspects directly comparable to the experiment. Some beam tests were performed inside a 2 T magnetic field in order to validate the pulsed mode concept and the associated power saving.

Each calorimeter and tracking sub-system has been prototyped extensively, from small prototypes addressing all critical elements of a given sub-system up to a complete, real scale, prototype operated on beam, sometimes in combination with prototypes of other sub-systems composing the experiment. This allowed in particular to study and mature the PFA strategy with real input.

The development of tracking and vertexing devices was governed by the need of pixellated low material budget components allowing excellent momentum resolution and displaced vertices characterization, including vertex charge, with performances exceeding typically by one order of magnitude those of existing experiments.

Two alternative approaches were investigated for the main tracker, one based on a TPC and one exploiting silicon sensors, possibly pixelated. The R\&D for the TPC addressed mainly the single point resolution and ion feedback mitigation with different micro-pattern read-out systems (MicroMegas, GEM, ...) and showed that the performance goals can indeed be reached, with a material budget of the end-caps not exceeding 30 \% X0. The approach relying on silicon detectors concentrated on the material budget, showing that the targeted momentum resolution was reachable despite the restricted number of detector layers allowed to mitigate multiple scattering effects. These efforts have also benefited a lot from the R\&D that has been conducted for the tracker upgrades for both ATLAS and CMS, which both rely on all-silicon tracking systems.
It was also shown that, while the performance was more than adequate in terms of momentum resolution, the tracking in dense jet environments could be improved by replacing the silicon-strip sensors with a large-area pixelated tracker. 

The R\& D for the vertex detector explored the potential of several thin highly granular pixel technologies (CMOS, DEPFET, FPCCD, SoI, ...) which could offer the projected spatial resolution and material budget. Intensive efforts were invested in read-out systems allowing to cope with the hit density induced by the beam related background, resulting in performances depending on the technology and the read-out architecture.The concept of double-sided layers was also investigated with some technologies and established up to the level of being operated near an e+e− interaction point.

A significant part of the R\&D effort for the ILC detector concentrated on calorimetry, which required strategies promoting very compact and high granularity detection technologies connected to very low average power read-out micro-circuits complying with power pulsing. Major issues were addressed within the CALICE collaboration, a consortium composed of more than 300 members coming from more than 50 institutes.

The R\&D for the EM calorimeter concentrated on optimized and cost effective sensor systems, on the designs of a low power, pulsed, integrated readout electronics and an effective thermal management and calibration strategy, and on finding a mechanical concept which combines large stability with minimal dead zones. A SiW based real size prototype was constructed and tested extensively on particle beams. The development of a more cost-effective technological solution, based on a scintillator and photo-multiplier matrix was also realised and some of its performances compared to those of the SiW concept.

The HCAL prototyping was governed by the need for an efficient and precise reconstruction of neutral hadron showers. Combined with stainless steel as conversion material, two read-out options were developed, one combining scintillator tiles with silicon photo-sensors read out with analog electronics, and an alternative approach based on gaseous devices (e.g. RPC) with higher segmentation but with signal encoding on one or two bits only.

The relative merits of the different ECAL and HCAL options were in particular evaluated using combined test beam campaigns providing common data which were processed with PFA software which it allowed developing and assessing at the same time. The possibility to achieve the targeted energy and topology resolutions was demonstrated even when operated in power pulsed mode.

Substantial effort was invested in developing technological solutions for the very forward calorimeters designed for robust electron and photon measurements used for integrated luminosity measurements and for bunch-to-bunch machine parameter monitoring. Satisfactory performances, including a radiation tolerance to 1 MGy, were obtained with tungsten absorber layers alternated with GaAs sensor planes read out with dedicated electronics featuring a dual gain charge amplifier providing a fast feedback for beam tuning.

Summarizing, despite constrained financial conditions, the proof of principle underlying each prominent sub-system of an experiment at the ILC has been completed. Most of the detector prototypes fabricated and tested allow extrapolating to the full detector performance and cost. Some technologies and detector concepts developed for the ILC have influenced existing experiments in their construction of upgrade phase, which in turn provide a full size validation of technological approaches first developed for the ILC. Further studies addressing system integration aspects are however still needed for some subsystems. The optimization of the complete experimental design, though quite advanced, also still requires additional studies. The few years foreseen to finalize the decision and procedure of constructing the ILC will provide the necessary time to make the most appropriate technological choices for each sub-system and to complete the full R\&D programme in due time.

\section{\label{sec:discussion}Discussion}

Marc, Ties, Phil, 2.5 pages

The ILC with a scientific program as outlined above, ranging from an initial stage at 250 GeV center-of-mass energy, and reaching to higher energies up to around 1 TeV, will make major contributions to our understanding of the universe. It will be a major international facility, which will complement and extend the science done at the LHC. It will pave the way - primarily through precision studies of the standard model - towards answering very deep questions on the nature of our universe. ILC will be central to define the route particle physics should take after the LHC, and thus not only make major scientific contributions, but also help to shape the future of the world-wide HEP program. 

The community supporting electron positron collisions as a means to study nature is very strong. Repetitively strategy processes on the national and international level have come to the conclusion, that electron positron collisions are essential for the progress in science. 

We have described above the degree of preparation and of community involvement in the development of the accelerator and of the detectors at the ILC. This community is strongly committed - as shown by a decade long investment in the R\&D and prototyping of technologies - to build the most advanced and best possible accelerator and detector. The community is already very sizeable, with some 100 institutes world - wide working in this area, in around 30 countries, and all three regions of the world. In Europe the community has united behind the series of large European funded projects like EUDET, AIDA and AIDA2020. This has resulted in a very visible leadership role Europe is playing in this area, and in the development of the detector concepts in particular. 

The community has shown a large capability to self-organise and come to decisions, including deciding on priorities and posteriorities. The most prominent example of this is the decision taken in 2006 to realise the ILC in superconducting technology. 

The community has also demonstrated large innovative power, as for example shown by the development of the concept of particle flow, and its realisation in the detector concepts - an approach which is now widely adapted by experiments at different facilities. 




\end{document}
%
% ****** End of file apssamp.tex ******

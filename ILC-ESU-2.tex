% ****** Start of file apssamp.tex ******
%
%   This file is part of the APS files in the REVTeX 4.1 distribution.
%   Version 4.1r of REVTeX, August 2010
%
%   Copyright (c) 2009, 2010 The American Physical Society.
%
%   See the REVTeX 4 README file for restrictions and more information.
%
% TeX'ing this file requires that you have AMS-LaTeX 2.0 installed
% as well as the rest of the prerequisites for REVTeX 4.1
%
% See the REVTeX 4 README file
% It also requires running BibTeX. The commands are as follows:
%
%  1)  latex apssamp.tex
%  2)  bibtex apssamp
%  3)  latex apssamp.tex
%  4)  latex apssamp.tex
%
\documentclass[%
 reprint,
%superscriptaddress,
%groupedaddress,
%unsortedaddress,
%runinaddress,
%frontmatterverbose, 
%preprint,
%showpacs,preprintnumbers,
%nofootinbib,
%nobibnotes,
%bibnotes,
 amsmath,amssymb,
 aps,
%pra,
%prb,
%rmp,
%prstab,
%prstper,
%floatfix,
]{revtex4-1}

\usepackage{graphicx}% Include figure files
\usepackage{dcolumn}% Align table columns on decimal point
\usepackage{bm}% bold math
%\usepackage{hyperref}% add hypertext capabilities
%\usepackage[mathlines]{lineno}%  Enable numbering of text and display math
%\linenumbers\relax % Commence numbering lines

%\usepackage[showframe,%Uncomment any one of the following lines to test 
%%scale=0.7, marginratio={1:1, 2:3}, ignoreall,% default settings
%%text={7in,10in},centering,
%%margin=1.5in,
%%total={6.5in,8.75in}, top=1.2in, left=0.9in, includefoot,
%%height=10in,a5paper,hmargin={3cm,0.8in},
%]{geometry}
\usepackage{acronym}

\begin{document}

\preprint{LCCPEB---}

\title{The International Linear Collider\\ An European Perspective}% Force line breaks with \\
\thanks{Version 1.3}%

\author{Juan Fuster}
% \altaffiliation[Also at ]{Physics Department, XYZ University.}%Lines break automatically or can be forced with \\
\author{etal}%
 \email{Second.Author@institution.edu}
\affiliation{%
 Authors' institution and/or address\\
% This line break forced with \textbackslash\textbackslash
}%

\collaboration{LCC Collaboration}%\noaffiliation

\date{\today}% It is always \today, today,
             %  but any date may be explicitly specified

\begin{abstract}
Input from the International Linear Collider community for the European Strategy Update

\end{abstract}

\pacs{Valid PACS appear here}% PACS, the Physics and Astronomy
                             % Classification Scheme.
%\keywords{Suggested keywords}%Use showkeys class option if keyword
                              %display desired
\maketitle

%\tableofcontents

\section{\label{sec:intro}Introduction}

Steinar, Jim, Juan, 1.5 pages

\section{\label{sec:acc}Accelerator}

Europe has a very strong scientific, technological and industrial basis
to make significant contributions to the construction of virtually any part 
of the ILC machine.

~\cite{Behnke:2013xla}

\section{\label{sec:det}Detector}
Marcel, Ties, 2 pages

Detectors at colliding beam facilities are international efforts, realised by strong collaborations. Typically many countries join forces, and contribute mostly in-kind to the building of the detector systems. 

Within the ILC project, the community has come together behind two complementary detector concepts, the ILD and the SID detector concept. After a call for ``letters of intent'' in 2009, the two detector concept groups submitted their proposals which were validated and then presented in Volume 4 of the ILC TDR ~\cite{ILC-TDR}.

Europe has participated strongly in the work on the ILC concepts and related technological developments. Initiated by the TESLA project at the start of the century, a international effort was put into place to develop the needed detector technologies. This is augmented by strong national support, but also by European programs like the EUDET initiative (2006-2010), the AIDA program (2011-2015) or the AIDA 2020 program (2016-2019). Programs in Asia, in particular in Japan, and through to a lesser extend, in the Americas were put in to place. 

The technological developments will be described in more detail in the next section. The detectors at the ILC are very different from the LHC ones in that they can focus much more strongly on precision measurements. This is possible since the environment at an electron positron collider is much more benign than at a hadron collider. Radiation hardness hardly plays a role in designing the detectors. Multiple interactions, which are a major challenge at the LHC, are essentially not present. These different boundary conditions allow a detector optimisation complementary to the one at the LHC. A focus on detailed event reconstruction, on high precision vertexing and high precision high efficiency tracking is possible. As a guiding principle Particle Flow has been widely accepted and is used by both SiD and ILD as a major indicator of the performance of the complete systems. 

European groups have played strong roles in nearly all areas of detector design and development. A core element of ILC detectors, particle flow calorimeters, has been strongly pushed by European groups. Many developments on high precision vertexing, in particular in the area of MAPS technology, have come from European groups. The development of high precision gaseous detectors based on micro-pattern gaseous detectors has been dominated by European groups. Last but not least Europe has played a central role in the integration of one of the two detector concepts, the ILD collaboration.  

\subsection{Detector preparations~\label{sec:prepphase:detectors}}
The preparation for detector construction will follow a path similar to that of the \acs{ILC} accelerator.
As was the case for the LHC detectors, a strong host lab basis is needed and will have to be provided by the Japanese ILC organisation (I think this is not decided ...).
This can and should be complemented by regional centers. In Europe CERN would be a natural candidate for such a regional center, but national laboratories like DESY or others could also play this role for parts of the detector. Examples for such more regional centers within the LHC are the Detector Assembly Facility at DESY, used for the construction of upgrades to the LHC detectors, or the CMS center at Fermilab in the United States. 

Different from the accelerator, the collaborations themselves will define and institutes the needed structures to design and build the detectors. There is less need that the host organization supports a central design team, although strong support from the host organisation on all issues of interfaces between the detectors and the accelerator is required. 

During the preparation phase, there are four major milestones for the
detectors.
\begin{description}
\item{\bfseries MDI studies}
With the choice of the final location of the interaction region of the ILC, the detector designs need to be adapted to the
conditions of the site in terms of hall size, transport capabilities and
assembly space. This adaptation process
will require close contact with the local experts in Japan,
and an European ILC design office can be of invaluable help for this
(see below).
\item{\bfseries Design optimisation}
The current designs of both SiD and ILD need to be further refined and
optimised. For this process --- which during the
preparation phase will lead to technology selections for the individual
sub-detector systems --- various considerations like
the site-specific conditions, technological developments and others will
have to be taken into account. 
\item{\bfseries Technical prototype tests}
During the optimization phase and after the proposal phase technical prototypes will
have to be produced that demonstrate the feasibility of the proposed
designs.
These technical prototypes will then be extensively tested during
test-beam campaigns. Extensive test beam will be needed. In Europe the facilities at CERN and at DESY will play a central role. Their availability will need to be ensured. 

\item{\bfseries Technical design report}
Following the usual procedures for major new experiments in  particle physics, the ILC laboratory will issue an invitation for letter of intends for experiments at the ILC. With ILD and SiD two well developed concepts exist and will most likely enter into the competition, but it is quite feasible that more groups will form and propose experiments. After a selection of a short list of experiments, technical design reports will be prepared, specifying the
baseline detector designs that will be put forward for construction at
the ILC. The goal
is the completion of the TDRs at the end of the preparation phase.
\end{description}

\subsection{\label{Section:constructionmodel:ILCDet} Estimation of a
contribution to the ILC detectors}

In high-energy physics, the financial contribution to the detector
construction and operation
is typically assumed to be proportional to the number of authors of
the detector collaboration. In the LHC experiments, Europe accounts for about 50\% of the members, in non-European experiments like Belle-II European groups contribute about a 1/3 share. Most likely an experiment at the ILC would be close to a Belle-like model, though, given the large interest in the community, possibly somewhere in between Belle and the LHC. 

Taking
the cost numbers from
the ILC TDR~\cite{ILC-TDR}, the European share for the detector
construction is approximately 270 M\Euro{}, which is comparable to the cost for the ongoing upgrade of the LHC detectors.

\section{\label{sec:RandD}R\&D strategy and achievements}

The capacity to reconstruct with extreme precision the characteristics of the final states produced at the ILC is mandatory to take advantage of the inherently accurate knowledge of the scattering conditions of the beam particles. Moreover, the relatively mild running conditions allow alleviating the constraints on radiation tolerance and read-out speed. The R\&D related to each sub-system was therefore driven by a challenging and innovative trade-off to be found between very demanding resolution (granularity) and material budget requirements on the one hand, and an acceptable speed and power consumption on the other hand. Moreover, the detector steering and read-out architectures obey specific conditions: they should be operated triggerless and designed to exploit the machine duty cycle and bunch spacing to mitigate the power consumption.
In several cases, individual performances targeted by the R\&D could have been considered as nearly achieved outside of the ILC programme, but intensive R\&D was needed to realise their combination at a level well beyond previous achievements, including detailed system integration aspects. In most cases, the R\&D was aiming at an order of magnitude improvement w.r.t. the state-of-the-art. The prototyping undertaken for each experiment sub-system had as a final goal to provide a realistic design with reliable performance and cost evaluation of the complete detector.

Overall, the R\&D on highly segmented sub-systems composing the experimental set-up was guided by the need to reconstruct each particle composing the final states using the so-called Particle Flow Approach (PFA). The performances demonstrated by the R\&D were transferred to the Monte-Carlo description of the experiment used to predict the achievable experimental performances for each component of the physics programme.

The R\&D was conducted by a large number of groups over many years. A wide variety of technological and design options could therefore be extensively explored. Technological alternatives were investigated for all sub-systems and were compared in terms of detection performances as well as for their cost. In most cases, these detection performances were evaluated on particle beams with system aspects directly comparable to the experiment. Some beam tests were performed inside a 2 T magnetic field in order to validate the pulsed mode concept and the associated power saving.

Each calorimeter and tracking sub-system has been prototyped extensively, from small prototypes addressing all critical elements of a given sub-system up to a complete, real scale, prototype operated on beam, sometimes in combination with prototypes of other sub-systems composing the experiment. This allowed in particular to study and mature the PFA strategy with real input.

The development of tracking and vertexing devices was governed by the need of pixellated low material budget components allowing excellent momentum resolution and displaced vertices characterization, including vertex charge, with performances exceeding typically by one order of magnitude those of existing experiments.

Two alternative approaches were investigated for the main tracker, one based on a TPC and one exploiting silicon sensors, possibly pixelated. The R\&D for the TPC addressed mainly the single point resolution and ion feedback mitigation with different micro-pattern read-out systems (MicroMegas, GEM, ...) and showed that the performance goals can indeed be reached, with a material budget of the end-caps not exceeding 30 \% X0. The approach relying on silicon detectors concentrated on the material budget, showing that the targeted momentum resolution was reachable despite the restricted number of detector layers allowed to mitigate multiple scattering effects. These efforts have also benefited a lot from the R\&D that has been conducted for the tracker upgrades for both ATLAS and CMS, which both rely on all-silicon tracking systems.
It was also shown that, while the performance was more than adequate in terms of momentum resolution, the tracking in dense jet environments could be improved by replacing the silicon-strip sensors with a large-area pixelated tracker. 

The R\& D for the vertex detector explored the potential of several thin highly granular pixel technologies (CMOS, DEPFET, FPCCD, SoI, ...) which could offer the projected spatial resolution and material budget. Intensive efforts were invested in read-out systems allowing to cope with the hit density induced by the beam related background, resulting in performances depending on the technology and the read-out architecture.The concept of double-sided layers was also investigated with some technologies and established up to the level of being operated near an e+e− interaction point.

A significant part of the R\&D effort for the ILC detector concentrated on calorimetry, which required strategies promoting very compact and high granularity detection technologies connected to very low average power read-out micro-circuits complying with power pulsing. Major issues were addressed within the CALICE collaboration, a consortium composed of more than 300 members coming from more than 50 institutes.

The R\&D for the EM calorimeter concentrated on optimized and cost effective sensor systems, on the designs of a low power, pulsed, integrated readout electronics and an effective thermal management and calibration strategy, and on finding a mechanical concept which combines large stability with minimal dead zones. A SiW based real size prototype was constructed and tested extensively on particle beams. The development of a more cost-effective technological solution, based on a scintillator and photo-multiplier matrix was also realised and some of its performances compared to those of the SiW concept.

The HCAL prototyping was governed by the need for an efficient and precise reconstruction of neutral hadron showers. Combined with stainless steel as conversion material, two read-out options were developed, one combining scintillator tiles with silicon photo-sensors read out with analog electronics, and an alternative approach based on gaseous devices (e.g. RPC) with higher segmentation but with signal encoding on one or two bits only.

The relative merits of the different ECAL and HCAL options were in particular evaluated using combined test beam campaigns providing common data which were processed with PFA software which it allowed developing and assessing at the same time. The possibility to achieve the targeted energy and topology resolutions was demonstrated even when operated in power pulsed mode.

Substantial effort was invested in developing technological solutions for the very forward calorimeters designed for robust electron and photon measurements used for integrated luminosity measurements and for bunch-to-bunch machine parameter monitoring. Satisfactory performances, including a radiation tolerance to 1 MGy, were obtained with tungsten absorber layers alternated with GaAs sensor planes read out with dedicated electronics featuring a dual gain charge amplifier providing a fast feedback for beam tuning.

Summarizing, despite constrained financial conditions, the proof of principle underlying each prominent sub-system of an experiment at the ILC has been completed. Most of the detector prototypes fabricated and tested allow extrapolating to the full detector performance and cost. Some technologies and detector concepts developed for the ILC have influenced existing experiments in their construction of upgrade phase, which in turn provide a full size validation of technological approaches first developed for the ILC. Further studies addressing system integration aspects are however still needed for some subsystems. The optimization of the complete experimental design, though quite advanced, also still requires additional studies. The few years foreseen to finalize the decision and procedure of constructing the ILC will provide the necessary time to make the most appropriate technological choices for each sub-system and to complete the full R\&D programme in due time.

\section{\label{sec:discussion}Discussion}

Marc, Ties, Phil, 2.5 pages

\end{document}
%
% ****** End of file apssamp.tex ******

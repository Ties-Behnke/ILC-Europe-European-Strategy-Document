% ****** Start of file apssamp.tex ******
%
%   This file is part of the APS files in the REVTeX 4.1 distribution.
%   Version 4.1r of REVTeX, August 2010
%
%   Copyright (c) 2009, 2010 The American Physical Society.
%
%   See the REVTeX 4 README file for restrictions and more information.
%
% TeX'ing this file requires that you have AMS-LaTeX 2.0 installed
% as well as the rest of the prerequisites for REVTeX 4.1
%
% See the REVTeX 4 README file
% It also requires running BibTeX. The commands are as follows:
%
%  1)  latex apssamp.tex
%  2)  bibtex apssamp
%  3)  latex apssamp.tex
%  4)  latex apssamp.tex
%
\documentclass[%
 reprint,
%superscriptaddress,
%groupedaddress,
%unsortedaddress,
%runinaddress,
%frontmatterverbose, 
%preprint,
%showpacs,preprintnumbers,
%nofootinbib,
%nobibnotes,
%bibnotes,
 amsmath,amssymb,
 aps,
%pra,
%prb,
%rmp,
%prstab,
%prstper,
%floatfix,
]{revtex4-1}

\usepackage{graphicx}% Include figure files
\usepackage{dcolumn}% Align table columns on decimal point
\usepackage{bm}% bold math
%\usepackage{hyperref}% add hypertext capabilities
%\usepackage[mathlines]{lineno}% Enable numbering of text and display math
%\linenumbers\relax % Commence numbering lines

%\usepackage[showframe,%Uncomment any one of the following lines to test 
%%scale=0.7, marginratio={1:1, 2:3}, ignoreall,% default settings
%%text={7in,10in},centering,
%%margin=1.5in,
%%total={6.5in,8.75in}, top=1.2in, left=0.9in, includefoot,
%%height=10in,a5paper,hmargin={3cm,0.8in},
%]{geometry}

\begin{document}

\preprint{LCCPEB---}

\title{The International Linear Collider\\ A European Perspective}% Force line breaks with \\
\thanks{Version 1.3}%

\author{Juan Fuster}
% \altaffiliation[Also at ]{Physics Department, XYZ University.}%Lines break automatically or can be forced with \\
\author{etal}%
 \email{Second.Author@institution.edu}
\affiliation{%
 Authors' institution and/or address\\
% This line break forced with \textbackslash\textbackslash
}%

\collaboration{LCC Collaboration}%\noaffiliation

\date{\today}% It is always \today, today,
             %  but any date may be explicitly specified

\begin{abstract}
Input from the International Linear Collider community for the European Strategy Update 

\end{abstract}

\pacs{Valid PACS appear here}% PACS, the Physics and Astronomy
                             % Classification Scheme.
%\keywords{Suggested keywords}%Use showkeys class option if keyword
                              %display desired
\maketitle

%\tableofcontents

\section{\label{sec:intro}Introduction}

Steinar, Jim, Juan, 1.5 pages

\section{\label{sec:acc}Accelerator}

Phil, Marcel, Steinar, 2 pages

\section{\label{sec:det}Detector}
Marcel, Ties, 2 pages

The science at the ILC drives the requirements on detectors. The main factors are:
\begin{itemize}
 \item The detector has to have an excellent track momentum
   resolution. The benchmark reaction here is the analysis 
of the di-lepton mass in the process $HZ \to H \ell^+
\ell^-$. This reaction allows the reconstruction of the 
Higgs mass independent of its decay mode via the 
reconstruction of the lepton recoil spectrum. In order that 
the momentum resolution of the detector does not limit 
the mass resolution achievable for the recoiling lepton 
system, stringent momentum resolution requirements have to be met. 
\item The reconstruction of the flavour of the final state can 
often be done best with the help of lifetime information of the 
decaying particles. For this, very powerful vertex detectors 
are needed. This is particularly important 
in the Higgs sector, where -- at least for light Higgs bosons -- 
a large fraction of the Higgs decays has bottom 
quarks in the final state. Many other physics signatures will 
produce complex final states with bottom or charm quarks as well. 
A supreme vertex detector therefore is needed to reconstruct these 
long lived particles with excellent resolution. 
\item The overall event is best reconstructed with the 
particle flow measurement. The particle flow technique combines 
the information from the tracking systems and from the 
calorimetric systems in an attempt to reconstruct the 
energy and the direction of all charged and 
neutral particles in the event. To minimize overlaps between 
neighboring particles, and to maximize the probability to 
correctly combine tracking and calorimeter information, 
excellent calorimeters are needed with very high granularity. 
\item Many physics signatures predict some undetectable particles, 
which escape from the detector. They can only be reconstructed by 
measuring the missing energy in the event. This requires 
that the detector is as hermetic as possible, to 
minimize the amount of energy that can escape detection. 
Particular care has to be given to the region surrounding the 
beampipe in the forward direction. 
\end{itemize}

Compared to the last large scale detector project in particle physics, the construction and upgrade of the LHC detectors, the emphasis for linear collider detector is shifted towards ultimate precision. This requires detcetor technologues which are driven towards ultimate precision, and this requires a minimisation of dead material in the detector, at an unprecedented level. This also requires a management and control of services and in particular a thermal management of the detector concept. Significant technological R\& D was needed to demonstrate the feasibility, and is, in fact, still ongoing, as will be discussed in the next section.  

Over the last decade two detector concepts have emerged from the discussions in the community. Both are based on the assumption that particle flow reconstruction plays a central role in the event reconstruction. Both therefore have highly granular calorimeters, placed inside the coil which is providing the central magnetic field. Both have excellent trackers and vertexing systems. The two approaches differ in the choice of tracker technology, and in the approach taken to maximize the overall precision of the event reconstruction. ILD has chosen a gaseous central tracker, a time projection chamber, combined with silicon detectors inside and outside the TPC. SiD relies on an all Silicon solution, similar to the LHC detectors. ILD tries to optimize the particle flow resolution by making the detector large, thus separating charged and neutral particles. SiD keeps the detector more compact, and compensates the reduced particle separation at the position of the calorimeters by using a higher central magnetic field. Both approaches have demonstrated excellent performance, meeting or even exceeding the performance requirements. 

For both detector concepts, communities have found themselves and pre-collaborations have formed. These organizations have over the last ten years or so pushed both concepts to a remarkable level of maturity, and have, in close interaction with the different groups performing detector R\&D from around the world, demonstrated the feasibility to build and operate such high precision detectors. 

European groups have played a central role in these efforts. The ILD concept group is formed from some 70 groups from around the world, with more than half coming from Europe. The SiD collaboration has a strong basis in the Americas, but also relies on significant participation from European groups. Major contributions to the development of all sub-systems have come from Europe. Significant technological breakthroughs for example in the area of highly granular calorimeter are strongly driven by European groups. 

An important aspect of the detector concept work has been in addition to the development and demonstration of the technology the integration of the detector into the collider and into the proposed site. The location of the experiment in an earth-quake prone area poses challenges which have been addressed through R\& D on detector stability, support and service. The scheme to operate two detectors in one interaction region, the so called push-pull scheme, has no example and needed significant engineering work to demonstrate its feasibility. With strong support from particle physics laboratories in Europe, in  particular DESY and CERN, many of the most relevant questions could be answered and the feasibility of the approach could be demonstrated at least in principle. 



\section{\label{sec:RandD}R\&D strategy and achievements}

The capacity to reconstruct with extreme precision the characteristics of the final states produced at the ILC is mandatory to take advantage of the inherently accurate knowledge of the scattering conditions of the beam particles. Moreover, the relatively mild running conditions allow alleviating the constraints on radiation tolerance and read-out speed. The R&D related to each sub-system was therefore driven by a challenging and innovative trade-off to be found between very demanding resolution (granularity) and material budget requirements on the one hand, and an acceptable speed and power consumption on the other hand. Moreover, the detector steering and read-out architectures obey specific conditions: they should be operated triggerless and designed to exploit the machine duty cycle and bunch spacing to mitigate the power consumption.
In several cases, individual performances targetted by the R&D could have been considered as nearly achieved outside of the ILC programme, but intensive R&D was needed to realise their combination at a level well beyond previous achievements, including detailed system integration aspects. In most cases, the R&D was aiming at an ordre of magnitude improvement w.r.t. the state-of-the-art. The prototyping undertaken for each experiment sub-system had as a final goal to provide a realistic design with reliable performance and cost evaluation of the complete detector.
Overall, the R&D on highly segmented sub-systems composing the experimental set-up was guided by the need to reconstruct each particle composing the final states using the so-called Particle Flow Approach (PFA). The performances demonstrated by the R&D were transfered to the Monte-Carlo description of the experiment used to predict the achievable experimental performances for each component of the physics programme.
The R&D was conducted by a large numbre of groups over many years. A wide variety of technological and design options could therefore be extensively explored. Technological alternatives were investigated for all sub-systems and were compared in terms of detection performances as well as for their cost. In most cases, these detection performances were evaluated on particle beams with system aspects directly transposable to the experiment. Some beam tests were performed inside a 2 T magnetic field in ordre to validate the pulsed mode concept and the associated power saving.
Each calorimetre and tracking sub-system has been prototyped extensively, from small prototypes addressing all critical elements of a given sub-system up to a complete, real scale, prototype operated on beam, sometimes in combination with prototypes of other sub-systems composing the experiment. This allowed in particular to study and mature the PFA strategy with real input.
The development of tracking and vertexing devices was governed by the need of pixellated low material budget components allowing excellent momentum resolution and displaced vertices characterisation, including vertex charge, with performances exceeding typically by one ordre of magnitude those of existing experiments.
Two alternative approaches were investigated for the main tracker, one based on a TPC and one exploiting silicon sensors, possibly pixelated. The R&D for the TPC addressed mainly the single point resolution and ion feedback mitigation with different micro-pattern read-out systems (MicroMegas, GEM, ...) and showed that the performance goals can indeed be reached, with a material budget of the end-caps not exceeding 30 % X0. The approach relying on silicon detectors concentrated on the material budget, showing that the targetted momentum resolution was within reach despite the restricted numbre of detector layers allowed to mitigate multiple scattering effects. The necessary momentum resolution was shown to be reachable but the restricted numbre of layers had shortcomings in terms of tracking performances, calling for a large area pixelated Si tracker.
The R&D for the vertex detector explored the potential of several thin highly granular pixel technologies (CMOS, DEPFET, FPCCD, SoI, ...) which could offer the ambitionned spatial resolution and material budget. Intensive effort was invested in read-out systems allowing to cope with the hit density induced by the beam related background, resulting in performances depending on the technology and the read-out architecture.The concept of double-sided layers was also investigated with some technologies and established up to the level of being operated near an e+e− interaction point.
Most of the R&D effort for the ILC detector concentrated on calorimetry, which required strategies promoting very compact and high granularity detection technologies connected to very low average power read-out micro-circuits complying with power pulsing. Major issues were addressed in CALICE, a consortium composed of more than 300 membres coming from more than 50 institutes.
The R&D for the EM calorimeter concentrated on optimised and cost effective sensor systems, on the designs of a low power, pulsed, integrated readout electronics and an effective thermal management and calibration strategy, and on finding a mechanical concept which combines large stability with minimal dead zones. A SiW based real size prototype was constructed and tested extensively on particle beams. The development of a more cost-effective technological solution, based on a scintillator and photo-multiplier matrix was also realised and some of its performances compared to those of the SiW concept.
The HCAL prototyping was governed by the need for an efficient and precise reconstruction of neutral hadron showers. Combined with stainless steel as conversion material, two read-out options were developed, one combining scintillator tiles with silicon photo-sensors read out with analog electronics, and an alternative approach based on gaseous devices (e.g. RPC) with higher segmentation but with signal encoding on one or two bits only.
The relative merits of the different ECAL and HCAL options were in particular evaluated using combined test beam campaigns providing common data which were processed with PFA software which it allowed developing and assessing at the same time. The possibility to achieve the targetted energy and topology resolutions was demonstrated even when operated in power pulsed mode.
Substantial effort was invested in developping technological solutions for the very forward calorimetres de-signed for robust electron and photon measurements used for integrated luminosity measurements and for bunch-to-bunch machine parameter monitoring. Satisfactory performances, including a radiation tolerance to 1 MGy, were obtained with tungsten absorber layers alternated with GaAs sensor planes read out with dedicated elec-tronics featuring a dual gain charge amplifier providing a fast feedback for beam tuning.
Summarising, despite constrained financial conditions, the proof of principle underlying each prominent sub-system of an experiment at the ILC has been completed. Most of the detector prototypes fabricated and tested allow exprapolating to the full detector performance and cost. Some technologies and detector concepts developed for the ILC have irrigated existing experiments in their construction of upgrade phase, which in turn provide a full size validation of technological approaches first developed for the ILC. Further studies addressing system integration aspects are however still needed for some subsystems. The optimisation of the complete experimental design, though quite advanced, also still requires additional studies. The few years foreseen to finalise the decision and procedure of constructing the ILC will provide the necessary time to make the most appropriate technological choices for each sub-system and to complete the full R&D programme in due time.

\section{\label{sec:discussion}Discussion}

Marc, Ties, Phil, 2.5 pages

\end{document}
%
% ****** End of file apssamp.tex ******

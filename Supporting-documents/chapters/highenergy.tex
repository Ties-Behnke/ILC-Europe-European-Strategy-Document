
% section on physics above 250 GeV


A key advantage of linear colliders is the possibility to upgrade
the center-of-mass energy.  The energy reach of circular electron-positron
colliders of a given circumference is limited by synchrotron
radiation, and this is difficult to overcome because of the steep
growth of synchrotron losses with energy.  Linear colliders, however,
can be upgraded to reach higher center of mass energies either by
increasing the length of the main linacs or by installing linac components
that support higher accelerating gradients.  

 After finalizing the programme discussed in
Sections~\ref{sec:physics} -- \ref{sec:searches}, the ILC can
be expanded to explore energies well beyond 250~GeV. Our 
discussion of the machine design in Sec.~\ref{sec:ilc} and the 
running scenarios in Sec.~\ref{sec:runscenarios} already incorporates a 
planned energy upgrade to 500~GeV.  Plans for a further energy upgrade
to 1\,TeV were already discussed in the ILC TDR.
  We thus see the ILC as having a long
future with operation at number of stages beyond the initial one, at
increasingly high energies. 

In this section, we will present a further discussion of the
possibilities for energy upgrade of the ILC.  Then, we will describe
the physics program that these energy upgrades will make available.

The physics goals for higher-energy $\ee$ colliders has already been 
discussed extensively in the literature. Particularly useful
references are the volumes presenting  detailed studies carried out for 
the ILC~\cite{Fujii:2015jha,Baer:2013cma}
and CLIC~\cite{Linssen:2012hp,deBlas:2018mhx,Roloff:2018dqu} design reports. 



%  In this section,
% we review the prospects and physics goals for ILC operation at higher
% energies. 

% {\bf Intro and scope:} While the energy reach of circular electron-positron
% colliders of a given circumference is limited by synchrotron radiation,
% linear colliders can be upgraded to reach higher center-of-mass energy.
% It is therefore natural to envisage a program beyond the baseline program
% at $\sqrt{s}= 250$~GeV discussed in Sections~\ref{sec:physics}, \ref{sec:higgs}
% and \ref{sec:searches}. The energy extendability of linear colliders
% provides great flexibility to respond to new discoveries, at the ILC or elsewhere,
% by adapting the collider design. The exact outline of the high-energy
% program will likely change with the developing insight in the physics of the Standard Model,
% and what lies beyond it. The high-energy program is moreover affected by new developments
% in accelerator technology. This section presents a brief status for the most relevant
% accelerator R\&D lines and a brief review of studies of the physics potential at
% center-of-mass energies greater than 250~GeV.  

% There exists an extensive literature on the subject. We refer in particular
% many detailed studies performed for the ILC~\cite{Fujii:2015jha,Baer:2013cma}
% and CLIC design reports~\cite{Linssen:2012hp,Roloff:2018dqu}.


\subsection{Technical aspects of ILC energy upgrades}
\label{subsec:highE:tech}

In the major particle physics laboratories, the lifetime of collider elements
and infrastructure has rarely been  limited to the scope of the project they were
designed and built for. A famous example is the Proton Synchrotron at CERN.
Initially commissioned in 1959, it is still in operation as
part of the accelerator complex that prepares protons for injection in the
Large Hadron Collider. In this accelerator complex, the  expensive civil engineering
efforts to construct each component are reused, so that their cost is effectively shared. The tunnel
that was constructed for LEP now hosts the Large Hadron Collider and its
luminosity upgrade. In very much the same way, we
expect  that the ILC will  form the seed for a facility that contributes to the
cutting edge of particle physics for decades.


For electron-positron collisions, any facility at energies much higher than
those already realized must be a linear collider in a long, straight
tunnel.
As we have already explained, 
the ILC infrastructure will provide a basis for collisions at 500~GeV
The most obvious energy upgrade path is an extension of the linear
accelerator sections of the colliders, which provides an increase
in center-of-mass energy that is proportional to the length of the linacs.
The design of the ILC presented in the Technical Design
Report~\cite{Behnke:2013xla,Adolphsen:2013jya,Adolphsen:2013kya} envisaged a
center-of-mass energy of 500~GeV{} in a facility with a total length of 31~km.
The ILC TDR also documents
a possible extension to 1~TeV based on current superconducting RF
technology.  We have reviewed the details of this extension 
 in 
Sec.~\ref{subsubsec:upg-optE}.

An even larger increase in center-of-mass energy may be achieved by
exploiting advances in accelerator technology. The development of
cavities with higher accelerating gradient can drive a significant
increase in the energy while maintaining a compact infrastructure.
Superconducting RF technology is evolving rapidly. Important
progress has been made in developing cavities with a gradient well
beyond the 35 MV/m required for 
the ILC~\cite{Grassellino:2018tqg,Grassellino:2017bod}.
and even beyond the 45 MV/m envisaged for
 the 1~Tev{} ILC.     In the longer term, alternate-shape
or thin-film-coated $Nb_3Sn$ cavities
 or multi-layer coated cavities offer the potential of
significantly increased cavity performance~\cite{Adolphsen:2013jya}.
Novel acceleration schemes may achieve even higher gradients. 
The CLIC drive beam concept has
achieved accelerating gradients of up to 100 MV/m~\cite{Aicheler:2012bya}.
Finally, the advent of acceleration schemes based on plasma
wakefield 
acceleration or another advanced concept could 
open up the energy regime up to 30~TeV. A report of the status of accelerator R\&D and remaining
challenges is found in Refs.~\cite{advancedLC2020,advancedLC}, with further details and
a brief description of the potential of such a machine in the
addendum~\cite{advancedLCaddendum}.

The most obvious energy upgrade path is an extension of the linear
accelerator (LINAC) sections of the colliders, which provides an increase
in center-of-mass energy that is proportional to the length of the LINACs.
The design of the ILC presented in the Technical Design
Report~\cite{Behnke:2013xla,Adolphsen:2013jya,Adolphsen:2013kya} envisaged a
center-of-mass energy of 500~GeV{} in a facility with a total length of 31~km.
The reference staging scenario (H20-staged~\cite{Barklow:2015tja}) consists of operation at
three energy stages: 250, 350, and 500~GeV. The ILC TDR also documents
a possible extension to 1~TeV based on current superconducting RF technology. 



An even larger increase in center-of-mass energy may be achieved by
exploiting advances in accelerator technology. The development of
cavities with higher accelerating gradient can drive a significant
increase in the energy while maintaining a compact infrastructure.
Superconducting RF technology is rapidly evolving and important
progress has been made in developing cavities with a gradient well
beyond the 35 MV/m required for the ILC~\cite{Grassellino:2018tqg,Grassellino:2017bod}.
and even beyond the 45 MV/m envisaged for the 1~Tev{} ILC. In the longer term, alternate-shape
or thin-film-coated $Nb_3Sn$ cavities or multi-layer coated cavities offer the potential of
significantly increased cavity performance~\cite{Adolphsen:2013jya}.
Novel acceleration schemes may achieve even higher gradient. The CLIC drive beam concept has
achieved accelerating gradients above  100~MV/m~\cite{Robson:2018enq}.
Finally, the advent of novel acceleration schemes based on plasma wakefield acceleration
may open up the energy regime up to 30~Tev. Recent reports on
 the status of advanced  accelerator R\&D and remaining
challenges can be found in Refs.~\cite{Cros:2017jxp,Cros:2019tns}.

These prospects give us much reason for optimism. In our opinion, the
 construction
of the ILC will allow Japan to host a laboratory
in Asia that will be the global host for experiments with electron
and positron beams that will define the energy frontier into the longer-term future.



\subsection{Physics goals of ILC energy upgrades}
\label{subsec:highE:physics}

Higher energy can enhance the scientific return of the installation in several
ways. It naturally increases the direct discovery reach of the collider. Experience at
previous $e^+e^-$ colliders suggests that for most scenarios it is straightforward to
discover or exclude pair-production of new particles with masses almost up to the
kinematic limit of half the center-of-mass energy~\cite{Fujii:2017ekh}. Additional
Higgs bosons of an extended Higgs sector can be discovered as long as the sum of
the masses is less than the center-of-mass energy. The potential
of the 500~GeV (and 1~TeV) ILC has been demonstrated in detailed simulations
for many important benchmark scenarios~\cite{Fujii:2015jha,Baer:2013cma}, including
scenarios with WIMP dark matter~\cite{Bartels:2012ex} and SUSY scenarios with
a light electroweak sparticle spectrum\cite{Berggren:2013vfa}.

An $e^+e^-$ collider with a center-of-mass energy beyond 250~GeV can also
probe several new Standard Model processes.
Top quark physics starts at the top quark pair production threshold at
$\sqrt{s}\sim 2 m_t$. A Linear Collider with a center-of-mass energy
above this threshold can provide a precise scrutiny of top quark
properties and its (electroweak) interactions.
Two further thresholds are found around 500~GeV, where Higgs boson pair production
and associated production of a Higgs boson with top quarks open up.
The analysis of these processes allow for a much more model-independent
extraction of the Higgs trilinear self-coupling (in $Zhh$ and $\nu \bar{\nu}hh$
production~\cite{Barklow:2017awn}) and the top quark Yukawa
coupling (in $t\bar{t}h$ production~\cite{Yonamine:2011jg}). 
Furthermore, many $t$-channel processes become accessible or even dominant
at higher energy. The cross sections for vector boson fusion production
of the Higgs boson and for vector boson scattering as a component of
di-boson production continue to increase as the center of mass energy
is raised.

Finally, the higher energy can be an important tool in tests of the SM
using Effective Field Theory.   We have emphasized that an analysis
within  EFT allows a more incisive search for new physics effects on
Higgs boson couplings by bringing new observables to bear on  the
analysis.  In the EFT formulae, different operator contributions have different
energy-dependence, with certain operators having an impact that grows 
strongly with energy.  Thus there is great advantage in combining a
data set taken at 250~GeV with one or more data sets taken at higher
energies.

We will expand on all of these points in the following subsections.




\subsection{Higgs physics}
\label{subsec:highE:Higgs}

The importance of the ILC for Higgs physics has been discussed in
Section~\ref{sec:higgs}.
Higher-energy operation extends the Higgs physics programme in several
ways.
The rate for vector-boson fusion production of the Higgs boson
($\ee\to \nu\bar\nu H$)  increases with
center-of-mass energy.  This process makes only a minor contribution 
to the study of the Higgs boson at 250~GeV, but it higher energies, it
plays an equal role to $\ee \to ZH$.  Even more importantly, data taken
at 500~GeV or higher energy opens two new channels that provide direct
access to key couplings that must be studied to gain the complete
picture of the Higgs boson interactions.  These are $\ee \to ZHH$ and
$\ee\to \nu\bar \nu HH$, which give access to the the Higgs 
self-coupling~\cite{Barklow:2017awn}
and $\ee\to t\bar t H$, which gives access to the top quark
 Yukawa coupling~\cite{Yonamine:2011jg}.

\subsubsection{Vector-boson fusion production of the Higgs boson}
\label{subsubsec:highE:VBFHiggs}

The $WW$ fusion process of Higgs boson production provides
complementary information to that obtained from the Higgstrahlung
process.  The cross section for Higgs production through this process
is larger than that for Higgsstrahlung at center of mass energies
above
450~GeV.   At high energies, then, the $WW$ process gives a new, independent data set
for the measurement of Higgs properties is also complementary in the way that it
depends on the Higgs boson couplings.  We have already discussed the
experimental
measurement of this process in Sec.~\ref{subsubsec:higgs:nunuee}, and
we have incorporated the projected results from this process into our
summary of the ILC Higgs capabilities presented in Sec.~\ref{subsec:higgs_lhcilc}.
Higher energy also improves the precision of the determination of the
triple gauge boson couplings that provide input to the EFT extraction
of Higgs couplings. 

Global fits for the ILC H20 scenario find that addition of the data
set at $\sqrt{s}=$ 500~GeV improves the precision on most of
couplings by approximately a factor two~\cite{Barklow:2017suo}.
CLIC studies arrive at the same conclusion after an analysis of
the potential of the runs at 1.5~TeV  and
3~Tev~\cite{Abramowicz:2016zbo,Robson:2018zje}.

\subsubsection{Higgs-boson pair production and measurement 
of the trilinear self-coupling}
\label{subsubsec:highE:tripleHiggs}

The measurement of the trilinear Higgs coupling is an important goal
of a complete program of study for the Higgs boson.  While the
measurement of the Higgs field vacuum expectation value and the mass
of the Higgs boson express the mass scale of the Higgs field potential
energy and its variation, the trilinear Higgs coupling gives
information on the shape of the potential energy function and brings
us closer to understanding its origin.

The trilinear Higgs coupling is sensitive to the nature of the phase
transition in the early universe that led to the present state of
broken electroweak symmetry.  
The  SM predicts a continuous phase transition.   This has
implications for models of the creation of the matter-antimatter
asymmetry that we observe in the unverse today. According to
Sakharov's classic analysis~\cite{Sakharov:1967dj}, the net baryon number of the universe
needed to be created in an epoch with substantial deviations from
thermal equilibrium in which $CP$- and baryon-number-violating
interactions were active.    The baryon number of the universe could
have been created at the electroweak phase transition, making use of new
$CP$-violating interactions in the Higgs sector, but only if the phase
transition was strongly first-order.   In explicit models, this
typically requires large deviations of this coupling, by a factor
1.5--3, from its SM value~\cite{Morrissey:2012db}.

The  trilinear Higgs coupling can be measured at colliders in two
different ways.   First, the coupling can be measured directly, using 
processes with Higgs pair production
that  diagrams that involve the triple Higgs coupling.  Second, the
coupling can be measured indirectly, since
radiative corrections to single-Higgs processes can include effects
due to the trilinear Higgs coupling.

The important Higgs pair production reactions at $\ee$ colliders are 
$\ee\to ZHH$ and $\ee\to \nu\bar\nu HH$.    The first of these
processes can be studied already at 500~GeV; the second, which is a
4-body process, requires
somewhat higher energy
Detailed simulation studies at a center-of-mass energy of 500~GeV show that a discovery of the
double Higgs-strahlung process is possible within the H20 program.
With 4~\iab\ at 500~GeV, a
a combination of several decay channels
would yield a precision of 16\% on the total cross section for
$\ee\to ZHH$~\cite{Duerig:2016dvi}.   Assuming the SM with only the
trilinear Higgs coupling free, this corresponds to an uncertainty of
27\% on that coupling.

At still higher energy vector boson fusion becomes the dominant
production channel. Making use of this channel, with  8~\iab\ at
1~TeV, the studies \cite{TianHHH,Roloff:2019crr} show that, in the
same context of varying the trilinear Higgs coupling only, this
coupling can be determined to better than 10\%. 

%%%[25]  J. Tian, LC-REP-2013-003, http://www-flc.desy.de/lcnotes/notes/LC-REP-2013-003.pdf
%%%[26]  M.   Kurata et  al,   LC-REP-2013-025,
%%%http://www-flc.desy.de/lcnotes/notes/LC-REP-2013-025.pdf


The indirect determination of the trilinear Higgs coupling is based on
the observation of McCullough~\cite{McCullough:2013rea} that the cross
section for $\ee\to ZH$ contains a radiative correction involving the
trilinear coupling that lower the cross section by about 1.5\% from
250~GeV to 500~GeV, with most of the decrease taking place below
350~GeV.  Taken a face value in the simple context with only the
trilinear coupling free, the ILC cross section measurements would determine the trilinear 
coupling to about 40\% [needs checking].

It is important to note, however, that the determination of the
trilinear coupling involves two separate questions.  First, is the SM
violated?   The accuracies with which this question can be answered
are those given above.  Second, can the violation of the SM be
attributed to a change in the trilinear coupling or the Higgs
potential rather than being due to other possible new physics
effects.  A precise way to ask this question is: Can the shift of the
trilinear coupling be measured  independently of possible effects of
all 
other dimension-6 EFT operators?   To our knowledge, this latter
question has only been addressed for determinations of the trilinear
coupling at lepton colliders.   In Ref.~\cite{Barklow:2017awn} it is
shown that, after the ILC H20 program of single-Higgs measurements is
complete, the uncertainty in the measurement of the total cross
section for  $\ee\to ZHH$ receives a negligible 2.5\% uncertainty due
to variation of the other relevant dimension-6 EFT perturbations.   In
Ref.~\cite{DiVita:2017vrr}, it is shown that, when the cross section
for $\ee\to ZH$ is fit together with other relevant observables at
250~GeV and 500~GeV
in a fit includes all other relevant EFT perturbations, the
uncertainty in the coupling is not substantially changed from the value of
40\%  [needs checking] quoted above.   These two
independent determinations of the trilinear couplings can be combined
into a determination of the trilinear coupling  with an accuracy of 23\% that is
also completely model-independent.  The cross section for $\ee\to ZHH$
increases as the  trilinear coupling increases.   So, if the trilinear coupling were
indeed 2 times its SM value, as expected in models of electroweak
baryogenesis, this effect would be observed at 6~$\sigma$ [needs
checking] and the value of ther trilinear determined to 15\% [needs
checking]. 


\subsubsection{Measurement of the top quark Yukawa coupling coupling}
\label{subsubsec:highE:topYukawa}

The top quark is the heaviest particle of the S. The reason that
this particle is especially heavy compared to other quarks and leptons
is not understood.  Thus, it is quite plausible that the top quark
coupling to the Higgs boson should show anomalies from the SM that
might not be visible for other fermions.  For this reason, it is
important to measure this parameter accurately. 

As for the trilinear Higgs coupling, the top quark Yukawa coupling can be
measured either directly or indirectly.  In the literature, most
estimates of the accuracy of determination of the top quark Yukawa
coupling are done within the simple context of the SM with only this
one parameter varied.   We will quote uncertainties
within
this model in this section and discuss the implications of a general 
EFT analysis in Sec.~\ref{subsec:highE:top}.

Consider first the indirect determination of the top quark Yukawa coupling.
   For the Higgs boson decays
$H\to gg$, $H\to \gamma\gamma$, and $H\to Z \gamma$, there are no SM tree
diagrams
and so diagrams with top quark loops give leading contributions.  For
$h\to gg$, the top quark loop diagram gives the single largest
contribution.   In Tab.~\ref{tab:ILCLHC}, it is shown that the ILC
program up to 
500~GeV will determine the effective coupling in this process to bettter
than 1\%.    Even high precision can be obtained in a joint fit
including also the top quark radiative corrections to the cross
sections for $e^+e^- \rightarrow ZH$, $e^+e^- \rightarrow \nu\bar\nu
H$, and $e^+e^- \to  \gamma H$~\cite{Boselli:2018zxr}.   However, one
should be uncomfortable that, for this determination, the simple model is
too simple, since new heavy colored particles can also contribute to
these processes at the 1-loop level.

An indirect deterimination that calls out the top quark more
specifically is the measurement of the influence of the top quark
Yukawa coupling on the shape of the  $t\bar{t}$ pair production 
cross section very close to the $t\bar{t}$ threshold, due to the Higgs
boson-exchange contribution to the $t\bar t$ potential. In principle,
this effect could give a 4\% determination of the Yukawa coupling if
the QCD  theory of the top quark threshold region were precisely
known.   However, the Higgs-exchange effect is of the same size as
the N$^3$LO QCD corrections.  At this time, the threshold shape is
calculated only to this N$^3$LO order, by the use of a very sophisticated
NRQCD framework~\cite{Beneke:2015kwa}, combined with NNLL resummation of
large logarithms~\cite{Hoang:2013uda}.   Propagating the QCD
uncertainties gives an uncertainty of 20\% on the top quark Yukawa
coupling~\cite{Vos:2016til} , and there is no clear path at this time to improve the
accuracy of the QCD result.

 FInally, the top quark Yukawa coupling can be determined directly by
 measuring the cross section for the process
$e^+e^- \rightarrow t\bar{t}h$.  In the SM, this cross section is simply
proportional to the square of the Yukawa coupling.   The cross 
section for $t\bar{t}h$ production increases rapidly above $\sqrt{s} \sim 500 $~GeV,
reaching several fb for $\sqrt{s} = 550$~GeV. Detailed studies of
selection
 and reconstruction of these complex multi-jet events
have been performed by the ILC at 500~GeV~\cite{Yonamine:2011jg} and
 1~TeV~\cite{Behnke:2013lya,Price:2014oca} and by CLIC
at 1.5~TeV~\cite{Abramowicz:2018rjq}. The direct measurement of the
top
 quark Yukawa coupling at the ILC reaches 3\%
precision~\cite{Fujii:2015jha}, with 4~\iab{} at 550~GeV.  With a sample of
2.5~\iab\  at 
1~TeV, this precision would improve to 2\%~\cite{Asner:2013psa}.  From
the energy-dependence of the cross section and the top polarizations,
this reaction can also be used to probe for non-standard forms of the
$tth$ coupling~\cite{Han:1999xd}.




\subsection{Top quark physics}
\label{subsec:highE:top}

The extension of the ILC to 350 and 500~GeV will allow the precision
study of the top quark.   This is an essential goal of precision
experiments on the SM, for two reasons.  First, similarly to the Higgs
boson, the top quark stands closer to the essential mysteries of the
SM than any other quark or lepton.   It is heavier than the next
lighter fermion, the $b$ quark, by a factor of 40 and heavier than the
lightest quark, the $u$ quark, by a factor of $10^5$.  The reasons for
this are unknown, but they must be related to other mysteries of the
Higgs sector and SM mass generation.  In fact, it is not understood
whether the top quark is ``heavy'' quark because of special
interactions that the other quarks do not share or, alternatively, whether the top
quark is an ``ordinary'' quark receiving an order-1 mass while the
masses of the other quarks are highly suppressed.  Competing extensions of the
SM such as supersymmetry and composite Higgs models differ in their
answers to this question.   

Second, the fact that the top quark has spin, couples to the
parity-violating weak interactions, and decays to nonzero spin
particles through  $t\to bW$ gives a large number of independent
observables for each $t\bar t$ production process. An $\ee$ collider
with beam polarization can take advantage of all of these observables,
especially if it can produce $t\bar t$ well above threshold at
500~GeV in the center of mass.  

Thus, top quark physics is a place in which we expect to find
deviations from the predictions of the SM, in a setting where we have
many handles to search for these new physics effects.   In the
remainder of this section, we will review highlights of the top quark
program of linear colliders.
The potential of linear $e^+e^-$ colliders for top quark physics is
discussed in more detail in the 
ILC design reports~\cite{Baer:2013cma,Behnke:2013lya}
and in Refs.~\cite{Agashe:2013hma,Vos:2016til,Abramowicz:2018rjq}.




\subsubsection{Measurement of the top-quark mass}
\label{subsubsec:highE:topmass}

The top quark mass is a fundamental parameter of the Standard Model, that has to be
determined experimentally. Precise measurements are essential for precise tests of
the internal consistency of the Standard Model, through the electro-weak
fit~\cite{Baak:2014ora} or the extrapolation of the Higgs potential to
very high energy
scales~\cite{Degrassi:2012ry}.   The precise value of the top quark is
also needed as input to the theory of flavor-changing weak
decays~\cite{Buras:2009if} and models of the grand unification of the
fundamental interactions~\cite{Langacker:1994vf}.

The top quark pair production threshold was identified long ago~\cite{Gusken:1985nf} as
an ideal laboratory to measure the top quark mass, and other properties such as the top quark
width and the Yukawa coupling and the strong coupling constant~\cite{Strassler:1990nw}.
The large natural width of the top quark acts as an infrared cut-off,
rendering the threshold cross section insensitive to the non-perturbative confining part
of the QCD potential and allowing a well-defined  cross section
calculation within  perturbative QCD.  This calculation has now been
carried out the N$^3$LO order~\cite{Beneke:2015kwa} with  NNLL resummation~\cite{Hoang:2013uda}. Fully
differential results are available in WHIZARD~\cite{Bach:2017ggt}.

Given this precise theoretical understanding of the shape of the
$t\bar t$ threshold cross section as a function of center of mass
energy, it is possible to extract the value of the top quark mass by
scanning the values of  this cross section near threshold.  We
emphasize that the top quark mass determined in this way is, directly,
a short-distance quantity that is not subject to significant
nonperturbative corrections.
It is also closely related to the $\msb$ top quark mass, the input to
the theory calculations listed above. The uncertainty in the
conversion is less than 10~MeV~\cite{Marquard:2015qpa}.  This
contrasts with the situation at hadron colliders, where the
conversion uncertainties, the nonperturbative corrections, and the
experimental systematics in the measured top quark mass
contribute  independent uncertainties, each of which is about
200~MeV. 

The scan of the $t\bar t$ threshold is expected to be done with
measurements at ten $e^+e^-$
center-of-mass energy points. A fit of the line shape will give
a precise extraction of the top quark mass~\cite{Martinez:2002st,Horiguchi:2013wra,Seidel:2013sqa}.
The statistical uncertainty on the threshold mass is reduced to below 20~MeV with a scan of
ten times 20~\ifb. The total uncertainty on the $\msb$ mass can be controlled to the level
of 50~MeV.   These systematic uncertainties
include a rigorous evaluation of theory uncertainties in the threshold calculation and in the conversion
to the $\msb$ scheme~\cite{Simon:2016pwp}. A linear $e^+e^-$ collider
can thus achieve a precision that goes well beyond
even the most optimistic scenarios for the evolution 
of the top quark mass measurement at the LHC.


\subsubsection{Top quark electroweak couplings}
\label{subsubsec:highE:topelectroweak}

Composite Higgs models and models with extra dimensions naturally
predict  large corrections to the top quark couplings to the $Z$ and
$W$ bosons~\cite{Richard:2014upa,Barducci:2015aoa,Durieux:2018ekg}.
The study of top quark pair production at an $e^+e⁻$ therefore provide
a stringent test of such extensions of the SM.

The potential of the 500~GeV ILC for the measurement of the cross section and forward-backward asymmetry in
$t\bar{t}$ is characterized in detail in Ref.~\cite{Amjad:2015mma}. It
is important to note that these measurements search for deviations
from the SM in the main production mechanism of the $t\bar t$ system
through $s$-channel $\gamma$ and $Z$ exchange.   With two
configurations of the beam polarization, measurement of the angular
distribution, and measurement variables sensitive to the $t$ and $\bar
t$ polarizations, all 6 possible $CP$-conserving form factors can be
disentangled and constrainted at the 1\% level. 
Especially designed $CP$-odd observables can also provide precise and
specific constraints on the $CP$-violating form factors~\cite{Bernreuther:2017cyi}. 

In principle, the corrections to the top quark electroweak couplings
and to the top quark Yukawa coupling should be parametrized by
dimension-6 operators of the SM EFT.   Just as in the case of the
trilinear Higgs coupling, vertices arising from dimension-6 operators
that do not directly involve the Higgs boson can affect the cross
section for $\ee\to t\bar t H$ and thus create ambiguity in the
extraction of the top quark Yukawa coupling.   In $\ee to t\bar t H$,
the EFT corrections come from 4-fermion $eett$ operators and from
operators that correct the $\gamma$ and $Z$ anomalous moments of the
$t$ quark.   Similarly, in hadron-hadron collisions, the cross section
for $gg\to t\bar t H$ is corrected by dimension-6 operator that alter
the top quark vector coupling to gluons and those which create a
possible axial vector coupling to gluons and a gluonic magnetic
moment.  To extract the top quark Yukawa coupling in a
model-independent way, as opposed to the simple model used in
Sec.~\ref{subsubsec:highE:topYukawa}, the relevant EFT coefficients
need to be measured systematically in other top quark reactions.

A first step in this direction has been taken in
Ref.~\cite{Durieux:2018tev}.   In this paper, the authors consider the
perturbation of the reaction $\ee\to t\bar t$ by the 10 dimension-6
operators that contribute to the cross section at the tree level.
They show that a 
combination of
 the 500~GeV run, with excellent sensitivity to two-fermion operators,
with 1~TeV{} data, with increased sensitivity to four-fermion
operators, yields  tight constraints independently on 
all operator coefficients.  This study demonstrates the feasibility of a global EFT analysis of the top sector
at the ILC.  It also gives an the expected sensitivity of the ILC to
top electroweak couplings that  exceeds that of the HL-LHC programme by one to two orders of
magnitude. Translated into discovery potential for concrete BSM scenarios, a linear collider operated above the top quark
pair production threshold can probe for compositeness of the Higgs
sector to very high scales, up to 10~TeV and
beyond~\cite{Durieux:2018ekg}. 


Similar analyses, now requiring  only 4 relevant dimension-6 operator coefficients, can
improve the constraints on four-fermion operators involving $b$, $c$,
and light-fermion sectors beyond the results projected in
Sec.~\ref{subsec:ew_ffana}.

\subsection{Additional  physics goals of higher-energy running}

The upgrade of the ILC to 500~GeV and, eventually, to 1~TeV will also
improve the ability of the collider to search for new physics effects
in the triple gauge boson vertices and  in $\ee\to ff$ processes with
light fermions.  It will also increase the search reach for new
particles.  For these topics, the increase in sensitivity for the step
to 500~GeV has already been discussed in Secs.~\ref{sec:ew} and
\ref{sec:searches}, respectively.  An energy upgrade to
1~TeV or above would be especially important for searches for extended
Higgs boson and electroweakinos, bringing model-independent search
methods to territory that is beyond even the model-dependent search
reach available from the HL-LHC.



% \subsection{Gauge boson pair production}
% \label{subsec:highE:gauge}


% Measurements of the $\gamma W^+W^-$ and $ZW^+W^-$ triple gauge boson 
% couplings (TGC) test the $SU(2) \times U(1)$
% gauge boson self-coupling structure of the SM and probe BSM physics. 
% A simultaneous fit of three independent couplings and
% the polarization on the 250~GeV{} data is expected to offer a
% substantial 
% improvement beyond the results of LEP 2 and the
% LHC~\cite{Fujii:2017vwa}. At higher energy the sensitivity grows and
% constraints 
% are even tighter. A factor of two in
% precision\cite{Marchesini:2011aka} with respect to the 250~GeV{}
% programme can be 
% achieved with an integrated
% luminosity of 4~\iab at $\sqrt{s}=500$~GeV. Further improvements are
% possible at   
% $\sqrt{s}=1$~TeV~\cite{Rosca:2016hcq}.

% %Ref.~\cite{Fleper:2016frz} analyzes the longitudinal vector boson scattering modes in$e^+e^- \rightarrow \nu \bar{\nu} W^+ W^-$ production., Steve Green's study


% \subsection{Direct searches for new particles}
% \label{subsec:highE:searches}

% In this section, we summarize the potential of high-energy stages of the ILC for directly producing
% new particles. In particular, we highlight cases where the discovery potential of the ILC is strongly enhanced and
% is complementary to that of the LHC. A summary of the most important studies is found in Ref.\cite{Fujii:2017ekh}.
% A comparison to other future collider projects was included in the Snowmass white paper~\cite{Baer:2013vqa}. 

% {\bf Dark matter}

% General WIMP with mono-photon analysis. SUSY dark matter.

% {\bf Extended Higgs sector}

% 2HDM.
% The ILC probes Higgs boson masses up to $\sqrt{s}/2$. It also probes low-mass Higgs bosons.

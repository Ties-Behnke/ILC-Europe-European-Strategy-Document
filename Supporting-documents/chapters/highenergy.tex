
% section on physics above 250 GeV


A key advantage of linear colliders is the possibility to upgrade
the center-of-mass energy.  The energy reach of circular electron-positron
colliders of a given circumference is limited by synchrotron
radiation, and this is difficult to overcome because of the steep
growth of synchrotron losses with energy.  Linear colliders, however,
can be upgraded to reach higher center of mass energies either by
increasing the length of the main linacs or by installing linac components
that support higher accelerating gradients.  

 After finalizing the programme discussed in
Sections~\ref{sec:physics} -- \ref{sec:searches}, the ILC can
be expanded to explore energies well beyond 250~GeV. Our 
discussion of the machine design in Sec.~\ref{sec:ilc} and the 
running scenarios in Sec.~\ref{sec:runscenarios} already incorporates a 
planned energy upgrade to 500~GeV.  Plans for a further energy upgrade
to 1\,TeV were already discussed in the ILC TDR.
  We thus see the ILC as having a long
future with operation at number of stages beyond the initial one, at
increasingly high energies. 

In this section, we will present a further discussion of the
possibilities for energy upgrade of the ILC.  Then, we will describe
the physics program that these energy upgrades will make available.

The physics goals for higher-energy $\ee$ colliders has already been 
discussed extensively in the literature. Particularly useful
references are the volumes presenting  detailed studies carried out for 
the ILC~\cite{\cite{Fujii:2015jha,Baer:2013cma}
and CLIC~\cite{Linssen:2012hp,deBlas:2018mhx,Roloff:2018dqu} design reports. 



%  In this section,
% we review the prospects and physics goals for ILC operation at higher
% energies. 

% {\bf Intro and scope:} While the energy reach of circular electron-positron
% colliders of a given circumference is limited by synchrotron radiation,
% linear colliders can be upgraded to reach higher center-of-mass energy.
% It is therefore natural to envisage a program beyond the baseline program
% at $\sqrt{s}= 250$~GeV discussed in Sections~\ref{sec:physics}, \ref{sec:higgs}
% and \ref{sec:searches}. The energy extendability of linear colliders
% provides great flexibility to respond to new discoveries, at the ILC or elsewhere,
% by adapting the collider design. The exact outline of the high-energy
% program will likely change with the developing insight in the physics of the Standard Model,
% and what lies beyond it. The high-energy program is moreover affected by new developments
% in accelerator technology. This section presents a brief status for the most relevant
% accelerator R\&D lines and a brief review of studies of the physics potential at
% center-of-mass energies greater than 250~GeV.  

% There exists an extensive literature on the subject. We refer in particular
% many detailed studies performed for the ILC~\cite{Fujii:2015jha,Baer:2013cma}
% and CLIC design reports~\cite{Linssen:2012hp,Roloff:2018dqu}.


\subsection{Technical aspects of ILC energy upgrades}
\label{subsec:highE:tech}

In the major particle physics laboratories, the lifetime of collider elements
and infrastructure has rarely been  limited to the scope of the project they were
designed and built for. A famous example is the Proton Synchrotron at CERN.
Initially commissioned in 1959, it is still in operation as
part of the accelerator complex that prepares protons for injection in the
Large Hadron Collider. In this accelerator complex, the  expensive civil engineering
efforts to construct each component are reused, so that their cost is effectively shared. The tunnel
that was constructed for LEP now hosts the Large Hadron Collider and its
luminosity upgrade. In very much the same way, we
expect  that the ILC will  form the seed for a facility that contributes to the
cutting edge of particle physics for decades.


For electron-positron collisions, any facility at energies much higher than
those already realized must be a linear collider in a long, straight
tunnel.
As we have already explained, 
the ILC infrastructure will provide a basis for collisions at 500~GeV
The most obvious energy upgrade path is an extension of the linear
accelerator sections of the colliders, which provides an increase
in center-of-mass energy that is proportional to the length of the linacs.
The design of the ILC presented in the Technical Design
Report~\cite{Behnke:2013xla,Adolphsen:2013jya,Adolphsen:2013kya} envisaged a
center-of-mass energy of 500~GeV{} in a facility with a total length of 31~km.
The ILC TDR also documents
a possible extension to 1~TeV based on current superconducting RF
technology.  We have reviewed the details of this extension 
 in 
Sec.~\ref{subsubsec:upg-optE}.

An even larger increase in center-of-mass energy may be achieved by
exploiting advances in accelerator technology. The development of
cavities with higher accelerating gradient can drive a significant
increase in the energy while maintaining a compact infrastructure.
Superconducting RF technology is evolving rapidly. Important
progress has been made in developing cavities with a gradient well
beyond the 35 MV/m required for 
the ILC~\cite{Grassellino:2018tqg,Grassellino:2017bod}.
and even beyond the 45 MV/m envisaged for
 the 1~Tev{} ILC.     In the longer term, alternate-shape
or thin-film-coated $Nb_3Sn$ cavities
 or multi-layer coated cavities offer the potential of
significantly increased cavity performance~\cite{Adolphsen:2013jya}.
Novel acceleration schemes may achieve even higher gradients. 
The CLIC drive beam concept has
achieved accelerating gradients of up to 100 MV/m~\cite{Aicheler:2012bya}.
Finally, the advent of acceleration schemes based on plasma
wakefield 
acceleration or another advanced concept could 
open up the energy regime up to 30~TeV. A report of the status of accelerator R\&D and remaining
challenges is found in Refs.~\cite{advancedLC2020,advancedLC}, with further details and
a brief description of the potential of such a machine in the
addendum~\cite{advancedLCaddendum}.

The most obvious energy upgrade path is an extension of the linear
accelerator (LINAC) sections of the colliders, which provides an increase
in center-of-mass energy that is proportional to the length of the LINACs.
The design of the ILC presented in the Technical Design
Report~\cite{Behnke:2013xla,Adolphsen:2013jya,Adolphsen:2013kya} envisaged a
center-of-mass energy of 500~GeV{} in a facility with a total length of 31~km.
The reference staging scenario (H20-staged~\cite{Barklow:2015tja}) consists of operation at
three energy stages: 250, 350, and 500~GeV. The ILC TDR also documents
a possible extension to 1~TeV based on current superconducting RF technology. 



An even larger increase in center-of-mass energy may be achieved by
exploiting advances in accelerator technology. The development of
cavities with higher accelerating gradient can drive a significant
increase in the energy while maintaining a compact infrastructure.
Superconducting RF technology is rapidly evolving and important
progress has been made in developing cavities with a gradient well
beyond the 35 MV/m required for the ILC~\cite{Grassellino:2018tqg,Grassellino:2017bod}.
and even beyond the 45 MV/m envisaged for the 1~Tev{} ILC. In the longer term, alternate-shape
or thin-film-coated $Nb_3Sn$ cavities or multi-layer coated cavities offer the potential of
significantly increased cavity performance~\cite{Adolphsen:2013jya}.
Novel acceleration schemes may achieve even higher gradient. The CLIC drive beam concept has
achieved accelerating gradients above  100~MV/m~\cite{Robson:2018enq}.
Finally, the advent of novel acceleration schemes based on plasma wakefield acceleration
may open up the energy regime up to 30~Tev. Recent reports on
 the status of advanced  accelerator R\&D and remaining
challenges can be found in Refs.~\cite{Cros:2017jxp,Cros:2019tns}.

These prospects give us much reason for optimism. In our opinion, the
 construction
of the ILC will allow Japan to host a laboratory
in Asia that will be the global host for experiments with electron
and positron beams that will define the energy frontier into the longer-term future.



\subsection{Physics goals of ILC energy upgrades}
\label{subsec:highE:physics}

Higher energy can enhance the scientific return of the installation in several
ways. It naturally increases the direct discovery reach of the collider. Experience at
previous $e^+e^-$ colliders suggests that for most scenarios it is straightforward to
discover or exclude pair-production of new particles with masses almost up to the
kinematic limit of half the center-of-mass energy~\cite{Fujii:2017ekh}. Additional
Higgs bosons of an extended Higgs sector can be discovered as long as the sum of
the masses is less than the center-of-mass energy. The potential
of the 500~GeV (and 1~TeV) ILC has been demonstrated in detailed simulations
for many important benchmark scenarios~\cite{Fujii:2015jha,Baer:2013cma}, including
scenarios with WIMP dark matter~\cite{Bartels:2012ex} and SUSY scenarios with
a light electroweak sparticle spectrum\cite{Berggren:2013vfa}.

An $e^+e^-$ collider with a center-of-mass energy beyond 250~GeV can also
probe several new Standard Model processes.
Top quark physics starts at the top quark pair production threshold at
$\sqrt{s}\sim 2 m_t$. A Linear Collider with a center-of-mass energy
above this threshold can provide a precise scrutiny of top quark
properties and its (electroweak) interactions.
Two further thresholds are found around 500~GeV, where Higgs boson pair production
and associated production of a Higgs boson with top quarks open up.
The analysis of these processes allow for a much more model-independent
extraction of the Higgs trilinear self-coupling (in $Zhh$ and $\nu \bar{\nu}hh$
production~\cite{Barklow:2017awn}) and the top quark Yukawa
coupling (in $t\bar{t}h$ production~\cite{Yonamine:2011jg}). 
Furthermore, many $t$-channel processes become accessible or even dominant
at higher energy. The cross sections for vector boson fusion production
of the Higgs boson and for vector boson scattering as a component of
di-boson production continue to increase as the center of mass energy
is raised.

Finally, the higher energy can be an important tool in tests of the SM
using Effective Field Theory.   We have emphasized that an analysis
within  EFT allows a more incisive search for new physics effects on
Higgs boson couplings by bringing new observables to bear on  the
analysis.  In the EFT formulae, different operator contributions have different
energy-dependence, with certain operators having an impact that grows 
strongly with energy.  Thus there is great advantage in combining a
data set taken at 250~GeV with one or more data sets taken at higher
energies.

We will expand on all of these points in the following subsections.




\subsection{Higgs physics}
\label{subsec:highE:Higgs}

%The importance of the ILC for Higgs physics has been discussed in
%Section~\ref{sec:higgs}. The study of the Higgsstrahlung process
%at $\sqrt{s}=$ 250~GeV{} represents the core of the program of the ILC.
Higher-energy operation extends the Higgs physics programme in several ways.
The rate for vector-boson fusion production of the Higgs boson increases with
center-of-mass energy. Even more importantly, data taken
at 500~GeV or higher energy opens two new channels that provide direct
access to key couplings, the Higgs self-coupling~\cite{Barklow:2017awn}
and the top quark Yukawa coupling~\cite{Yonamine:2011jg}.
%% assume Higgs couplings are already discussed, include only direct ttH and triple Higgs couplings 

\subsubsection{Vector-boson fusion production of the Higgs boson}
\label{subsubsec:highE:VBFHiggs}

The $WW$-fusion Higgs-production process,
$e^+e^- \rightarrow \nu \bar{\nu} h$, opens up at 500~GeV, which
provides another powerful channel to study Higgs boson couplings. The
addition of this second production channel provides complementary
constraints in a global fit of the Higgs boson couplings.
Global fits of the ILC H20 scenario find that addition of the data
set at $\sqrt{s}=$ 500~GeV improves the precision on most of
couplings by approximately a factor two~\cite{Barklow:2017suo}.
CLIC studies arrive at the same conclusion after an analysis of
the potential of the runs at 1.5~TeV  and
3~Tev~\cite{Abramowicz:2016zbo}.

\subsubsection{Higgs-boson pair production and measurement 
of the trilinear self-coupling}
\label{subsubsec:highE:tripleHiggs}

The value of the trilinear Higgs coupling gives evidence on the nature of the phase
transition  in  the  early  universe  from  the  symmetric  state  of  the  weak  interaction
theory to the state of broken symmetry with a nonzero value of the Higgs field. The
Standard Model predicts a continuous transition. Models with a first-order phase transition,
which may provide an explanation for the baryon asymmetry of the universe,
can predict large (order 100\%) deviations from the SM prediction for the Higgs self coupling.

A high-energy $e^+e^-$ collider can probe the trilinear Higgs self coupling directly in Higgs pair production.
Detailed simulation studies at a center-of-mass energy of 500~GeV show that a discovery of the
double Higgs-strahlung process is possible with less than 4~\iab. A combination of several decay channels
could yield a precision of 27\% on the trilinear Higgs self coupling~\cite{Duerig:2016dvi}.
At still higher energy vector boson fusion becomes the dominant production channel.

%%%[25]  J. Tian, LC-REP-2013-003, http://www-flc.desy.de/lcnotes/notes/LC-REP-2013-003.pdf
%%%[26]  M.   Kurata et  al,   LC-REP-2013-025, http://www-flc.desy.de/lcnotes/notes/LC-REP-2013-025.pdf

The trilinear Higgs self coupling can also be extracted indirectly from a very precise
measurement of the $e^+ e^- \rightarrow Zh$ cross section~\cite{McCullough:2013rea}. Barklow et al.~\cite{Barklow:2017awn}
and Grojean et al.~\cite{DiVita:2017vrr} revisit this claim with a realistic fit of
all relevant EFT parameters, finding that with 5~\iab{} at 240~GeV
the global constraint is very poor, of order (few), while the determination in Higgs boson pair production is robust. 

\subsubsection{Measurement of the top quark Yukawa coupling coupling}
\label{subsubsec:highE:topYukawa}

The top quark Yukawa coupling is one of the key parameters of the Standard Model. 
It can be inferred, indirectly through loop contributions, from the $e^+e^- \rightarrow Zh$
cross section at $\sqrt{s}=  250$~GeV  and the $h \rightarrow gg$, $h \rightarrow \gamma \gamma$
and $h \rightarrow \gamma Z$ partial widths. Especially $h \rightarrow gg$ is very promising, with potentially
better than 1\% precision~\cite{Boselli:2018zxr}. This precision exceeds the HL-LHC expectation by an order of magnitude.
In a realistic multi-parameter fit, the $h \rightarrow \gamma \gamma$, $h \rightarrow \gamma Z$ and $h \rightarrow gg$
partial widths constrain combinations of the top quark Yukawa and other Higgs~\cite{Azatov:2016xik} and top quark
operators~\cite{Vryonidou:2018eyv}. The potential of the 250~GeV data to isolate the Yukawa
coupling in a global analysis remains to be performed.

The top-quark Yukawa coupling can furthermore be extracted from the $t\bar{t}$
cross section very close to the $t\bar{t}$ production threshold. A 4~\% precision can be achieved~\cite{Horiguchi:2013wra}
if the theory prediction can be made more precise (propagation of the scale uncertainty of the current NNNLO calculation yields
an uncertainty of approximately 20~\%~\cite{Vos:2016til} on $y_t$).

A direct determination of the top-quark Yukawa coupling can be performed in $e^+e^- \rightarrow t\bar{t}h$ production.
In this channel the impact of the top-quark Yukawa coupling is accessible in a model-independent fashion. If the
measurements at $\sqrt{s}=250$~GeV yield evidence of a deviation from the SM, the direct measurement provides
an essential confirmation of the result, and valuable additional information to fingerprint the new physics that
is responsible for the effect. The cross section for $t\bar{t}h$ production increases rapidly above $\sqrt{s} \sim 500 $~GeV,
reaching several fb for $\sqrt{s} = 550$~GeV. Detailed studies of selection and reconstruction of these complex multi-jet events
have been performed by the ILC at 500~GeV~\cite{Yonamine:2011jg} and 1~TeV~\cite{Price:2014oca} and by CLIC
at 1.5~TeV~\cite{Abramowicz:2018rjq}. The direct measurement of the top quark Yukawa coupling at the ILC can reach 3\%
precision~\cite{Fujii:2015jha}, with 4~\iab{} at 550~GeV.




\subsection{Top quark physics}
\label{subsec:highE:top}

Among the Standard Model fermions, he top quark stands out for a number of reasons. First, the Higgs boson
and the top quark are the only two Standard Model particles to escape scrutiny at the previous generation
of $e^+e^-$ colliders. Top quark couplings, especially those to the electroweak gauge bosons, are therefore
relatively unconstrained by experiment. Second, in many extensions of the Standard Model, including
supersymmetric scenarios and composite Higgs models, the top sector plays a crucial role in the dynamics
of electroweak symmetry breaking. Top quark precision measurements may therefore provide the first direct signs of new physics.
And, third, the fact that the top quark decays (to a $W$-boson and a $b$-quark) gives access to important information. Top quark
pair production, with a six-fermion final state, is readily distinguished from other Standard Model processses. Top quarks
and anti-quarks can be distinguished efficiently and the top quark polarization and $t\bar{t}$ spin correlations are acccessible. 

The potential of linear $e^+e^-$ colliders for top quark physics is discussed in detail in the ILC design reports~\cite{}
and in Refs.~\cite{Agashe:2013hma,Vos:2016til,Abramowicz:2018rjq}.

\subsubsection{Measurement of the top-quark mass}
\label{subsubsec:highE:topmass}

The top quark mass is a fundamental parameter of the Standard Model, that has to be
determined experimentally. Precise measurements are essential for precise tests of
the internal consistency of the Standard Model, through the electro-weak
fit~\cite{Baak:2014ora} or and extrapolation of the Higgs potential to very high
scale~\cite{Degrassi:2012ry}.

The top quark pair production threshold was identified long ago~\cite{Gusken:1985nf} as
an ideal laboratory to measure the top quark mass, and other properties such as the top quark
width and the Yukawa coupling and the strong coupling constant~\cite{Strassler:1990nw}.
The large natural width of the top quark acts as an infrared cut-off,
rendering the threshold cross section insensitive to the non-perturbative confining part
of the QCD potential and allowing reliable cross section calculation with perturbative QCD
to the NNNLO~\cite{Beneke:2015kwa} and NNLL resummation~\cite{Hoang:2013uda}. Fully
differential results are available in WHIZARD~\cite{Bach:2017ggt}.

The threshold scan involves measurements of the $t\bar{t}$ cross section at ten $e^+e^-$
center-of-mass energy points around the threshold region. A fit of the line shape allows for
a precise extraction of the top quark mass~\cite{Martinez:2002st,Horiguchi:2013wra,Seidel:2013sqa}.
The statistical uncertainty on the threshold mass is reduced to below 20~MeV with a scan of
ten times 20~\ifb. The total uncertainty on the $\msb$ mass can be controlled to the level
of 50~MeV. The extraction from the threshold scan is not affected by the ambiguities in the
interpretation that plague direct mass measurements at hadron colliders. The systematic uncertainties
include a rigorous evaluation of theory uncertainties in the threshold calculation and in the conversion
to the $\msb$ scheme~\cite{Simon:2016pwp}. A linear $e^+e^-$ collider can achieve a precision that goes well beyond
even the most optimistic scenarios for the evolution of the measurements at the LHC.


\subsubsection{Top quark electroweak couplings}
\label{subsubsec:highE:topelectroweak}

In composite Higgs models and models with extra dimensions, large corrections to the top quark couplings to
neutral electroweak gauge bosons are naturally predicted~\cite{Richard:2014upa,Barducci:2015aoa,Durieux:2018ekg}.
The study of top quark pair production at an $e^+e⁻$ therefore form a stringent test of such extensions of the SM.

The potential of the 500~GeV{} ILC for the measurement of the cross section and forward-backward asymmetry in
$t\bar{t}$ is characterized in detail in Ref.~\cite{Amjad:2015mma}. With two configurations of the beam polarization
the contributions of the photon and $Z$-boson are disentangled and very precise constraints on anomalous top quark
couplings are achieved. Especially designed CP-odd observables provide precise constraints on CP-violation in the
top quark sector~\cite{Bernreuther:2017cyi}. Finally, the authors of Ref.~\cite{Durieux:2018tev} perform an EFT fit
with twelve degrees of freedom. A combination of the 500~GeV run, with excellent sensitivity to two-fermion operators,
with 1~TeV{} data, with increased sensitivity to four-fermion operators, is found to yield tight constraints on the
coefficients of all operator coefficients\footnote{The energy dependence of four-fermion operators is also present in
other two-fermion productions processes, such as $e^+e^- \rightarrow b\bar{b}$ production~\cite{Bilokin:2017lco}. It is
therefore expected that also the global constraints in the bottom, charm and light-fermion sectors improve with the addition
of higher-energy data.}. This study demonstrates the feasibility of a global EFT analysis of the top sector
at the ILC and shows that the expected sensitivity of the ILC exceeds that of the HL-LHC programme by one to two orders of
magnitude. Translated into discovery potential for concrete BSM scenarios, a linear collider operated above the top quark
pair production threshold can probe very high scales, up to 10~TeV and beyond~\cite{Durieux:2018ekg}. 

\subsection{Gauge boson pair production}
\label{subsec:highE:gauge}


Measurements of the $\gamma W^+W^-$ and $ZW^+W^-$ triple gauge boson couplings (TGC) test the $SU(2) \times U(1)$
gauge boson self-coupling structure of the SM and probe BSM physics. A simultaneous fit of three independent couplings and
the polarization on the 250~GeV{} data is expected to offer a substantial improvement beyond the results of LEP 2 and the
LHC~\cite{Fujii:2017vwa}. At higher energy the sensitivity grows and constraints are even tighter. A factor of two in
precision\cite{Marchesini:2011aka} with respect to the 250~GeV{} programme can be achieved with an integrated
luminosity of 4~\iab at $\sqrt{s}=500$~GeV. Further improvements are possible at   $\sqrt{s}=1$~TeV~\cite{Rosca:2016hcq}.

%Ref.~\cite{Fleper:2016frz} analyzes the longitudinal vector boson scattering modes in$e^+e^- \rightarrow \nu \bar{\nu} W^+ W^-$ production., Steve Green's study


\subsection{Direct searches for new particles}
\label{subsec:highE:searches}

In this section, we summarize the potential of high-energy stages of the ILC for directly producing
new particles. In particular, we highlight cases where the discovery potential of the ILC is strongly enhanced and
is complementary to that of the LHC. A summary of the most important studies is found in Ref.\cite{Fujii:2017ekh}.
A comparison to other future collider projects was included in the Snowmass white paper~\cite{Baer:2013vqa}. 

% {\bf Dark matter}

% General WIMP with mono-photon analysis. SUSY dark matter.

% {\bf Extended Higgs sector}

% 2HDM.
% The ILC probes Higgs boson masses up to $\sqrt{s}/2$. It also probes low-mass Higgs bosons.

% section on physics above 250 GeV


A key advantage of linear colliders is the possibility to upgrade
the center-of-mass energy.  The energy reach of circular electron-positron
colliders of a given circumference is limited by synchrotron
radiation, and this is difficult to overcome because of the steep
growth of synchrotron losses with energy.  Linear colliders, however,
can be upgraded to reach higher center of mass energies either by
increasing the length of the main linacs or by installing linac components
that support higher accelerating gradients.  

After the planned data-taking at 250~GeV, the ILC can
be expanded to explore higher center of mass energies.
discussion of the machine design in Sec.~\ref{sec:ilc} and the 
running scenarios in Sec.~\ref{sec:runscenarios} already incorporates a 
planned energy upgrade to 500~GeV.  Plans for a further energy upgrade
to 1\,TeV were already discussed in the ILC TDR.
  We thus see the ILC as having a long
future with operation at number of stages beyond the initial one, at
increasingly high energies. 

In this section, we will brief describe the further physics
opportunities that the ILC will offer at 350~GeV, 500~GeV, and beyond.
The physics goals for higher-energy $\ee$ colliders has already been 
discussed extensively in the literature. Particularly useful
references are the volumes presenting  detailed studies carried out for 
the ILC~\cite{Fujii:2015jha,Baer:2013cma}
and CLIC~\cite{Linssen:2012hp,deBlas:2018mhx,Roloff:2018dqu} design reports. 




\subsection{Planning for the ILC energy upgrades}
\label{subsec:highE:tech}

The ILC project  being considered today would be an $\ee$ colllider of
250~GeV in the center of mass.  However, considerable thought and
planning has already gone into the extension of this machine to higher energies.
In the major particle physics laboratories, the lifetime of collider elements
and infrastructure has rarely been  limited to the scope of the project they were
designed and built for. A famous example is the Proton Synchrotron at CERN.
Initially commissioned in 1959, it is still in operation as
part of the accelerator complex that prepares protons for injection in the
Large Hadron Collider. In this accelerator complex, the  expensive civil engineering
efforts to construct each component are reused, so that their cost is effectively shared. The tunnel
that was constructed for LEP now hosts the Large Hadron Collider and its
luminosity upgrade. In very much the same way, we
expect  that the ILC will  form the seed for a facility that contributes to the
cutting edge of particle physics for decades.


For electron-positron collisions, any facility at energies much higher than
those already realized must be a linear collider in a long, straight
tunnel.
As we have already explained, 
the ILC infrastructure will provide a basis for collisions at 500~GeV
The most obvious energy upgrade path is an extension of the linear
accelerator sections of the colliders, which provides an increase
in center-of-mass energy that is proportional to the length of the linacs.
The design of the ILC presented in the Technical Design
Report~\cite{Behnke:2013xla,Adolphsen:2013jya,Adolphsen:2013kya} envisaged a
center-of-mass energy of 500~GeV{} in a facility with a total length of 31~km.
The ILC TDR also documents
a possible extension to 1~TeV based on current superconducting RF
technology.  We have reviewed the details of this extension 
 in 
Sec.~\ref{subsubsec:upg-optE}.

An even larger increase in center-of-mass energy may be achieved by
exploiting advances in accelerator technology. The development of
cavities with higher accelerating gradient can drive a significant
increase in the energy while maintaining a compact infrastructure.
Superconducting RF technology is evolving rapidly. Important
progress has been made in developing cavities with a gradient well
beyond the 35 MV/m required for 
the ILC~\cite{Grassellino:2018tqg,Grassellino:2017bod}.
and even beyond the 45 MV/m envisaged for
 the 1~Tev{} ILC.     In the longer term, alternate-shape
or thin-film-coated $Nb_3Sn$ cavities
 or multi-layer coated cavities offer the potential of
significantly increased cavity performance~\cite{Adolphsen:2013jya}.
Novel acceleration schemes may achieve even higher gradients. 
The CLIC drive beam concept has
achieved accelerating gradients of up to 100 MV/m~\cite{Aicheler:2012bya}.
Finally, the advent of acceleration schemes based on plasma
wakefield 
acceleration or another advanced concept could 
open up the energy regime up to 30~TeV. A report of the status of accelerator R\&D and remaining
challenges is found in Refs.~\cite{advancedLC2020,advancedLC}, with further details and
a brief description of the potential of such a machine in the
addendum~\cite{advancedLCaddendum}.

These prospects give us much reason for optimism. In our opinion, the
 construction
of the ILC will allow Japan to host a laboratory
in Asia that will be the global host for experiments with electron
and positron beams that will define the energy frontier into the longer-term future.



\subsection{Improvement of ILC precision at higher energy}
\label{subsec:highE:physics}

Operation of the ILC at higher energies will produce new data sets
that will substantially improve the capabilites of the ILC for all of
the physics topics presented in Sec.~\ref{sec:physics}.    The ILC
simulation studies included extensive studies at 500~GeV 
 and also studies at 1~TeV in the center of mass.   We will present
 the results of these studies together with our studies from 250~GeV in
 the following sections.

Data-taking at higher energies will improve the results from 250~GeV
and give access to new SM reactions.   Let us first summarize the
expected
improvements in the areas that we have discussed so far:
\begin{itemize}
\item  For Higgs production, running   at 500~GeV will add a new data set
 of  $\ee\to ZH$ events.  It will also provide   a substantial data set of $\ee\to
 \nu\bar\nu H$ events, corresponding to $WW$ fusion production of the
 Higgs boson.   With these new samples, it will be possible to confirm
 any anomalies in the Higgs boson coupling seen at 250~GeV and to
 provide an independent comparison of the $ZZ$ and $WW$ couplings.
Though the backgrounds to the Higgs production processes are
relatively small in for both reactions, they are different in the two
cases, providing a nontrivial check of some systematics. In global
analysis, as we will see in Sec.~\ref{sec:global}, the addition of
500~GeV data leads to a decrease in the uncertainties on Higgs boson
couplings by about a factor of 2. 

\item For $WW$ production, running at 500~GeV will give a data set of
  roughly the same size as that obtained at 250~GeV.  Further, since
  the effects of anomalous $W$ couplings, or the corresponding
  dimension-6 operators, increase as $s/m_W^2$, the new data set will
  provide much more sensitive constraints on their effects.

\item For $f\bar f$ production, similarly, the possible new effects   
  due to heavy gauge bosons or contact interactions scale as $s/M^2$,
  where $M$ is the new mass scale.   The discovery potential for $M$,
  or new limits, will increase by a factor close to 2. 

\item For new particle searches, the reach in $\ee$ pair production is
  close to half the $\ee$  center of mass energy. The  improvement in
  reach is particularly relevant for color-singlet particles such as
  heavy Higgs bosons, electroweakinos, Higgsinos, and dark matter
  particles.  
\end{itemize}

All of these observations illustrate the more general point that
higher energy can be an important tool in tests of the SM
using Effective Field Theory.   We have emphasized that an analysis
within  EFT allows a more incisive search for new physics effects on
Higgs boson couplings by bringing together a large number of
observables from different physical processes.
 In the EFT formulae, the various operator contributions have different
energy-dependence, with certain operators having an impact that grows 
strongly with energy.  Thus there is great advantage in combining a
data set taken at 250~GeV with one or more data sets taken at higher
energies.

\subsection{New Higgs physics at higher energy}

Beyond the improvement in areas that we have already discussed, the
operation of the ILC above 250~GeV can give access to new and
important SM reactions.   Among Higgs boson couplings, there are two
that are 
inaccessible at 250~GeV.  These are the top quark Yukawa coupling and
the Higgs boson self-coupling.   in Secs.~\ref{subsec:higgsself}
and \ref{subsec:top:topYukawa}, we will describe the measurement of
these couplings at the ILC at 500~GeV and above.   Since these
couplings can show large deviations from the SM expectation in certain
classes of new physics models, it is necessary to measure these
couplings accurately to complete the full picture of the Higgs boson
interactions.

The interest in the top quark Yukawa coupling is obvious.   The top
quark is the heaviest SM particle, and, within the SM, its mass is
proportional to this coupling constant.   If there are new
interactions that promote the large value of the top quark mass, the
Yukawa coupling will receive corrections, and so it is important to
probe for them.

The measurement of the trilinear Higgs coupling is an equally  important goal
of a complete program of study for the Higgs boson.  While the
measurement of the Higgs field vacuum expectation value and the mass
of the Higgs boson express the mass scale of the Higgs field potential
energy and its variation, the trilinear Higgs coupling gives
information on the shape of the potential energy function and brings
us closer to understanding its origin.

The trilinear Higgs coupling is sensitive to the nature of the phase
transition in the early universe that led to the present state of
broken electroweak symmetry.  
The  SM predicts a continuous phase transition.   This has
implications for models of the creation of the matter-antimatter
asymmetry that we observe in the unverse today. According to
Sakharov's classic analysis~\cite{Sakharov:1967dj}, the net baryon number of the universe
needed to be created in an epoch with substantial deviations from
thermal equilibrium in which $CP$- and baryon-number-violating
interactions were active.    The baryon number of the universe could
have been created at the electroweak phase transition, making use of new
$CP$-violating interactions in the Higgs sector, but only if the phase
transition was strongly first-order.   In explicit models, this
typically requires large deviations of this coupling, by a factor
1.5--3, from its SM value~\cite{Morrissey:2012db}.





\subsection{Study of the top quark in $\ee$ reactions}

In addition, the extension of the ILC to 350 and 500~GeV will allow the precision
study of the top quark.   This is an essential goal of precision
experiments on the SM, for two reasons.  First, similarly to the Higgs
boson, the top quark stands closer to the essential mysteries of the
SM than any other quark or lepton.   It is heavier than the next
lighter fermion, the $b$ quark, by a factor of 40 and heavier than the
lightest quark, the $u$ quark, by a factor of $10^5$.  The reasons for
this are unknown, but they must be related to other mysteries of the
Higgs sector and SM mass generation.  In fact, it is not understood
whether the top quark is ``heavy'' quark because of special
interactions that the other quarks do not share or, alternatively, whether the top
quark is an ``ordinary'' quark receiving an order-1 mass while the
masses of the other quarks are highly suppressed.  Competing extensions of the
SM such as supersymmetry and composite Higgs models differ in their
answers to this question.   

Second, the fact that the top quark has spin, couples to the
parity-violating weak interactions, and decays to nonzero spin
particles through  $t\to bW$ gives a large number of independent
observables for each $t\bar t$ production process. An $\ee$ collider
with beam polarization can take advantage of all of these observables,
especially if it can produce $t\bar t$ well above threshold at
500~GeV in the center of mass.  

Thus, top quark physics is a place in which we expect to find
deviations from the predictions of the SM, in a setting where we have
many handles to search for these new physics effects.

In the following sections, we will present the capabilities of the ILC
in all of these areas.  Our discussion will be based on 
explicit simulation studies using the accelerator properties
and run plan  described above and the detector models to be presented
in  Sec.~\ref{sec:detectors}.
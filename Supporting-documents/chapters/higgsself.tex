


The  trilinear Higgs coupling can be measured at colliders in two
different ways.   First, the coupling can be measured directly, using 
processes with Higgs pair production
that  diagrams that involve the triple Higgs coupling.  Second, the
coupling can be measured indirectly, since
radiative corrections to single-Higgs processes can include effects
due to the trilinear Higgs coupling.

The important Higgs pair production reactions at $\ee$ colliders are 
$\ee\to ZHH$ and $\ee\to \nu\bar\nu HH$.    The first of these
processes can be studied already at 500~GeV; the second, which is a
4-body process, requires
somewhat higher energy
Detailed simulation studies at a center-of-mass energy of 500~GeV show that a discovery of the
double Higgs-strahlung process is possible within the H20 program.
With 4~\iab\ at 500~GeV, a
a combination of several decay channels
would yield a precision of 16\% on the total cross section for
$\ee\to ZHH$~\cite{Duerig:2016dvi}.   Assuming the SM with only the
trilinear Higgs coupling free, this corresponds to an uncertainty of
27\% on that coupling.

At still higher energy vector boson fusion becomes the dominant
production channel. Making use of this channel, with  8~\iab\ at
1~TeV, the studies \cite{TianHHH,Roloff:2019crr} show that, in the
same context of varying the trilinear Higgs coupling only, this
coupling can be determined to better than 10\%. 

%%%[25]  J. Tian, LC-REP-2013-003, http://www-flc.desy.de/lcnotes/notes/LC-REP-2013-003.pdf
%%%[26]  M.   Kurata et  al,   LC-REP-2013-025,
%%%http://www-flc.desy.de/lcnotes/notes/LC-REP-2013-025.pdf


The indirect determination of the trilinear Higgs coupling is based on
the observation of McCullough~\cite{McCullough:2013rea} that the cross
section for $\ee\to ZH$ contains a radiative correction involving the
trilinear coupling that lower the cross section by about 1.5\% from
250~GeV to 500~GeV, with most of the decrease taking place below
350~GeV.  Taken a face value in the simple context with only the
trilinear coupling free, the ILC cross section measurements would determine the trilinear 
coupling to about 40\% [needs checking].

It is important to note, however, that the determination of the
trilinear coupling involves two separate questions.  First, is the SM
violated?   The accuracies with which this question can be answered
are those given above.  Second, can the violation of the SM be
attributed to a change in the trilinear coupling or the Higgs
potential rather than being due to other possible new physics
effects.  A precise way to ask this question is: Can the shift of the
trilinear coupling be measured  independently of possible effects of
all 
other dimension-6 EFT operators?   To our knowledge, this latter
question has only been addressed for determinations of the trilinear
coupling at lepton colliders.   In Ref.~\cite{Barklow:2017awn} it is
shown that, after the ILC H20 program of single-Higgs measurements is
complete, the uncertainty in the measurement of the total cross
section for  $\ee\to ZHH$ receives a negligible 2.5\% uncertainty due
to variation of the other relevant dimension-6 EFT perturbations.   In
Ref.~\cite{DiVita:2017vrr}, it is shown that, when the cross section
for $\ee\to ZH$ is fit together with other relevant observables at
250~GeV and 500~GeV
in a fit includes all other relevant EFT perturbations, the
uncertainty in the coupling is not substantially changed from the value of
40\%  [needs checking] quoted above.   These two
independent determinations of the trilinear couplings can be combined
into a determination of the trilinear coupling  with an accuracy of 23\% that is
also completely model-independent.  The cross section for $\ee\to ZHH$
increases as the  trilinear coupling increases.   So, if the trilinear coupling were
indeed 2 times its SM value, as expected in models of electroweak
baryogenesis, this effect would be observed at 6~$\sigma$ [needs
checking] and the value of ther trilinear determined to 15\% [needs
checking]. 




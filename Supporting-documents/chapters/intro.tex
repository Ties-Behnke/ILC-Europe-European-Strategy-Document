%  Introduction to the document
While the Standard Model (SM) is a highly successful theory of
the fundamental interactions, shortcomings are known and new physics
is required to address them.  A central focus of particle physics
now involves
searching for this physics in the form
of new particles and/or new interactions.  The SM is theoretically
 self-consistent, but it does not explain dark matter or dark energy
 as observed in the cosmos,
and it cannot explain the excess of matter over antimatter.   It does
not address the mass scale of quarks, leptons, and gauge 
bosons, which is significantly lower than the Planck scale.   
Specifically, it does not explain
the large mass ratios among the SM particles.   
These and other considerations provide a compelling motivation
for the search, while the success of the SM indicates
the degree of challenge this search faces.
The first evidence for the new physics
may appear in deviations to SM predictions 
in high-precision measurements. 
Such evidence could be the first clues to what lies beyond the SM.

The SM was completed in 2012 with the discovery of the Higgs boson.
Its role in electroweak symmetry-breaking and the mechanism of
mass generation for all known fundamental particles places the Higgs boson
in a central role for the SM.  Its properties are precisely
specified by the SM, making precision measurements of these
properties particular sensitive to the new physics.
 
The Higgs boson properties are the first significant goal for
experimentation at the International Linear Collider (ILC).
A large, world-wide community of physicists is working to realise
the exceptional physics program
of electron-positron collisions with the ILC.
The ILC has been designed to optimally address the
central physics issues described above.  The ILC's unprecedented, model-independent measurement precision
provides a leap in sensitivity to the deviations.
At the first stage of the ILC the well-defined SM specifications
of the Higgs boson properties will be tested in great detail.
These measurements 
discriminate between the SM and many proposed extensions of the SM.
The ILC will be sensitive to invisible and other exotic Higgs decays,
testing new physics models including models of dark
matter.  With no new particles beyond the SM having been
discovered at the LHC, the search for new physics through
these high-precision studies is urgent and compelling. 
ILC will also test other SM expectations 
and search for direct evidence of new physics
in pair-production of weakly interacting particles.

Using an $\ee$ linear collider to explore this scientific arena
brings several excellent experimental properties to the
challenge of the measurements.
The ILC has a well-defined, adjustable centre-of-mass energy 
in the production and faces a small background level.
In addition, both colliding beams offer high levels of polarisation
that introduces additional observables opening access to
important properties, such as
the electroweak couplings of right-handed
fermions, 
which are largely unconstrained today.
After operation at the starting energy,
the linear collider can undergo an energy upgrade in a fairly
straightforward way; this will expand the list of physics processes that
can be studied with precision, providing additional pathways
to finding evidence of the new physics.
These energy upgrades can access the properties of the top quark,
including top-quark Yukawa coupling, and to the Higgs self-coupling.
For centre-of-mass energies above the top-quark pair-production threshold
the ILC becomes a top-quark factory.
As the energy rises, new BSM particles of mass up to half
the centre-of-mass energy will be searched for.
Even higher $Z'$ mass reach of up to 10 TeV
 is achieved through its 
mixing with the  $Z$ in the intermediate state.
Because of this upgrade capability and the unique reach, the ILC may
continue to be a leading discovery machine in the 
world of particle physics for decades.
This method of searching for 
new physics beyond the SM is orthogonal to and complements 
the LHC physics program.


The ILC technology is mature and construction-ready, following
more than twenty years of a worldwide-coordinated 
research program to develop the technology required.
As the linear collider technology was developing
decades ago,
committees of   the International Committee for Future Accelerators
(ICFA) guided its successive stages.
In the mid-1990's various technology options to
realise a high-energy linear collider were emerging. 
ICFA asked the 
Linear Collider Technical Review Committee to develop a standardised
way to  compare  these  technologies based on their parameters, such as
power consumption and luminosity. A second
review panel was organised by ICFA in 2002;
it concluded that both warm and cold technologies had
developed to the point where either could be the basis for a
high energy linear
collider. In 2004, the  International Technology Review Panel
(ITRP) was charged by ICFA to recommend an option that could focus the
worldwide R\&D effort.  This panel chose the  superconducting
radiofrequency technology (SCRF), in a large part due to its
energy efficiency and potential for broader applications. 


The collider design is the result of nearly twenty years of
R\&D. The heart of the ILC, the SCRF cavities, is based on
pioneering work of the TESLA Technology Collaboration. Other aspects of the 
technology
emerged from the R\&D carried out for the JLC/GLC and NLC projects,
which were based on room-temperature accelerating structures. 
The effort to design and
establish the technology for the linear collider culminated in the
publication of the Technical Design Report (TDR) for the International
Linear Collider (ILC) in 2013~\cite{Behnke:2013xla}. 
Twenty-four hundred (2400) scientists, from 48 countries and 392 institutes and university
groups,
 signed the TDR,
 that presented optimised collider and detector designs, and associated 
physics analyses. 
From
2005 to the publication of the TDR, the
design of the ILC accelerator was conducted under the mandate of ICFA
as a worldwide
international collaboration, the Global Design Effort (GDE). 
Since 2013, ICFA has placed the  international activities for both the ILC and CLIC
projects under a single organisation, 
the Linear Collider Collaboration (LCC),

So, the ILC proposal
is supported by extensive R\&D and prototyping. The successful construction and
operation 
of the European XFEL at DESY provides
confidence both in the high reliability of the basic
technology and in the reliability of its performance and cost in 
industrial realisation.   
Some specific optimisations and technological choices remain.
But the ILC is now ready to move forward to construction. 


With knowledge of the mass of the Higgs boson, it was established in 2012 that the
linear collider could start its ambitious physics program with an initial centre-of-mass energy of 250 GeV at a cost
reduced from the TDR. A revised design of the ILC, the ILC250, was
thus  presented~\cite{Evans:2017rvt} retaining the final-focus and
beam-dump capability to extend the centre-of-mass energy to higher
energies. Advances in the theoretical understanding of the impact of precision
measurements at the 
 ILC250 have justified that this operating point already gives
 substantial 
sensitivity to physics beyond the SM~\cite{Barklow:2017suo,Fujii:2017vwa}. 
 The cost estimate for ILC250 
  is similar in scale to the LHC project cost.


In its current
form, the ILC250 is a $250\,{\mathrm{GeV}}$ centre-of-mass energy
(extendable up to $1\,{\mathrm{TeV}}$) linear $e^+e^-$ collider, based
on $1.3\,{\mathrm{GHz}}$ SCRF
cavities. It is designed to achieve a luminosity of $1.35\cdot
10^{34}~{\mathrm{cm}}^{-2}{\mathrm{s}}^{-1}$ and provide an integrated
luminosity of $400\,{\mathrm{fb}}^{-1}$ in the first four years of
running. The electron beam will be polarised to $80\,\%$, and the baseline plan includes an 
undulator-based
positron source which will  deliver
$30\,\%$ positron  polarisation. 


The experimental community has developed
designs for two complementary detectors, ILD and SiD, 
as described in \cite{Behnke:2013lya}. These detectors are designed to 
optimally address the
ILC physics goals, with complementary approaches. One detector is based on
TPC tracking (ILD) and one on silicon tracking (SiD).
Both employ particle flow calorimetry based on
calorimeters with unprecedented fine segmentation.
As with the collider technology, 
extensive R\&D and prototyping gives confidence that the
unprecedented,
high-precision detectors envisaged to achieve the 
physics programme can be realised.
An extensive course of
prototyping underlies the estimates of full-detector performance 
and cost.  
The detector R\&D program leading to these designs
has 
contributed a number of advances in 
detector capabilities with applications well beyond the linear
collider program. 
As for the
collider, optimisations and final technological choices 
will need to be completed in the next few years. 

There is broad interest in Japan to host the
international effort to realise the ILC project.  
This interest has been growing over many years.
Political entities are promoting the plan to host,
including the Japanese Diet,
a large industrial consortium (AAA),
and the representatives of the particle physics community (JAHEP).
Detailed reviews in Japan of the many aspects of the
project is nearing a conclusion.
First,
since 2013 the MEXT ministry has examined the ILC project in
great detail, including the aspect of risk minimisation.
This concluded when
MEXT's ILC Advisory Panel released 
its report~\cite{AdvPanel} on July 4, 2018, summarising the
studies of the several working groups (WG) that
reviewed 
a broad range of aspects of the ILC.  The most recent studies include
a specific review of the scientific merit and the technical design for the ILC250. 
The  Physics WG scrutinised the scientific merit of the ILC250,
leading to their strong and positive statement on the importance of
the ILC250 to 
measure precisely the couplings of the Higgs boson \cite{AdvPanel}.
The TDR WG reviewed issues addressed in the Technical Design Report
and the ILC250 design, including the  cost estimate and technical feasibility.  
Other working groups of the MEXT review commented on manpower needs, 
organisational aspects, and the experience of previous large projects.
The report of the ILC Advisory Panel was followed by the beginning of
deliberations in a committee and technical working group 
established by the Science Council of Japan (SCJ),
the second stage of the review process. 
The SCJ review is nearing completion.
  The Japanese government is preparing for the next step,
a decision on hosting, and to move to the next phase of international negotiations.
It is an important aspect of the discussions of the ILC in Japan that the
ILC is seen as global project that will foster exchange between Japan
and other nations.   Thus, the  
scientific interest and political engagement of partner countries is a
major 
concern for the Japanese authorities.  
Another independent committee (ILC Liaison Council),
led by leaders of the Liberal Democratic 
Party, the majority party in the Diet,  has now 
convened to encourage the national government to proceed with the ILC.
The project could start within a few years. 
The initial phase of a potential timeline 
would last about four years to obtain
 international agreements, prepare construction and form 
 the requisite international
collaboration; the construction phase would need nine years.


This report details the current status of this effort, describing
the physics reach, the technological maturity of the accelerator,
detector, and software/computing designs,
plus a discussion on the further steps 
 needed to realise the project.
 The ILC accelerator design and technology is described
 in Section \ref{sec:ilc}.  Many aspects of the development of
 the SCRF is presented, as well as general aspects
 of the machine. Luminosity and energy upgrade options are
 discussed, as well as civil engineering plans, including site
 specific details, and the cost and schedule estimates.
 Section \ref{sec:runscenarios}  presents the current
 thinking about the operations of the ILC, estimating 
 collection of integrated luminosity.
 Section \ref{sec:physics} details the physics case for the ILC.
 This includes an overview of the significance of the Higgs boson 
 as a tool for searching for BSM physics,  
 the details of the Higgs boson studies at the ILC, 
 complementarity to the LHC, and examples of other physics
 of interest including SM fermion pair-production and
 new particle pair production.  Section \ref{sec:detectors}
 provides detailed descriptions of the ILC detector designs
 that have been developed by the community,
 through detector R\&D and prototyping, and used as detector
 models to show the simulated performance on the various
 physics channels. Section \ref{sec:software}
 summarises the computing needs of the ILC program,
 including software.    Section \ref{sec:higgs}
 gives a description of many physics simulations
 that have been completed based on the detectors of
 Section \ref{sec:detectors} to understand the
 reach of the physics program.
 The possible ILD physics program for energies
 beyond 250 GeV are described in
 Section \ref{sec:searches}.


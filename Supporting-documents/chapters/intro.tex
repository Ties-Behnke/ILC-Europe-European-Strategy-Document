%  Introduction to the document

While the Standard Model (SM) is a highly successful theory of
the fundamental interactions, it has serious shortcomings. New
fundamental interactions are 
required to address them.   A central focus of particle physics
now involves
searching for these new interactions and associated new particles.  
The SM is theoretically
 self-consistent, but it does not answer many obvious questions about
 particle physics.  It has no explanation for the 
dark matter or dark energy
that is observed in the cosmos,
or for  the cosmic excess of matter over antimatter.   It does
not address the mass scale of quarks, leptons, and gauge 
bosons, which is significantly lower than the Planck scale.   
It does not explain
the large mass ratios among the SM particles.   
These and other considerations provide a compelling motivation
for new interactions beyond the SM.   On the other hand,
the current success of the SM indicates  that furthere search will be
very challenging and, most likely, requires new approaches and new methods.

The SM was completed in 2012 with the discovery of the Higgs boson.
The Higgs boson is the agent for  electroweak symmetry breaking
and the generation of the masses of all known elementary
particles.  Thus, it occupies a central role in the SM and,
specifically, in the unresolved issues that we have listed above. 
 The properties of the Higgs boson are precisely specified
in the SM, while models of new interactions that address these issues
lead to
significant corrections to those predictions.  Thus, high-precision
measurement of the Higgs boson offers a new and promising 
 avenue for searches for
new physics beyond the Standard Model.   The discovery of deviations
of the Higgs boson properties from the SM predictions could well
provide the first evidence for new physics beyond the SM. 
 
This study of the Higgs boson properties is the most prominent goal of
the International Linear Collider (ILC).   The ILC has been designed
with this goal in mind, to provide a complete, high-precision picture
of the Higgs boson and its interactions.  Though the properties of the
Higgs boson are already being studied at the LHC, the ILC offers
significant advantages.  It will bring the measurements
to a new, qualitatively superior, level of precision, and it will
remove the many model-dependent assumptions required for the analysis
of the Higgs boson measurements at hadron colliders.   The ILC will be highly
sensitive to Higgs boson decays that yield invisible or
other exotic final states, giving unique tests of models of new weakly
interacting particles and dark matter. The ILC can also probe for
direct pair-production of particles with very weak interactions.  Since direct searches
at high-energy hadron colliders have not discovered new particles, it
is urgent and compelling to open this new path to the search for
physics beyond the SM.

As an $\ee$ linear  collider, the ILC brings a number of very powerful
experimental tools to bear on the challenge of producing a precise,
model-independent accounting of the Higgs boson properties.
The ILC has a well-defined, adjustable centre-of-mass energy.   It
produces conventional SM events at a level that is comparable to,
rather than overwhelmingly larger than, Higgs signal processes,
allowing easy selection of Higgs boson events.   At its initial stage
of 250~GeV, Higgs boson events are explicitly tagged by a recoil $Z$
boson.    At a linear collider, both the electron and positron beams
can be polarized, introducing additional observables.  Because all
electroweak  reactions at energies above the $Z$ resonance have
order-1  parity violation, beam polarization effects are large  and
provide access to critical physics information. 

After operation of a linear collider at the starting energy, it is
straightforward to upgrade the centre-of-mass energy.  This is the
natural path of evolution  for a new high-energy physics laboratory.
An upgrade in energy
systematically expands the list of physics processes that can be
studied with high precision and polarized beams.   An upgrade to
500~GeV
accesses the Higgs boson coupling to the top quark and the Higgs boson
self-coupling.  Together with the 250~GeV results, this  will give
 a complete accounting of the Higgs boson
profile.
An energy upgrade to 350~GeV begins the use of the ILC as a top quark
factory,
offering precision measurements of the top quark mass and electroweak
couplings.  At the same time, the ILC will study the reactions $\ee\to
f\bar f$ and $\ee\to W^+W^-$ with high precision.   Here also,
deviations from the SM predictions can indicate new physics.  In the
case of fermion pair production, this capabilities includes searches
for $Z'$ resonances up to 10~TeV in mass.   Finally, the ILC will
search directly for pair production of  weakly coupled particles  with
masses up to half
the centre-of-mass energy, without the requirement of special
signatures needed for searches at hadron colliders. 
Because of its upgrade capability and the unique access that $\ee$ beams give
to many important reactions, the ILC will 
continue to be a leading discovery machine in the 
world of particle physics for decades.


The ILC technology is mature and construction-ready.   The technology
of the ILC has been advanced through a global program coordinated by
the International Committee for Future Accelerators
(ICFA).
In the mid-1990's, various technology options to
realise a high-energy linear collider were emerging. 
ICFA asked the 
Linear Collider Technical Review Committee to develop a standardised
way to  compare  these  technologies based on their parameters, such as
power consumption and luminosity. A second
review panel was organised by ICFA in 2002;
it concluded that both warm and cold technologies had
developed to the point where either could be the basis for a
high energy linear
collider. In 2004, the  International Technology Review Panel
(ITRP) was charged by ICFA to recommend an option that could focus the
worldwide R\&D effort.  This panel chose the  superconducting
radiofrequency technology (SCRF), in a large part due to its
energy efficiency and potential for broader applications. 

Today's design of the ILC accelerator is the result of 
nearly twenty years of
R\&D that has involved a broad, global community. 
The heart of the ILC, the SCRF cavities, is based on
pioneering work of the TESLA Technology Collaboration. Other aspects of the 
technology
emerged from the R\&D carried out for the JLC/GLC and NLC projects,
which were based on room-temperature accelerating structures. 
The ILC proposal
is supported by extensive R\&D and prototyping. The successful construction and
operation 
of the European XFEL at DESY provides
confidence both in the high reliability of the basic
technology and in the reliability of its performance and cost in 
industrial realisation.  Other communities acknowledge this; the SCRF 
 technology has also been chosen for new
free electron laser projects now under construction in the US and China.
Some specific optimisations and technological choices remain.
But the ILC is now ready to move forward to construction. 

The effort to design and
establish the technology for the linear collider culminated in the
publication of the Technical Design Report (TDR) for the International
Linear Collider (ILC) in 2013~\cite{Behnke:2013xla}. 
Twenty-four hundred (2400) scientists, from 48 countries and 
392 institutes and university
groups,
 signed the TDR.  This document
presented optimised collider and detector designs, and associated 
physics analyses based on their  expected performance.
From
2005 to the publication of the TDR, the
design of the ILC accelerator was conducted under the mandate of ICFA
as a worldwide
international collaboration, the Global Design Effort (GDE). 
Since 2013, ICFA has placed the  international activities for both the ILC and CLIC
projects under a single organisation, 
the Linear Collider Collaboration (LCC).


With knowledge of the mass of the Higgs boson, it became clear that
the 
linear collider could start its ambitious physics program
 with an initial centre-of-mass energy of 250~GeV at a cost
reduced from the TDR. A revised design of the ILC, the ILC250, was
thus  presented~\cite{Evans:2017rvt}.  This design  retains the final-focus and
beam-dump capability to extend the centre-of-mass energy to energies
as high as 1~TeV. Advances in the theoretical understanding of the impact of precision
measurements at the 
 ILC250 have justified that the 250~GeV operating point already gives
 substantial 
sensitivity to physics beyond the SM~\cite{Barklow:2017suo,Fujii:2017vwa}. 
 The cost estimate for ILC250 
  is similar in scale to that of the LHC.


In its current
form, the ILC250 is a $250\,{\mathrm{GeV}}$ centre-of-mass energy
(extendable up to a 1~TeV) linear $e^+e^-$ collider, based
on $1.3\,{\mathrm{GHz}}$ SCRF
cavities. It is designed to achieve a luminosity of $1.35\cdot
10^{34}~{\mathrm{cm}}^{-2}{\mathrm{s}}^{-1}$ and provide an integrated
luminosity of $400\,{\mathrm{fb}}^{-1}$ in the first four years of
running.  The scenario described in Section III gives a complete
program of $2\,{\mathrm{ab}}^{-1}$  of data at 250~GeV over 12 years.
The electron beam will be polarised to $80\,\%$, and the baseline plan includes an 
undulator-based
positron source which will  deliver
$30\,\%$ positron  polarisation. 


The experimental community has developed
designs for two complementary detectors, ILD and SiD.  These detectors
are described in 
 \cite{Behnke:2013lya}. They are designed to 
optimally address the
ILC physics goals, with complementary approaches. One detector is based on
TPC tracking (ILD) and one on silicon tracking (SiD).
Both employ particle flow calorimetry based on
calorimeters with unprecedented fine segmentation.
As with the collider technology, 
extensive R\&D and prototyping gives confidence that the
unprecedented levels of performance in calorimetry, tracking, and
heavy particle identification 
required to achieve the 
physics programme can be realised.
The  extensive course of
prototyping justifies our estimates of full-detector performance 
and cost.  
The detector R\&D program leading to these designs
has  contributed a number of advances in 
detector capabilities with applications well beyond the linear
collider program.    Similarly to the situation for the collider, some 
final optimizations and technology choices 
will need to be completed in the next few years. 

There is broad interest in Japan to host the
international effort to realise the ILC project.  
This interest has been growing over many years.
Political entities promoting the plan to host the ILC in Japan include 
the Federation of Diet Members for ILC and the Advanced Accelerator
Association, a consortium of industrial representatives that includes
most of the large high-tech companies in Japan.   The ILC has been
endorsed by the community of Japanese particle physicists (JAHEP)~\cite{Asai:2017pwp}. 
Detailed review in Japan of the many aspects of the
project is nearing a conclusion.
Since 2013 the MEXT ministry has been examining the ILC project in
great detail, including the aspect of risk minimisation.
This review concluded when
MEXT's ILC Advisory Panel released 
its report~\cite{AdvPanel} on July 4, 2018, summarising the
studies of the several working groups (WG) that
reviewed 
a broad range of aspects of the ILC.  The most recent studies include
a specific review of the scientific merit and the technical design for the ILC250. 
The  Physics WG scrutinised the scientific merit of the ILC250,
leading to their strong and positive statement on the importance of
the ILC250 to 
measure precisely the couplings of the Higgs boson \cite{AdvPanel}.
The TDR WG reviewed issues addressed in the Technical Design Report
and the ILC250 design, including the  cost estimate and technical feasibility.  
Other working groups of the MEXT review commented on manpower needs, 
organisational aspects, and the experience of previous large projects.
The report of the ILC Advisory Panel was followed by the beginning of
deliberations in a committee and technical working group 
established by the Science Council of Japan (SCJ),
the second stage of the review process.   The SCJ released its review
on December 19, 2018~\cite{SJCreport}.   The review concluded that ``the research topic
of precise measurement of Higgs couplings is extremely important'' but
expressed doubts about the cost of the project, which is well beyond
the costs of proposals brought forward by chemists and biologists.
The financing of the project will depend on negotiations with
international partners, led by the Japanese government after a clear
statement of interest.   The Japanese government is new preparing for
this step, 
a decision on hosting the collider, which can be followed by a move to the next phase of international
negotiations.  A new  independent committee (LDP Coordination Council
for the Realization of ILC),
led by high-ranking members of the Liberal Democratic 
Party, the majority party in the Diet,  has now 
convened to encourage the national government along this path.

Given a positive signal by the Japanese government, the ILC could move 
forward rapidly.
The potential timeline would have an initial state of 
about 4 years to obtain
 international agreements, prepare for the  construction, and form 
 the requisite international
collaboration.  The construction phase would then need 9 years.


It is an important aspect of the discussions of the ILC in Japan that the
ILC is seen as global project that will foster exchange between Japan
and other nations.   Thus, the  
scientific interest and political engagement of partner countries is of
major importance
for the Japanese authorities.  

The purpose of this report is to set out in detail the current status
of the ILC project.  We discuss 
the physics reach of the ILC, the technological maturity of the accelerator,
detector, and software/computing designs,
and the further steps 
 needed to concretely realise the project.  Section~\ref{sec:ilc}
 describes the accelerator design and technology, reviewing both 
 current status of SCRF development and the general layout of the
 machine.  This section also presents luminosity and energy upgrade
 options,  as well as civil engineering plans, including site
 specific details, and cost and schedule estimates.
 Section \ref{sec:runscenarios}  presents the current
 thinking about the operations of the ILC, with estimates of the plan
 and schedule for the 
 collection of integrated luminosity.
 Section \ref{sec:physics} gives an overview of the physics case for
 the ILC as a 250~GeV collider.
 This includes a more detailed discussion of the significance of the Higgs boson 
 as a tool for searching for physics beyond the SM, the qualitative
 comparison of the ILC to the LHC as a facility for precision Higgs
 studies, and the theoretical approach for extracting Higgs boson
 couplings from $\ee$ data.  This section also discusses the physics
 opportunities of searches for exotic Higgs decays and studies of
 other processes of interest including SM fermion pair-production and
 searches for
 new particle pair production. Section~\ref{sec:highenergy} described
 the additional opportunities that the energy extension to 500~GeV
 will make available. 
 
 Section \ref{sec:detectors}
 provides detailed descriptions of the ILC detector designs
 that have been developed by the community,
 through detector R\&D and prototyping, and used as detector
 models to show the simulated performance on the various
 physics channels. Section \ref{sec:software}
 summarises the computing needs of the ILC program,
 including software.  These two sections provide the basis for a
 discussion of the experimental measurements of reactions crucial to
 the ILC program. All of the projections of experimental
 uncertainties given in this paper are based on full-simulation
 studies using the model detectors described in
 Sec.~\ref{sec:detectors},
 with capabilities justified by
 extensive R\&D programs. 

Building on this discussion, 
Sec.~\ref{sec:higgs}
 gives a description of physics simulations involving Higgs
 boson  reactions.  Section~\ref{sec:ew} describes physics simulations
 carried out for the reactions $\ee\to W^+W^-$ and $\ee\to f\bar f$.
 Section~\ref{sec:top} discusses simulations of measurements of top
 quark properties at the energy-upgraded ILC.
 Section~\ref{sec:gigaz} described the possible program of the ILC at
 the $Z$ boson pole, to improve the results of LEP precision
 electroweak measurements. These studies lead to  concrete
 quantitative estimates for the expected uncertainties in Higgs boson
 coupling determinations.   Based on the results of these studies, we
present in Sec.~\ref{sec:global}  what we feel are conservative
estimates for the precision that the ILC will attain in a highly
model-independent analsys for the determination of the Higgs boson
width and absolutely normalized couplings.   We compare these
estimates to those presented in the CDRs for other  $\ee$ Higgs
factories in those expected from the high-luminosity phase of the LHC.

Section~\ref{sec:searches} describes the capability of the ILC for
direct searches for pair-production of new particles, covering  a number of scenarios
that are difficult for the LHC but which can be investigated in detail
at $\ee$ colliders.  Section~\ref{sec:conclusion} gives our conclusions.



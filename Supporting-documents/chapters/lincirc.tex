

In this section, we will present a comparison of the capabilities of the ILC for precision Higgs measurement with those of other proposed linear and circular colliders, including CLIC, CEPC, and FCC-ee.  We will present two sets of comparisons, organized in the following way.

FIrst, each collider proposal has presented its own set of projections in its documentation for the European strategy study.   We have copied the relevant numbers for projected Higgs boson coupling uncertainties into Tab.~\ref{tab:askthem}.  Typically, these estimates are made using the more model-dependent kappa fit.

It is interesting to ask how the proposals would compare if a common fitting technique is used. In almost all cases, the measurement errors are dominated by statistics and the efficiencies used in the analyses are similar.  So a simple way to make the comparison would be to use the results of ILC analyses to estimate efficiencies and statistical errors for all of the colliders, assume the luminosity samples in the collider proposal, assume the same measurement errors per unit of luminosity that we assumed in generating Tab.~\ref{tab:ILCHiggs},  take account of differences in the cross sections resulting from the use (or not) of polarized beams, and rerun our fitting  program for those conditions.   This is the method used to generate Tab.~3 of Ref.~\cite{Barklow:2017suo}.   We assume, for CEPC, a sample of  5~\iab at 250~GeV without polarization, and, for FCC-ee, a sample of 5-\iab at 250~GeV plus 1.5~\iab at 350~GeV,
without polarization.  The run plan for CLIC includes only 1~\iab at 380~GeV before the energy upgrade to 1~TeV.  Since we are unconfortable using the EFT formalism with dimension-6 operators only at 1~TeV and above, we represent CLIC by a sample of
2~\iab, similar to ILC, with $e^-$ polarization only, at 350~GeV.  The results are presented in Tab.~\ref{tab:oursimple}. 

Finally, we introduce effects that add further realism to the comparison.  For CEPC and FCC-ee, we include the improvement in precision electroweak measurements expected from the program of $Z$ pole running in the their proposals.  For colliders without beam polarization, we increase the size of the sysematic errors according to the discussion in  Sec.~\ref{subsec:polarization}. The results are presented in 
Tab.~\ref{tab:ournotsosimple}. 


[discussion of the results]
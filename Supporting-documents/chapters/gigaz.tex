

Electroweak precision observables (EWPO) measured at LEP and SLD at the Z-pole today still provide the backbone of the interpretation of measurements in the electroweak sector. An comprehensive overview is given in~\ref{lepsld05}.    
It is possible to run the ILC at the $Z$ boson pole at 91~GeV.   This
program, at an instantaneous luminosity of  $0.7\cdot10^{34}$ , would provide  about $0.2\,{\rm ab^{-1}}$ in a 1-year run equivalent to about 200 times more than what has been collected at LEP by all four experiments, improving thus in first approach the precision of measurements of electroweak observables by one order of magnitude.  In this section, we describe this program and its projected results. 

A central precision observable is the effective weak mixing angle $\sin^2\theta^{\ell}_{eff.}$. Note that the most precise measurements at LEP and SLD of the effective weak mixing angle differ by about 3 standard deviations. At a linear collider with polarised beams the effective weak mixing can be extracted in several ways but notably by measuring the left right asymmetry $A_{LR}$. In this case we can make full use of the hadronic decays of the $Z$ with only little sensitivity to experimental effects such as detector acceptance or beam energy measurements. The branching ratio to hadrons is about 20 times higher than that to e.g. muons. Therefore the larger versatility of the linear collider compensates largely for the lower luminosity available at linear colliders compared with circular colliders. 
For individual fermion asymmetries it is instructive to recall how polarisation increases the analysis power of an asymmetry measurement. It is \begin{equation}
A^{0,f}_{FB} = \frac{3}{4}\frac{({\cal A}_e - {\cal P}_e){\cal A}_f}{(1-{\cal P}_e{\cal A}_e)} =  \frac{3}{4} A_P {\cal A}_f  
\end{equation} 
with $A_P$ being the analysis power. It is $A_P\approx 0.85$ for ${\cal P}_e = -0.8$, $A_P\approx 0.74$ for ${\cal P}_e = 0.8$ and $A_P = {\cal A}_e \approx 0.15$ for ${\cal P}_e = 0$.
It follows that already the measurement for individual beam polarisations lead to an increase in sensitivity between 5 and 6.5 compared to the non-polarised case.    
It is finally 
\begin{equation}
A^{0,f}_{FB,LR} = \frac{3}{4} {\cal P}_e {\cal A}_f
\label{eq:afblr}
\end{equation}
The different approaches will allow us to either combine the measurements into one single quantity as in Eq.~\ref{eq:afblr} or look with already high precision at the individual results. The individual results may have different systematics as e.g. migrations in the polar angle spectra are less important in case of right handed electron beams. Note also that the measurement of $A_{LR}$ will allow to measure ${\cal A}_e$ with high precision. Therefore no model dependent assumptions as lepton universality will have to be made needed to extract e.g. the effective weak mixing angle from $e^+e^- \rightarrow \mu^+\mu^-$ with unpolarised beams. Note also that at LEP the most precise value of $\sin^2\theta^{\ell}_{eff.}$ from $A^{0,b}_{FB}$ has been extracted by taking the Standard Model value of $\sin^2\theta^{b}_{eff.}$   
Following the argumentation before, the analysis power given by polarisation will increase the effective luminosity by about a factor of 30 w.r.t. the non-polarised case. Statistical errors can therefore be neglected. This puts the systematic errors into the focus. The analysis power given by polarisation is crucial to remedy a drawback of linear colliders w.r.t. circular colliders on what concerns the imperfect knowledge of the beam energy. On the other hand the linear collider requires the control of the beam polarisation. A running at Z Pole may require the introduction of a conventional $e^+$ source, which would exclude the application of the Blondel scheme and require a precise calibration of the polarimeters. The ILC will be run at higher energies at which the process $e^+e^- \rightarrow WW$ will be available. This process will allow for calibrating the polarimeters to a high precision such that a precision of the knowledge of the polarisation of 0.1\% seems to be well in reach. The control of the other systematical errors that are discussed in the LEPSLD-REPORT will benefit from the progress on detectors and theory and can thus be neglected in first order compared with the polarisation. HOWEVER, CHECK INFLUENCE OF BEAM ENERGY. Tab.~\ref{tab:gigazresults} lists the precision of observables as relevant for the Higgs fit. 

%%%%%%%%%%%%%%%%%%%%%%%%%%%%%%%%%%%
%\begin{table*}[tb]
%\begin{footnotesize}
%  \begin{center}
%    \begin{tabular}{|c|c|c|c|c|c|c|c|c|c|}
%      \hline
%      Quantity & $\fonevA$ & $\fonevZ$  & $ \foneaZ$  & $\ftwovA$ & $\ftwovZ$ & %$\glA$ &   $\grA$ & $\glZ$  & $\grZ$ \\ 
%      \hline
%      \hline
%      %$-100\%,+100\%$  &0.317 & 0.308 &1.7 \\ \cline
%       SM Value at tree level & 2/3 &  0.230 & -0.595 & 0 & 0 & 2/3 & 2/3  & %0.824 & -0.364 \\ 
%      \hline
%       Standard deviation      & 0.002 &  0.003 & 0.007 & 0.001 & 0.002  & %0.005 & 0.005 & 0.008 & 0.009 \\ 
%      \hline
%      Relative precision [\%] & 0.3 &  0.9 & 1.2  & - & - & 0.8 & 0.8 & 1.0 & %2.5 \\ 
%      \hline
%      %$+100\%,-100\%$& 0.433 & 0.442 & 1.9\\ \hline
%      %$+0.8,-0.3$& 0.63 & 1.3\\ \hline
%    \end{tabular}
%  \end{center}
%  \caption{\sl Standard deviations and resulting relative precision of form %factors and couplings derived from the statistical precision on the %observables' cross section and $\afbt$ as listed in Table~\ref{tab:resafb}.}
%  \label{tab:gigazresults}
%\end{footnotesize}
%\end{table*}
\subsection{GigaZ and $e^+e^- \rightarrow b\bar{b}$}
The GigaZ running develops its full power when combined with measurements above the the Z-pole. This can be impressively demonstrated with the process $e^+e^- \rightarrow b\bar{b}$. There are no propagator or interference effects at the Z Pole. Potential deviations from the Standard Model can be thus fully attributed to a mixing of the $Z$ with vector bosons of new physics.   
%%%%%%%%%%%%%%%%%%%%%%%%%%%


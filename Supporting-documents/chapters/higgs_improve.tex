







As elaborated in the previous section, all the estimates for Higgs measurements shown in
~\ref{tab:higgserrors} are based on available full simulation studies, which are performed using the analysis
techniques known at present. These estimates are clearly too
conservative in the sense that we have not been able to exploit all
useful signal channels.  In addition, analysts working closely with
the data are always able to invent algorithms to that more cleverly
optimize the data that is actually collected.   The projected
uncertainties  quoted here and
here and in Sec.~\ref{subsec:global:elements} do not take advantage of
these  likely improvements.

Since the formal estimates of the performance of HL-LHC given in the
HL-LHC Yellow Report~\cite{Cepeda:2019klc} do take into account
improvements in systematic uncertainties that are anticipated but not
yet realised, it seems to us reasonable to define also for ILC an
optimistic scenario with improved performance.   We use this scenario
to compare to the results of Ref.~\cite{Cepeda:2019klc} in the manner
explained in Sec.~\ref{subsec:higgs:ilclhc}.   This scenario, which we 
refer to as ``S2'' in that discussion, includes the following
improvements in the analysis just described.  In all cases, these
improvements are under study by our group using our full simulation
tools
and are  suggested, if not yet validated, by our current results: 
\begin{itemize}
\item 10\% improvement in signal efficiency of the  jet clustering algorithm.
\item 20\% improvement in the performance of the  flavor tagging algorithm. 
\item 20\% improvement in statistics by including more signal channels
  in  $\sigma_{Zh}\cdot BR(h\to WW*)$. 
\item a factor of 10 improvement in the precision electroweak input
  $A_{\ell}$ thorugh the measurment of $e^+e^-\to\gamma Z$ with
  polarized beams at 250~GeV.
\end{itemize}
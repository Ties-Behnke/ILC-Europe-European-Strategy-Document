% CHAPTER ON ILC MACHINE

{\it PLEASE DO NOT EDIT THIS PART IN OVERLEAF! This is work in progress! }


Includes general design principle (low power consumption) and some history.

% CHAPTER ON ILC MACHINE

{\it Status: Dec 31, 2018}

{\it Some introduction here- 1 page. 

Includes general design principle (low power consumption) and some history.


}

%===============================================================================
 \begin{figure*}[tb]
 %\epsfysize=9.0cm
 \begin{center}
 \includegraphics[width=\hsize]{chapters/figures/TDR-machine-layout-cartoon-staged-mirror.pdf}
\caption{Schematic layout of the ILC in the $250\,{\mathrm{GeV}}$ staged configuration.
\label{fig_ilc-schematic}}
 \end{center}
 \end{figure*}

\subsection{Design Evolution since the TDR}
\label{sec:design_evo}

{\it Describe how design has evolved since TDR:

Staging to 250GeV,
longer undulator,
Shield wall reduction,
common L*,
new lumi parameters for 250GeV,
mention longer tunnel at 500GeV
}


\subsection{Superconducting RF Technology}


%{\it Description of the TESLA SCRF technology - 4 pages
%
%Stresses long experience - FLASH, STF, XFEL, broad industrial base - LCLS-II, SHINE:
%addresses concerns mentioned in Nomura report and others (yield / gradient, MARX modulator)
%
%Nomura issue: MARX modulator
%
%Figures: Cavity, Cryomodule (Rey Hory) 
%
%}


The heart of the ILC accelerator are the two superconducting Main Linacs that accelerate both beams from \num{5} to \siunit{125}{GeV}.
These linacs are based on the TELAS technology:
beams are accelerated in \siunit{1.3}{GHz} nine-cell superconducting cavities made of Niobium and operated at \siunit{2}{K} (Fig.~\ref{fig:tesla-cavity}), 
which are assembled into cryo modules comprising eight to nine cavities, an optional quadrupole / corrector / beam position monitor unit, and all necessary cryogenic supply lines (Fig.~\ref{fig:crymodule}). 
Pulsed klystrons supply the necessary radio frequency power (High-Level RF HLRF) to the cavities by means of a waveguid power distribution system and input couplers, one per cavity.

This technology was primarily developed at DESY for the TESLA accelerator project that was proposed in 2001.
Since then, the TESLA technology collaboration~\cite{bib:ttc} has been improving this technology, which is now being used in several accelerators in operation (FLASH at DESY~\cite{bib:flash}, European XFEL in Hamburg~\cite{bib:xfel}), under construction (LCLS-II at SLAC, Stanford, CA~\cite{bib:lcls-ii}) or planned (SHINE in Shanghai~\cite{Zhao:2018lcl}).

\subsubsection{The Quest for High Gradients}

The single most important parameter for the cost and performance of the ILC is the accelerating gradient $g$.
The TDR baseline value is an average gradient $g = \siunit{31.5}{MV/m}$ for beam operation, with a $\pm 20\,\%$ gradient spread between individual cavities.
Recent progress in R\&D for high gradient cavities raises the hope to raise the gradient by  $10\,\%$  to  $g = \siunit{35}{MV/m}$, which would reduce the total cost of the \siunit{250}{GeV} accelerator by about  $6\,\%$.
%% 6% = 50BY / 803BY
% Copied from TDR Vol 3.II p. 23
To achieve the desired gradient in beam operation, the gradient achieved in the low-power vertical test (mass production acceptance test) is specified $10\,\%$ higher to allow for operational gradient overhead for low-level
RF (LLRF) controls, as well as some degradation during cryomodule installation (few ${\mathrm{MV/m}}$).

\paragraph{Gradient impact on costs}
To the extent that the cost of cavities, cryomodules and tunnel infrastructure is independent of the achievable gradient, the investment cost per GeV of beam energy is inversely proportional to the average gradient achieved, which is the reason for the enormous cost saving potential from higher gradients.
This effect is partially offset for two reasons: both, the energy stored in the electromagnetic field of the cavity, and the dynamic heat load to the cavity from the electromagnetic field, grow quadratically with the gradient, and thus linearly for a given beam energy.
The electromagnetic energy stored in the cavity must be replenished by the RF source during the filling time that precedes the time when the RF is used to accelerate the beam passing through the cavity; this energy is lost after each pulse and thus reduces the overall efficiency and requires more or more powerful modulators and klystrons.
The overall cryogenic load is dominated by the dynamic heat load from the cavities, and thus operation at higher gradient requires larger cryogenic capacity.
Cost models that parametrise these effects indicate that the minimum of the investment cost per GeV beam energy lies at \num{50} or more GeV, depending on the relative costs of tunnel, SCRF infrastructure and cryo plants, and depending on the achievable $Q_0$. 
{\it CHECK THIS; reference?}
At any rate, the optimal gradient is significantly higher than the value of approximately \siunit{35}{MV/m} that is currently realistic, underpinning the relevance of achieving higher gradients.

However, it should be noted that in contrast to the initial invest, the operating costs invariably rise when the gradient is increased.

\paragraph{Gradient limitations}

Fundamentaly, the achievable gradient of a SC cavity is limited by the point where the magnetic field at the cavity walls surpasses the critical field $H\sub{crit,RF}$ of the superconductor, which 
This gradient depends on the material, operating temperature, and the cavity geometry. 
For the TESLA type cavities employed at the ILC, this limit is about \siunit{48}{MV/m} at \siunit{2}{K},
the best XFEL production cavity reached \siunit{44.6}{MV/m} (XXX CHECK XXX) (Fig.~\ref{fig:cavity-gradient}).
The record for single cell cavities operating at \siunit{1.3}{GHz} is \siunit{59}{MV/m}~\cite{Eremeev:2007zza}.
%
% Explanation:  
% 48MV/m using  $B\sub{pk}/g = \SIunit{4.26}{mT / (MV/m)}$ for ILC cavities~\cite{Adolphsen:2013kya}
% and the critical field for niobium at \SIunit{2}{K} has been measured to be \SIunit{205}{mT}~\cite{Padamsee:2009,Hays:1995zd}.
% , not far from the theoretical prediction of \SIunit{230}{mT}~\cite{Geng:2006wf}
% Padamsee:1998vf page 101 quotes 47MV/m for 4.2mT/MV/m
% Record 59MV/m corresponds to 206.5mT

As niobium is a type-II superconductor, it has two distinct superconducting phases, the Meissner state, with complete magnetic flux expulsion, which exists up to a field strength $H\sub{c1} \approx \siunit{180}{mT}/\mu_0$ ($\mu_0 = 4\pi \siunit{10^{-7}}{T\,m/A}$ being the vacuum permeability), and a mixed state in which flux vortices penetrate the material, up to a higher field strength $H\sub{c1}$, at which superconductivity breaks down completely.
In RF fields, the penetrating vortices move due to the changing fields and thus dissipate energy, causing a thermal breakdown; 
however, for RF fields the Meissner state may persis metastably up to the superheating field strength $H\sub{sh} \approx \siunit{240}{mT}/\mu_0$, which is expected to be the critical RF field critical field $H\sub{crit,RF}$~\cite{Padamsee:1998vf}.
Experimentally, niobium RF cavities have been operated at field strengths as high as $H=\siunit{206}{mT}/\mu_0$~\cite{Eremeev:2007zza}, the best XFEL production cavities reach about $\siunit{190}{mT}$.
Recently, even \siunit{210}{mT} have been achieved at FNAL~\cite{Grassellino:2018tqg}.
In recent years, theoretical understanding of the nature of this metastable state and the mechanisms at the surface that prevent flux penetation has significantly improved~\cite{Gurevich:2017vnn,Kubo:2017cww}.
It appears that a thin layer of ``dirty'' niobium, e.g. with interstitial impurities, on top of a clean bulk with good thermal conductivity, is favourable for high field operation.  

The gradient at which a SC cavity can be operated in practice is limited by three factors~\cite{Padamsee:1998vf}:
\begin{itemize}
\item the thermal breakdown of superconductivity, when local power dissipation causes a local quench of the superconductor,
\item the decrease of the quality factor $Q_0$ at high gradients that leads to increased power dissipation,
\item and the onset of field emission that causes the breakdown of the field in the cavity.
\end{itemize}
The onset of these adverse effects is mostly caused by micro-metre sized surface defects of various kinds. 
Producing a sufficiently defect-free surface in an economic way is thus the central challenge in cavity production.

More than $20$ years of industrial production of TESLA type cavities have resulted in a good understanding which production steps steps and quality controls are necessary to produce cavities with high-quality, nearly defect-free surfaces that are capable of achieving the desired high field strengths at a reasonable production yiels.

\paragraph{Results from XFEL cavity production}

The production and testing of $831$ cavities for the European XFEL~\cite{Singer:2016fbf,Reschke:2017gjp} provides the biggest sample of cavity production data so far. 
Cavities were acquired from two different vendors RI and EZ,
of which vendor RI employed a production process with a final surface treatment closely following the ILC specifications, including a final electropolishing (EP) step,
while the second vendor EZ used buffered chemical polishing (BCP).
The XFEL specifications asked for a usable gradient of \siunit{23.6}{MV/m} with a $Q_0 \ge 1  \cdot 10^{10}$ for operation in the cryomodule;
with a $10\,\%$ margin this corresponds to a target value of \siunit{26}{MV/m} for the performance in the vertical test stand for single cavities.
Fig.~\ref{fig:cavity-gradient} shows the $Q_0$ data versus accelerating gradient of the best cavities received, with several cavities reaching more than \siunit{40}{MV/m}, significantly beyond the ILC goal, already with $Q_0$ values that approach the target value $1.6\cdot10^{10}$ that is the goal of future high-gradient R\&D.

\begin{figure}[htbp]
   \includegraphics[width=\hsize]{chapters/figures/prab-19-092001-fig19-mod}
\caption{Examples of the $Q_0\,(E\sub{acc})$ curves of some of the best
cavities, either treated at RI using ``EP final'', or at EZ using
``BCP flash.''
\cite[Fig. 19]{Singer:2016fbf}. 
Vendor ``RI'' employs a production process that closely follows the ILC specifications, with a final electropolishing step.
The ILC gradient / $Q_0$ goals are overlaid.}
% Paper published under CC-BY-3.0
\label{fig:cavity-gradient}
\end{figure}

XFEL production data, in particular from vendor RI, provide excellent statistics for the cavity performance as received from the vendors, as shown in Fig.~\ref{fig:cavity-yield}.
For vendor ``RI'', the yield for cavities above \siunit{28}{MV/m} is $85\,\%$ of for the maximum gradient, with an average of \siunit{35.2}{MV/m}.

\begin{figure}[htbp]
   \includegraphics[width=\hsize]{chapters/figures/prab-20-042004-fig33}
\caption{Distribution and yield of the ``as received'' maximum
gradient of cavities produced for the European XFEL, separated by vendor \cite[Fig. 33]{Reschke:2017gjp}. 
Vendor ``RI'' employs a production process that closely follows the ILC specifications, with a final electro polishing step.}
% Paper published under CC-BY-4.0
\label{fig:cavity-yield}
\end{figure}

Since the XFEL performance goal was substantially lower than the ILC specifications, cavities with gradient below \siunit{28}{MV/m}, which would not meet ILC specifications, were not generally re-treated for higher gradients, limiting our knowledge of the effectiveness of re-treatment for large gradients.
Still, with some extrapolation it is possible to extract yield numbers applicable to the ILC specifications ~\cite{bib:Walker:2017.lcws}.

The XFEL data indicate that after re-treating cavities with gradients outside the ILC specifications of $\siunit{35}{MV/m} \pm 20\,\%$, i.e. below \siunit{28}{MV/m}, a yield of $94\,\%$ for a maximum gradient above \siunit{28}{MV/m} can be achieved, wit an average value of \siunit{35}{MV/m}, meeting the ILC specifications.
Taking into account limitations from $Q_0$ and the onset of field emission, the usable gradient, however, is lower, with a $82\,(91)\,\%$ yield and an average usable gradient of \siunit{33.4}{MV/m} after up to one (two) re-treatments.
The re-treatment and testing rate is significantly higher than assumed in the TDR, but the XFEL experience shows that re-treatment can mostly be limited to a simple high-pressure rinse (HPR) rather than an expensive EP step.  

Overall, the XFEL cavity production data prove that it is possible to mass-produce cavities meeting the ILC specifications as laid out in the TDR with the required performance and yield.


\paragraph{High-gradient R\&D: nitrogen infusion}

In recent years, new techniques have emerged that seem to indicate that higher fields combined with higher quality factors are attainable in bulk niobium cavities.

In the early 2010s, nitrogen doping was developed as a method to substantially increase $Q_0$ by adding nitrogen during the \siunit{800}{^\circ C} baking, which leads to interstitial nitrogen close to the niobium surface.
This technique has been employed successfully in the production of the cavities for LCLS-II~\cite{XXX}.
However, nitrogen doping reduces the critical RF field of the material and thus limits the achievable gradients to values below \siunit{30}{MV/m}, rendering doped material useless for high gradient applications.

By contrast, in nitrogen infusion the nitrogen is added during the low temperature baking at \siunit{120}{^\circ C}.
Experimental results seem to indicate that Nitrogen infusion may offer a combination of three advantages:
\begin{itemize}
\item Reaching higher accelerating gradients,
\item at higher $Q_0$ values, resulting in a reduced cryogenic load, 
\item with a simplified and less expensive production process that does away with the final electropolishing step.
\end{itemize}

Fig.~\ref{Grassellino:2017bod} \cite[Fig. 5]{Grassellino:2017bod} shows how the addition of nitrogen during the final \siunit{48}{h} long \siunit{120}{^\circ C} bake of a one--cell cavity drastically improves the cavity quality factor as well as the maximum gradient, which comes close to the best XFEL cavity results, but at higher $Q_0$.

% Add nine-cell result


XXX CONTINUE HERE XXX experimental status, success, transfer to other labs

% The role of the 120°C bake in in the formation of a dirty surface layer was elucidated in a study measuring the B field penetration depth with muon spin rotation (muSR) \cite{Romanenko:2013saa}. 

% JLAB result: \cite{Dhakal:2017xxq},\cite{Konomi:2018vni}
% Cornell result: \cite{Koufalis:2017blg}
% KEK: mixed results {bib:Umemori:2018.lcws}

% New FNAL result: 49MV/m: {Grassellino:2018tqg}

\begin{figure}[htbp]
   \includegraphics[width=\hsize]{chapters/figures/sst-30-094004-fig5}
\caption{Effect of successive cavity treatments on a single cavity: $\siunit{800}{^\circ C}$ bake for five days (black, lowest curve), followed by $48$ hours baking at $\siunit{120}{^\circ C}$ (red, middle curve). 
A third heat treatment including nitrogen infusion (green, top curve) significantly  raises the breakdown gradient and the quality factor of the cavity
\cite[Fig. 5]{Grassellino:2017bod}.
}
% Paper is published under CC-BY-3.0
\label{fig:n2infusion}
\end{figure}

\paragraph{High-gradient R\&D: alternative cavity shapes}

Fundamentally, the achievable gradient in a niobium cavity is limited by the maximum magnetic field at the cavity surface, not the electrical field strengths.
The ratio between peak surface field $B\sub{pk}$ and gradient $g$ depends on the cavity geometry and is $B\sub{pk}/g = \siunit{4.26}{mT / (MV/m)}$ for TESLA type cavities.
A number of alternative cavity shapes have been investigated with lower ratios~\cite{Geng:2006wf},
resulting in single cells gradients up to \siunit{59}{MV/m}~\cite{Eremeev:2007zza}.
The reduced magnetic field, however, has to be balanced with other factors that favour the TESLA cavity shape, namely: a reasonable peak electrical field to limit the risk of field emission, sufficient iris width and cell-to--cell RF coupling, and a mechanical shape that can be efficiently fabricated.

Recently, new five-cell cavities with a new ``low surface field'' (LSF) shape~\cite{Li:2008a} have been produced at JLAB and have achieved gradient of up to~\siunit{50}{MV/m} in three of the five cells, which is a new record for multi-cell cavities~\cite{bib:Geng:2018.lcws}. 
The LSF shape aims to constitutes a good compromise between the goal of a low magnetic field and the other criteria, and demonstrates that further improvements in gradient may be achieved in the future.


\subsubsection{Further Cost Reduction R\&D}

\paragraph{Low $RRR$ material}

The niobium raw material and preparation of sheets  are a significant cost driver; R\&D is underway to re-evaluate the stringent limits on impurities, especially of tantalum, and the demand for a high residual resistivity ratio $RRR > 300$~\footnote{$RRR$ is the ratio of the material's room temperature resistivity to the normal conducting resistivity close to \siunit{0}{K}; heat conductivity from electrons is proportial to $RRR$. $RRR$ is reduced by impurities, in particular interstitial ones  from hydrogen, nitrogen and oxygen.}, to reduce the raw material cost. 
Conductivity for electricity and heat transport by electrons are proportional,
thus large $RRR$ values, indicative of low impurity content, make the cavities less susceptible to thermal breakdown from surface defects~\cite{bib:Kubo:2018.ttc}.

\paragraph{Ingot and large-grain niobium}

Together with direct slicing of discs from large niobium ingots, without rolling, forging and grinding or polishing steps, the cost for niobium sheets has the potential to be reduced by $50\,\%$~\cite{Evans:2017rvt,Kneisel:2014uqa}.

% Test with ingot niobium promising: \cite{Dhakal:2015xac}


\subsubsection{Basic Parameters}

% Copied from TDR Vol 3.II p. 23
The choice of operating frequency is a balance between the higher cost of larger, lower-frequency cavities and the increased cost at higher frequency associated with the lower sustainable gradient from the increased surface resistivity. 
The optimum frequency is in the region of \siunit{1.5}{GHz}, but during the early R\&D on the technology, \siunit{1.3}{GHz} was chosen due to the commercial availability of high-power klystrons at that frequency.

\subsubsection{Cavities}

The superconducting accelerating cavities for the ILC are nine-cell structures made out of high-purity niobium (Fig.~\ref{fig:tesla-cavity}), with an overall length of \siunit{1.25}{m}.
Cavity production starts from niobium ingots which are forged and rolled into \siunit{2.8}{mm} thick niobium sheets that are individually checked for defects by an eddy current scan and optical inspection~\cite{Adolphsen:2013jya}.
Cavity cells are produced by deep-drawing the sheets into half cells, \num{18} of which are joined by electron beam welding with two end groups to form the whole structure.
The welding process is one of the critical and cost-intensive steps of the cavity manufacturing procedure. 
Utmost care must be taken to avoid irregularities, impurities and inclusions in the weld itself, and deposition of molten material at the inner surface of the cavity that can lead to field emission.

\begin{figure}[htbp]
   \includegraphics[width=\hsize]{chapters/figures/tesla9cell-cavity-2}
\caption{A $1.3\,{\mathrm{GHz}}$ superconducting niobium nine-cell cavity.
}
% Figure from TDR Executive Summary
% https://svnsrv.desy.de/k5websvn/wsvn/General.ilctdr/tags/FormOne-Print-Release/tdres/accel/figs/tesla9cell-cavity-2.jpg
\label{fig:tesla-cavity}
\end{figure}

After welding, the inner surface of the cavity must be prepared.
The process is designed to remove material damage incurred by chemical procedures during the fabrication process, remove chemical residues from earlier production steps, remove hydrogen in the bulk niobium from earlier chemical processing, and remove and particulate contamination.
in a last step, the cavity is closed to form a hermetically sealed structure ready for transport.
The treatment steps involve a series of rinses with ethanol or high pressure water, annealing in a high purity vacuum furnace at \siunit{800^\circ}{C} and \siunit{120^\circ}{C}, and electro polishing or buffered chemical polishing.
The recipe for the surface preparation has been developed over a long time and still subject to optimisation, as it is a major cost driver for the cavity production and largely determines the overall performance and yield of the cavities.
In particular the electro polishing steps are complicated and costly, as they require complex infrastructure and highly toxic chemicals.

{\it 
ADD SOMETHING ABOUT QA AND RETREATMENT HERE.

EXPLAIN INTEGRATION WITH MAGNETIC SHIELD AND TUNER}
 


\subsubsection{Power Coupler}

The power coupler transfers the radio frequency (RF) power from the waveguide system to the cavity. 
At the ILC, couplers with a variable coupling are employed, which is realised by a movable antenna.
At the same time, it separates the cavity vacuum from the atmospheric pressure in the waveguide, and insulates the cavity at \siunit{2}{K} from the surrounding room temperature.
In summary, the coupler has to fulfil a number of demanding requirements: transmission of high RF power with minimal losses and no sparking, vacuum tightness and robustness against window breaking, minimal heat conductivity.  
As a consequence, the coupler design is highly complex, with a large number of components and several critical high-tech manufacturing steps.

The baseline coupler design was originally developed in the 1990s for the Tesla Test Facility (TTF, now FLASH) at DESY,
and has since been modified by a collaboration of LAL and DESY for use in the European XFEL.
$800$ of these couplers (depicted in Fig. \ref{fig:xfelcoupler}) were fabricated for the European XFEL~\cite{Kaabi:2013wna} by two different companies and are now in operation.
A lot of experience has been gained from this production~\cite{Sierra:2017wyc}.



\begin{figure}[htbp]
   \includegraphics[width=\hsize]{chapters/figures/xfelcoupler}
\caption{An XFEL type coupler.
  % Fig from ILC TDR, should be OK to use
}
\label{fig:xfelcoupler}
\end{figure}


\subsubsection{Cryomodules}

\begin{figure}[htbp]
   \includegraphics[width=\hsize]{chapters/figures/10_ILC_cryomodule}
\caption{An ILC type cryomodule. \copyright Rey.Hori/KEK.}
% Figure from Rey.Hori - check copyright!
% http://www.linearcollider.org/images/pid/1000890/gallery/10_ILC_cryomodule.jpg
\label{fig:crymodule}
\end{figure}


\begin{figure}[htbp]
   \includegraphics[width=\hsize]{chapters/figures/srf17-mopb106-fig1}
\caption{Average of the operating (blue) and maximum
(green) gradient for cavities in each European XFEL serial-production cryomodule.
The specification of \siunit{23.6}{MV/m} is marked by a red line
\cite{Kasprzak:2018kkr}.
}
% Paper is published under CC-BY-3.0
\label{fig:cryomodules-performance}
\end{figure}


Record at Fermilab: \cite{Broemmelsiek:2018iqr}
record at STF2: \cite{Yamamoto:2018kml}

\subsubsection{Plug-compatible design}

In order to allow various designs of sub-components from different countries and vendors to work together in the same cryomodule, a set of interface definitions has been internationally agreed upon.
This ``plug-compatible'' design ensures that components are interchangeable between modules from different regions and thus reduces the cost risk.
Corresponding interface definitions exist for the cavity, the fundamental-mode power coupler, the mechanical tuner and the helium tank.
The ``S1Global'' project~\cite{bib:s1g} has successfully built a single cryomodule from several cavities equipped with different couplers and tuners, demonstrating the viability of this concept.


\subsubsection{High-Level Radio-frequency}

The high-level radio-frequency (HLRF) system provides the RF power that drives the accelerating cavities.
The system comprises modulators, pulsed klystrons, and a waveguide power distribution system.


\paragraph{Modulators}
The modulators provide the short, high-power electrical pulses required by the pulsed klystrons from a continuous supply of electricity. 
The ILC design foresees the use of novel, solid state Marx modulators.
These modulators are based on a solid-state switched capacitor network, where capacitors are charged in parallel over the long time between pulses, and discharged in series during the short pulse duration,
transforming continuous low-current, low voltage electricity into short high-power pulses of the required high voltage of \siunit{120}{kV} at a current of \siunit{140}{A}, over \siunit{1.65}{ms}.
Such Marx modulators have been developed at SLAC~\cite{Kemp:2011zz} successfully tested at KEK~\cite{Gaudreau:2014pza}.
However, long-term data about the required large mean time between failures (MTFB) are not yet available.

\paragraph{Klystrons}
The RF power to drive the accelerating cavities is provided by \siunit{10}{MW} L-band multi-beam klystrons. 
Devices meeting the ILC specifications were initially developed for the TESLA project, and later for the European XFEL.
They are now commercially available from two vendors (Thales and Toshiba), both of which provided klystrons for the E-XFEL.
The ILC specifications ask for an $65\,\%$ efficiency (drive beam to output RF power), which are met by the existing devices.

Recently, the High Efficiency International Klystron Activity (HEIKA) collaboration~\cite{Syratchev:2015a, Gerigk:2018ebm} has been formed that investigates novel techniques for high--efficiency klystrons.
Taking advantage of modern beam dynamic tools, methods such as the Bunching, Alignment and Collecting (BAC) method~\cite{Guzilov:2014a} and the Core Oscillation Method (COM)~\cite{Constable:2017hha} (Fig.~\ref{fig:com})
 have been developed that promise increased efficiencies up to $90\,\%$~\cite{Baikov:2015bif}.  
One advantage of these methods is that it is possible to increase the efficiency of existing klystrons by equipping them with a new electron optics, as was demonstrated retrofitting an existing tube from VDBT, Moscow. 
This increased the output power by almost 50\,\% and its efficiency from 42\,\% to 66\,\%~\cite{Jensen:2016a}.

To operate the ILC at an increased gradient of \siunit{35}{MV/m} would require that the maximum klystron output power is increased from $10$ to \siunit{11}{MW}. 
It is assumed that this will be possible by applying the results from the R\&D effort into high--efficiency klystrons.
 
\begin{figure}[htbp]
   \includegraphics[width=\hsize]{chapters/figures/eefact16-wet3ah2-fig1}
\caption{Electron phase  profile of an \siunit{800}{MHz} klystron employing the Core Oscillation Method (COM)~\cite{Constable:2017hha}.
}
% Paper is published under CC-BY-3.0
\label{fig:com}
\end{figure}

\paragraph{Local Power--Distribution System (LPDS)}

In the baseline design, a single RF station with one modulator and klystron supplies RF to $39$ cavities, which corresponds to $4.5$ cryo--modules~\cite[Sec. 3.6.4]{Adolphsen:2013kya}, so $2$ klystrons drive a $9$ cryo--module cryo-string unit.
The power is distributed by the LPDS, a system of waveguides, power dividers and loads. 
All cavities from a $9$-cavity module and half of a $8$--cavity module are connected in one LPDS, and three such LPDS units are connected to one klystron.
This arrangement allows an easy refurbishment such that a third klystron can be added to a cryo-string, increasing the available power per cavity by $50\,\%$ for a luminosity upgrade (cf.\ Sec.~\ref{subsec:upg-opt}).

The LPDS design must provide a cost--effective solution to to distribute the RF power with minimal losses, and at the same time provide the flexibility to adjust the power delivered to each cavity by at least $\pm20\,\%$ to allow for the specified spread in maximum gradient. 
The LPDS design therefore contains remotely controlled, motor-driven Variable Power Dividers (VPD), phase shifters, and H--hybrids that can distribute the power with the required flexibility.
This design allows to optimise the power distribution during operation, based on the cavity performance in the installed cryo--module, and thus to get the optimum performance out of the system.
It does not require a measurement of the individual cavity gradients after the module assembly, and is thus compatible with the ILC production scheme, where only part of the cryo--modules are tested.
This is a notable difference to the scheme employed at the European XFEL, where $100\,\%$ of the modules were tested, and the the power distribution for each module was tailored to the measured cavity gradients, saving investment costs for the LPDS but making the system less flexible.

\subsubsection{Cryogenics}

The operation of the large number of superconducting cryo--modules for the main linacs and the booster linacs of the sources requires a large--scale supply of liquid helium.
The cyo--modules operate at \siunit{2}{K} and are cooled with superfluid helium, which at \siunit{2}{K} has a vapour pressure of about \siunit{32}{mbar}.


The accelerator supplied with liquid helium by several cryogenic plants~\cite[Sec. 3.5]{Adolphsen:2013kya} of a size similar to those in operation at CERN for the LHC, at Fermilab, and DESY,
with a cooling capacity equivalent to about \siunit{19}{kW} at \siunit{4.5}{K}.
The \siunit{2}{K} and \siunit{4.5}{K} helium refrigerators are located in an underground access hall~\cite{bib:cr-0014} that is connected to the surface, where the helium compressors, gas tanks and further cryogenic infrastructure are located.
The total helium inventory is approximately half of the \siunit{500}{GeV} machine, namely $310 000$ liquid litres or about $41$ metric tonnes, about one third of the LHC's helium inventory.


\subsubsection{Existing and Future Facilities}

\subsubsection{Series Production and Industrialisation, Worldwide and in Europe}

Due to the construction of the European XFEL, the industrial basis for the key SCRF components is broad and mature, in particular in Europe.
Europe has a leading supplier for raw material. 
In all three regions (Europe, America, Asia), several vendors for cavities have been qualified for ILC type cavities, and provided cost estimates in the past.
Two leading cavity vendors are European companies that have profited from large scale production of cavities for XFEL; 
both have won contracts for LCLS-II as a consequence, which illustrates the potential to create revenue from outside Europe should the ILC be built.
RF couplers have also been successfully produced by European and  American vendors for the XFEL and LCLS-II projects.

ILC/TESLA type cryo modules have been built in laboratories around the world (DESY, CEA in Europe, FNAL and JLAB in America, KEK in Asia).
Series production has been established in America at Fermilab and JLAB for LCLS-II,.
The largest series production was conducted by CEA in France, again for the XFEL, with the assembly of \num{103} cryo modules in total by an industrial partner under the supervision of CEA personnel, with a final throughput of one cryo module produced every four working days.

ILC type, pulsed \siunit{10}{MW} klystrons are commercially available from two vendors in Japan and Europe.

For XFEL, China has been a supplier for niobium raw material and cryomodule cold masses (the cryostat with internal insulation and tubing).
For the planned SCLF project in Shanghai, China has started to develop cavity and cryo module production capabilities, which will further broaden the worldwide production capabilities for SCRF components.
This reduces the risk that prices are pushed up by a monopoly of manufacturers for a large scale order of components as required for the iLC.

Overall, European industry is well prepared to produce the high-tech, high-value SCRF components needed for the ILC, which would likely constitute largest fraction of any European in-kind contribution (IKC) to the ILC, at very competitive prices.
Thus, expenditure for the European IKC will likely stay in Europe, with an excellent chance to stay within the price range assumed in the value estimate.
Moreover, European companies are well poised to win additional contracts from other regions, increasing the economic benefit for Europe from an ILC project.



%===============================================================================

\subsection{Accelerator Design}

{\it 
Description of accelerator design - 5 pages 

Table: Accelerator parameters (250 GeV initial stage, energy and luminosity upgrades
}



\subsubsection{Electron and Positron Sources}

The electron and positron sources are designed to produce \siunit{5}{GeV} beam pulses with the a bunch charge that is $50\,\%$ higher than the design bunch charge of \siunit{3.2}{nC} ($\siunit{2\cdot 10^{10}}{e}$), in order to have sufficient reserve to compensate  any unforeseen inefficiencies in the beam transport.
in the baseline design, both sources produce polarized beams with the same time structure as the main beam, i.e. $1312$ bunches in a $\siunit{727}{\mu s}$ long pulse.

The electron source design~\cite{Adolphsen:2013kya} is based on the SLC polarized electron source, which has demonstarted that the bunch charge, polarisation and cathode lifetime parameters are feasible.
Only the long bunch trains of the ILC require a newly developed laser system and powerful preaccelerator structures, for which preliminary designs are available.
The design foresees a Ti:sapphire laser impinging on a photocathode based on a strained GaAs/GaAsP superlattice
structure, which will produce electron bunches with an expected polarisation of \siunit{85}{\%},
sufficient for \siunit{80}{\%} beam polarization at the interaction point, as demonstrated at SLAC~\cite{Alley:1995ia}.

The positron source poses a larger challenge. 

In the baseline design, hard gamma rays are produced in a helical undulator driven by the main electron beam, which are converted to positrons in a rotating target.
Positrons are captured in a flux concentrator or a quarter wave transformer, accelerated to \siunit{400}{MeV} in two normal conducting preaccelerators followed by a superconducting accelerator very similar to the main linac, before they are injected into the damping rings at \siunit{5}{GeV}.
Compared to planar undulators, the helical undulators produce twice as many photons, and with circular polarisation, which is transferred to the positrons produced in the target.
The positron polarisation thus achieved is $30\,\%$.
The E-166 experiment at SLAC has successfully demonstrates this concept  \cite{Alexander:2009nb}, albeit at intensities much lower than foreseen for the ILC. 
Technological challenges of the undulator source concept are the target heat load, radiation load in the flux concentrator device, and dumping the high intensity photon beam remnant.

As an alternative, an electron driven positron source concept has been developed.
In the electron driven scheme, a \siunit{3}{GeV} electron beam from a dedicated normal conducting linac produces positrons in a rotating target.
The electron drive beam, being independent from the main linac, has a completely different time structure. 
Positrons are produced in $20$ pulses at \siunit{300}{Hz} with $66$ bunches each, i.e., it takes about \siunit{67}{ms} to produce all positrons for a single Main Linac pulse with its $1312$ bunches, compared to \siunit{0.8}{ms} for the undulator source.
This different time structure reduces spreads the heat load on the target over a longer time, allowing a target rotation speed of only \siunit{5}{m/s} rather than \siunit{100}{m/s}, which reduces the engineering complexity of the target design, in particular the vacuum seals of the rotating parts.
Although not free from its own engineering challenges, such as the high beam loading in the normal conducting cavities, the electron driven design is currently considered to be a low risk design that is sure to work.

Aside from the low technical risk, the main advantage of the electron driven design is the independence of positron production and electron main linac operation, which is an advantage for accelerator commissioning and operation in general.
In particular, electron beam energies below \siunit{120}{GeV} for operation at the $Z$ resonance or the $WW$ threshold would be no problem.
The undulator source, on the other hand, offers the possibility to provide beams at the maximum repetition rate of \siunit{10}{Hz} given by the damping time in the damping rings of \siunit{100}{ms}, whereas the electron driven scheme is limited to \siunit{6}{Hz} due to the additional \siunit{66}{ms} for positron production.
The main difference, however, between the concepts is the positron polarisation offered by the undulator source, which adds significantly to the physics capabilities of the machine, as discussed elsewhere in this report. 

Both concepts have been reviewed recently \cite{PWG:2018a} inside the ILC community, with the result that both source concepts appear viable, with no known show stoppers, but require some more engineering work. 
The decision on the choice will be taken once the project has been approved, based on the physics requirements, operational aspects, and technological maturity and risks. 


\begin{table*}
\begin{tabular}{lccccc}
Quantity & Symbol & Unit & Initial &  \multicolumn{2}{c}{Upgrades} \\
\hline
Centre of mass energy & $\sqrt{s}$ & ${\mathrm{GeV}}$ & $250$ & $500$ & $1000$ \\
Luminosity & \multicolumn{2}{c}{${\mathcal{L}}$ $10^{34}{\mathrm{cm^{-2}s^{-1}}}$} & $1.35$ & $1.8$ & $4.9$ \\
Repetition frequency &$f\sub{{rep}}$ & ${\mathrm{Hz}}$  & $5$ & $5$ & $4$ \\
Bunches per pulse  &$n\sub{{bunch}}$ & 1  & $1312$ & $1312$ & $2450$ \\
Bunch population  &$N\sub{{e}}$ & $10^{10}$ &$2$ & $2$ & $1.74$ \\
Linac bunch interval & $\Delta t\sub{{b}}$ & ${\mathrm{ns}}$ & $554$ & $554$ & $366$ \\
Beam current in pulse & $I\sub{{pulse}}$ & ${\mathrm{mA}}$& $5.8$ & $5.8$ & $7.6$  \\
Beam pulse duration  & $t\sub{{pulse}}$ & ${\mathrm{\mu s}}$ &$727$ & $727$ & $897$ \\
Average beam power  & $P\sub{{ave}}$   & ${\mathrm{MW}}$ & $5.3$   &$10.5$  & $27.2$ \\  
Norm. hor. emitt. at IP & $\gamma\epsilon\sub{{x}}$ & ${\mathrm{\mu m}}$& $5$ & $10$ & $10$  \\ 
Norm. vert. emitt. at IP & $\gamma\epsilon\sub{{y}}$ & ${\mathrm{nm}}$ & $35$ & $35$ & $35$ \\ 
RMS hor. beam size at IP  & $\sigma^*\sub{{x}}$ & ${\mathrm{nm}}$  & $516$ & $474$ & $335$ \\
RMS vert. beam size at IP &$\sigma^*\sub{{y}}$ & ${\mathrm{nm}}$ & $7.7$  & $5.9$ & $2.7$ \\
Site AC power  & $P\sub{{site}}$ &  ${\mathrm{MW}}$ & $129$ & $163$ & $300$ \\
Site length & $L\sub{{site}}$ &  ${\mathrm{km}}$ & $20.5$ & $31$ & $40$ \\
\end{tabular}
\caption{Summary table of the ILC accelerator parameters in the initial $250\,{\mathrm{GeV}}$ staged configuration
and possible upgrades.
{\it XXX add column for TDR values at 250, add rows with beam parameters at IP XXX}
\label{tab:ilc-params}}
\end{table*}


\subsubsection{Damping Rings}

The ILC comprises two oval damping rings of \siunit{3.2}{km} circumference, sharing a common tunnel in the central accelerator complex.
The damping rings reduce the horizontal and vertical emittance of the beams by almost six orders of magnitude\footnote{The vertical emittance of the positrons is reduced from $\epsilon_{\mathrm{y}} \approx 0.8\,{\mathrm{\mu m}}$ to $2\,{\mathrm{pm}}$.} within a time span of only \siunit{100}{ms}, to provide the low emittance beams required at the interaction point. 
Both damping rings operate at an energy of \siunit{5}{GeV}.

The damping rings' main objectives are
\begin{itemize} 
\item accept electron and positron beams at large emittance and produce the low-emittance beams required for high-luminosity production,
\item dampen the incoming beam jitter to provide highly stable beams,
delay bunches from the source to allow feed-forward systems to compensate for pulse-to-pulse variations in parameters such as the bunch charge.
\end{itemize}

Compared to today's fourth generation light sources, the target value for the normalized beam emittance ($\siunit{4}{\mu m}$/\siunit{20}{nm} for the normalised horizontal / vertical beam emittance) is low, but not a record value, and thus considered to be a realistic goal.

The main challenges for the damping ring design are:
\begin{itemize} 
\item a sufficient dynamic aperture to cope with the large injected emittance of the positrons,
\item a low equilibrium emittance in the horizontal plane,
\item a very low emittance in the vertical plane,
\item a small damping time constant,
\item instabilities from electron clouds (for the positron DR) and fast ions (for the electron DR),
\item the small (\siunit{3.2-6.4}{ns}) bunch spacing, requiring very fast kickers for injection and ejection.
\end{itemize}

Careful optimization has resulted in a TME (Theoretical Minimum Emittance) style lattice for the arcs that balances a low horizontal emittance with the required large dynamic aperture~\cite[Chap. 6]{Adolphsen:2013kya}. 
Recently, the horizontal emittance has been reduced further by lowering the dispersion in the arcs through the use of longer dipoles~\cite{bib:cr-0016}.
The emittance in the vertical plane is minimised by careful alignment of the magnets and tuning of the closed orbit to compensate for misalignments and field errors, as demonstrated at the CESRTA accelerator~\cite{Billing:2011zc}.

The required small damping time constant requires large synchrotron radiation damping, which is provided by the insertion of $54$ wigglers in each ring.
This results in an energy loss of up to $7.7\,{\mathrm{MV}}$ per turn and up to $3.3\,{\mathrm{MW}}$ RF power necessary to 
store the a positron beam at the design current of $390\,{\mathrm{mA}}$, which
actually exceeds the average beam power of the accelerated positron beam of $2.6\,{\mathrm{MW}}$ at 
a $250\,{\mathrm{GeV}}$.

Electron cloud (EC) and fast ion (FI) instabilities limit the overall current in the damping rings to about \siunit{400-800}{mA}, where the EC limit that affects the positrons is assumed to be more stringent. 
These instabilities arise from electrons and ions being attracted by the circulating beam towards the beam axis. 
A low base vacuum pressure of \siunit{10^{-7}}{Pa} is required to limit these effects to the required level.
In addition, gaps between bunch trains of around $50$ bunches are required in the DR filling pattern, which permits the use of clearing electrodes to mitigate EC formation.
These techniques have been developed and tested at the CESR-TA facility~\cite{Conway:2012zza}

In the damping rings, bunch separation is only \siunit{6.4}{ns} (\siunit{3.2}{ns} for a luminosity upgrade to $2625$ bunches). 
Extracting individual bunches without affecting their emittance requires kickers with rise/fall times of \siunit{3}{ns} or less.
Such systems have been tested at ATF~\cite{Naito:2010zzb}.

The damping ring RF system will employ superconducting cavities operating at half the Main Linac frequency (\siunit{650}{MHz}).
Klystrons and accelerator modules can be scaled from existing \siunit{500}{MHz} units in operation at CESR and KEK~\cite[Sec. 6.6]{Adolphsen:2013kya}. 


\subsubsection{Low Emittance Beam Transport: Ring to Main Linac (RTML)}

The Ring to Main Linac (RTML) system~\cite[Chap. 7]{Adolphsen:2013kya} is responsible for transporting and matching the beam from the Damping Ring to the entrance of the Main Linac.
Its main objectives are
\begin{itemize} 
\item transport of the beams from the Damping Rings at the center of the accelerator complex to the upstream ends of the Main Linacs,
\item collimation of the beam halo generated in the Damping Rings,
\item and rotation of the spin polarisation vector from the vertical to the desired angle at the IP (typically, in longitudinal direction).
\end{itemize}

The RTML consists of two arms for the positrons and the electrons; 
each arm comprises a damping ring extraction line transferring the beams from the damping ring extraction into the main linac tunnel, a long low emittance transfer line (LTL), the turnaround section at the upstream end of each accelerator arm, and a spin rotation and diagnostics section.

The long transport line is the largest most costly part of the RTML.
The main challenge is to transport the low emittance beam at \siunit{5}{GeV} with minimal emittance increase, and in a cost-effective manner, considering that its total length is about \siunit{14}{km} for the \siunit{250}{GeV} machine.

In order to preserve the polarisation of the particles generated in the sources, their spins are rotated into a vertical direction (perpendicular to the Damping Ring plane) before injection into the Damping Rings. 
A set of two rotators~\cite{Emma:1995kf} employing superconducting solenoids allows to rotate the spin into any direction required.

At the end of the RTML, after the spin rotation section and before injection into the bunch compressors (which are considered part of the Main Linac, not the RTML~\cite{bib:cr-0010}), a diagnostics section allows to measure the emittance and coupling between horizontal and vertical plane; 
a skew quadrupole system is included to correct for any such coupling.

A number of circular, fixed aperture and rectangular, variable aperture collimators in the RTML provide betatron collimation at the beginning of the LTL, in the turn around and before the bunch compressors.


\subsubsection{Bunch Compressors and Main Linac}

\begin{figure}[htbp]
   \includegraphics[width=\hsize]{chapters/figures/ILC2016_tunnel_A1_160826-low4}
\caption{Artist's rendition of the ILC Main Linac tunnel. The shield wall in the middle has been removed.
\copyright Rey.Hori/KEK.}
\label{fig:ilc-tunnel}
\end{figure}

The heart of the ILC are the two Main Linacs, which accelerate the beams from $5$ to \siunit{125}{GeV}.
The linac tunnel, as depicted in Figs.~\ref{fig:ilc-tunnel} and \ref{fig:ml-tunnel}, has two parts, separated by a shield wall. 
One side (right in Fig.~\ref{fig:ilc-tunnel}) houses the beamline with the accelerating cryo modules as well as the RTML beamline hanging on the ceiling.
The other side contains power supplies, control electronics, and the modulators and klystrons of the High-Level RF system.
The concrete shield wall (indicated as a dark-grey strip in in Fig.~\ref{fig:ilc-tunnel}) has a thickness of \siunit{1.5}{m}~\cite{bib:cr-0012}.
The shield wall allows access to the electronics, klystrons and modulators during operation of the klystrons with cold, resonant cavities, protecting personnel from X-ray radiation emanating from the cavities caused by dark currents.
Access during beam operation, which would require a wall thickness of \siunit{3.5}{m}, is not possible.

\begin{figure}[htbp]
   \includegraphics[width=\hsize]{chapters/figures/ML-cross-section}
\caption{Cross section through the Main Linac tunnel.}
\label{fig:ml-tunnel}
\end{figure}

The first part of the Main Linac is a two-stage bunch compressor system~\cite[Sec. 7.3.3.5]{Adolphsen:2013kya}, each consisting of an accelerating section followed by a wiggler. 
The first stage operates at \siunit{5}{GeV}, with no net acceleration, the second stage accelerates the beam to \siunit{15}{GeV}.
The bunch compressors reduce the bunch length from $6$ to \siunit{0.3}{mm}.

After the bunch compressors, the Main Linac continues for about \siunit{6}{km} with four long strings of cryomodules. 

\paragraph{RF distribution}

Each \siunit{12.65}{m} long cromodule contains $9$ cavities, or for every third module, $8$ cavities and a package with a superconducting quadrupole, corrector magnets, and beam position monitor.
Nine such modules, with a total of $117$ cavities, are powered by $2$ klystrons and provide \siunit{3.83 (4.29)}{GeV} at a gradient of \siunit{31.5 (35)}{MV/m}.
The waveguide distribution system allows an easy refurbishment to connect a third klystron for a luminosity upgrade.
The $50\,\%$ RF power increase would allow $50\,\%$ higher current through smaller bunch separation, and longer beam pulses because of a reduced filling time, so that the number of bunches per pulse and hence the luminosity can be doubled, while the RF pulse duration of \siunit{1.65}{ms} stays constant.

\paragraph{Cryogenic supply}


Each $9$ module unit \siunit{114}{m} long, forms a cryo string, which is connected to the helium supply line with a Joule-Thomson valve.
All helium lines are part of the cryomodule, obliterating the need for a separate helium transfer line. 
Up to $21$ strings with $189$ modules and \siunit{2.4}{km} total length can be connected to a single plant; 
this is limited by practical plant sizes and the gas--return header pressure drop.  


\paragraph{Cost reduction from larger gradients}

Fig.~\ref{fig:ml-cryo-opta} shows the layout of the cryogenic supply system for the \siunit{250}{GeV} machine.
At the top, the situation is depicted for the gradient of \siunit{31.5}{MV/m} with a quality factor of $Q\sub{0}=1.0\cdot 10^10$, as assumed in the TDR~\cite{Adolphsen:2013kya}. 
In this case, the access points PM$\pm 10$ would house two cryogenic plants, each supplying up to $189$ cryomodules or an equivalent cryogenic load.
The bottom picture shows the situation for a gradient of \siunit{35}{MV/m} with $Q\sub{0}=1.6\cdot 10^10$, as could be expected from successful R\&D. 
The increased gradient would  allow to reduce the total number of cryomodules by roughly $10\,\%$ from $987$ to $906$, and the increased quality factor would reduce the dynamic losses such that $4$ cryo plants would provide sufficient helium, while in the top configuration $6$ large plants in the access halls plus $2$ smaller plants in the central region would be needed.

Thus, the accelerator is designed to make good use of any anticipated performance gain from continued high gradient R\&D, without incurring unwanted technology risk in the case that it turns out that raising the gradient is beneficial from an economical point of view.

\begin{figure}[htbp]
   \includegraphics[width=\hsize]{chapters/figures/arxiv-1711-00568-fig-3-4}
   \includegraphics[width=\hsize]{chapters/figures/arxiv-1711-00568-fig-3-7}
\caption{Cryogenic layout for a gradient of \siunit{31.5}{MV/m} (top) and \siunit{35}{MV/m} (bottom)~\cite{Evans:2017rvt}.
``Module space'' indicates how many cryomodules can be physically installed, ``cryomodules'' and ``RF unit'' indicates the number of actually installed modules and klystrons (one klystron per 4.5 cryomodules). ``E gain'' indicates the energy gain in GeV. ``BC'', ``ML'', ``e+ inj'', ``e- inj'' and ``UND'' refer to the sections with need for liquid helium: bunch compressor, main linac, 5GeV boosters in the positron and electron source, and the positron source undulator section, respectively. PM$\pm8, 10, 12$ refer to access hall locations, ``C'' to cryo plants; meter numbers on top indicate the length of the corresponding section.}
\label{fig:ml-cryo-opta}
\end{figure}



\subsubsection{Beam Delivery System and Machine Detector Interface}

%{\it Stress ATF2, FONT (maybe SLAC FFTB)
%Nomura issues: Dumps, crab cavities
%Mention beam dumps as well}

The Beam Delivery System (BDS) transports the $e^+/e^-$ beams from the end of the main linacs, focusses them to the required small beam spot at the Interaction Point (IP), brings them into collision, and transports the spent beams to the main dumps~\cite[Chap. 8]{Adolphsen:2013kya}.
The mainfunctions of the BDS are
\begin{itemize}
\item measuring the main linac beam and matching it into the final focus,
\item protecting beamline and detector from mis-steered beams~\footnote{On the electron side, the protective fast beam abort system is actually located upstream of the positron source undulator.},
\item remove large amplitude (beam--halo) and off--momentum particles from the beam to minimize background in the detector,
\item accurately measure the key parameters energy and polarisation before and after the collisions.
\end{itemize}
The BDS must provide sufficient diagnostics and feedback systems to achieve these goals.

The BDS is designed such that it can be upgraded to a maximum beam energy of \siunit{500}{GeV}; components such as the dumps that are not cost drivers for the overall project but would be cumbersome to replace later are dimensioned for the maximum beam energy from the beginning.
In other places, such as the energy collimation dogleg, those components necessary for \siunit{125}{GeV} beam operation are installed and space for a later upgrade is reserved.

Overall, the BDS is \siunit{2254}{m} long from the end of the main linac (or the undulator and target bypass insert of the positron source on the electron side, respectively) to the IP.
It starts with a diagnostics section, where emittance, energy and polarisation are measured and any coupling between the vertical and horizontal planes is corrected by a set of skew quadrupoles.
The energy measurement is incorporated into the machine protection system and can extract e.g. off--momentum bunches caused by a klystron failure in the main linac that would otherwise damage the machine or detector.
An emergency dump~\cite{bib:cr-0013} is dimensioned such that it can absorb a full beam pulse at \siunit{500}{GeV}, sufficient for \siunit{1}{TeV} operation.

\paragraph{Diagnostics and collimation section}
The diagnostics section is followed by a collimation system, which first removes beam halo particles (betatron collimation). 
then off--momentum particles are removed.
In this energy collimation section, sufficient dispersion has to be generated by bending the beam in a dogleg, while avoiding excessive synchrotron radiation generation in dispersive regions that leads to an increase of the horizontal emittance.
This emittance dilution effect grows as $E\sub{beam}^6$ at constant bending radius for the normalized emittance, and determines the overall length of the energy collimation section for a maximum \siunit{500}{GeV} beam energy to about \siunit{400}{m}.



XXXXXXXXXXXXX CONTINUE HERE XXXXXXXXXXXXXX

\paragraph {Final focus with feedback system and crab cavities}

\paragraph {Test results from ATF2}
The Accelerator Test Facility 2 (ATF2) was built at KEK in 2008 as a test bed for the ILC final focus scheme~\cite[Sec. 3.6]{Adolphsen:2013jya}.
Its primary goals are~\cite{Grishanov:2005ek,Grishanov:2006kx} to achieve a \siunit{37}{nm} vertical beam size at the interaction point (IP), and demonstrate beam stabilisation at the nanometre level.
After scaling for the different beam energies (ATF2 operates at $E\sub{beam}=\siunit{1.3}{GeV}$), the \siunit{37}{nm} beam size corresponds to the TDR design value of $\sigma\sub{y}^* = \siunit{5.7}{nm}$ at \siunit{250}{GeV} beam energy.
As Fig.~\ref{fig:atf-results} shows, this goal has been reached within $10\,\%$~\cite{Okugi:2017jji} by the successive application of various correction and stabilisation techniques, 
validating the final focus design, including the local chromaticity correction~\cite{White:2014vwa}.

\begin{figure}[htbp]
   \includegraphics[width=\hsize]{chapters/figures/ATF2trend2018}
\caption{Beamsizes achieved at the Accelerator Test Facility 2 (ATF2) as a function of time~\cite{bib:atf2esu}. The latest result (\siunit{41}{nm}~\cite{Okugi:2017jji}) is within $10\,\%$ of the goal beam size of \siunit{37}{nm}.}
\label{fig:atf-results}
\end{figure}

\paragraph {Machine detector interface (MDI)}

\paragraph {Extraction line}

\paragraph {Main dump}

%===============================================================================

\subsection{Upgrade Options \label{subsec:upg-opt}}


{\it 
Description of luminosity and energy upgrades - 1 page }

Given the high initial invest for a facility as large as the ILC, it is mandatory to have an interesting physics programme for several decades, with the possibility to adapt the programme to the needs arising from the knowledge obtained by the LHC, the ILC itself, all other particle physics experiments, and other branches of physics such as cosmology.
Several options exist for upgrades of the ILC in terms of energy, luminosity, and beam polarisation.

\subsubsection{Energy upgrade}
\label{subsubsec:upg-optE}

The obvious advantage of a linear collider is its upgradeability in energy.
Basically, the main linacs can be extended as far as desired, at constant cost per added beam energy, with some added cost for the relocation of the turn arounds. and bunch compressors.
Additional costs arise when the beam delivery system (BDS), including the beam dumps, has to be extended to handle the increased beam anergy; 
the current ILC BDS is designed to be easily upgradeable for centre of mass energies up to \siunit{1}{TeV} at minimal cost.

Depending on the actual gradient achieved for the construction of the ILC, there may be space for the installation of up to $171$ additional cryomodules, which would rise the centre--of--mass energy by about \siunit{54}{GeV} to around \siunit{304}{GeV}, as Fig.~\ref{fig:ml-cryo-opta} shows, 
and possibly require the installation of two additional cryo plants.

A further energy upgrade would require en extension of the tunnel.
The Kitakami site can accommodate a total accelerator length of at least \siunit{50}{km}, more than enough for \siunit{1}{TeV} centre--of--mass energy.
Any extension of the accelerator would proceed by adding new cryomodules at the low energy (upstream) ends of the accelerator, there is no need to move modules already installed. 

An upgrade would likely proceed in two phases: a preparation phase while the accelerator is still operated and produces data, and a refurbishment phase where the accelerator is shut down.

During the preparation phase, the necessary components would be acquired and built, in particular the cryomodules, klystrons, and modulators. 
At the same time, civil engineering would proceed with the excavation of new access tunnels, underground halls, and the main tunnel.
Recent studies conducted during road tunnel construction in the Kitakami area, in the same rock formation as foreseen for the ILC, indicate that the level of vibrations caused by tunnelling activities would allow to bring the new tunnels quite close to the existing ones before machine operation would be affected~\cite{bib:sanuki:lcws2018}, minimising the shutdown time necessary.

During the installation phase, the newly built tunnels would be connected to the existing ones, the beam lines at the turn-around and the wiggler sections of the bunch compressors would be dismantled, and the new cryo modules would be installed as well as the new turn-around and bunch compressors. 
At the same time, any necessary modifications to the positron source and the final focus can be made.
With the cryo modules ready for installation at the beginning of the shut down period, it is estimated that the shutdown could be limited to about a year for an energy upgrade.

{\it XXX Mention quantization from timing constraint, leads to stage with 500-600 GeV, optimal for tth and testing Higgs self coupling in the region relevant for electroweak baryogenesis Sec. \ref{subsubsec:runscen_ilc500} XXX }

\subsubsection{Luminosity upgrade}
\label{subsubsec:upg-optL}

Fundamentally, the luminosity of the ILC could be increased by increasing the luminosity per bunch (or per colliding charge), or increasing the number of bunches per second.

Increasing the luminosity per bunch requires a smaller beam spot size, which may be achieved by tighter focussing and/or smaller beam emittance.
Studies indicate that with enough operating experience, there is potential for a further luminosity increase. 
This route to increased luminosity is, however, invariably linked to higher beam disruption, which brings larger energy spread and higher backgrounds and thus more challenging conditions for the experiments.

The ILC design also has the potential to increase the number of colliding bunches per second, by doubling the number of bunches per pulse, and possibly by increasing the pulse repetition frequency.

Doubling the number of bunches per pulse to $2625$ would coincide with a smaller bunch distance, require the installation of $50\,\%$ more klystrons and modulators. 
As the RF pulse length of \siunit{1.65}{ms} is unchanged, the cryogenic load is essentially unchanged.
Doubling the number of bunches would double to beam current in the damping rings;
for the positron damping ring, this may surpass the limitations from electron cloud (EC) instabilities. 
To mitigate this risk, the damping ring tunnel is large enough to house a third damping ring, so that the positron current would be distributed over two rings.

The pulse repetition rate (\siunit{5}{Hz} in the baseline configuration) is limited by the available cryogenic capacity, the damping time in the damping rings, and the target heat load in the positron source target.
The damping rings are designed for a \siunit{100}{ms} damping time and thus capable of a repetition rate of up to \siunit{10}{Hz}, twice the nominal rate.
Operation at an increased repetition rate would be possible if after an energy upgrade the machine is operated below its maximum energy (e.g., \siunit{250}{GeV} operation of a \siunit{500}{GeV} machine for a larger low-energy data set), or if additional cryogenic capacity is installed.

\subsubsection{Polarisation upgrade}
\label{subsubsec:upg-optP}

The baseline design foresees at least $80\,\%$ electron polarisation at the IP, combined with $30\,\%$ positron polarisation for the undulator positron source.
At beam energies above \siunit{125}{GeV}, the undulator photon flux increases rapidly. 
Photons polarisation is maximal at zero emission angle; it is decreased and even inverted at larger angles.
Thus, collimating the surplus photon flux at larger emission angles increases the net polarisation. 
Studies indicate that $60\,\%$ positron polarisation at the IP may be possible at \siunit{500}{GeV} centre--of--mass energy with the addition of a photon collimator.
 


%===============================================================================

\subsection{Civil Engineering and Site}

{\it 
Description of Civil engineering plans and Kitakami site - 3 pages

Figures: Kitakami Site}



\begin{figure}[htbp]
   \includegraphics[width=\hsize]{chapters/figures/Kitakami_Geology}
\caption{Geological situation at the Kitakami site.}
\label{fig:kitakami-geology}
\end{figure}


%===============================================================================

\subsection{Cost and Schedule}

{\it 
Description of Cost estimate and schedule - 1 page 

include human resources, cost reduction effect by R\&D, operating costs
}




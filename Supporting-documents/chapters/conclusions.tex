%  Conclusions section

In this report, we have reviewed the full panorama of the
International Lineaer Collider project.   

This machine addresses
compelling physics questions.   In
our quest to discover new interactions beyond the Standard Model, the
couplings of the Higgs boson are the most obvious place to look, and
the one place that today we are not looking with sufficient power. 
 It will supply the capabilities that
we need to study the Higgs boson and other particles with the degree
of precision that is actually required to learn their secrets. 

The machine is ready for construction.   We have described the
detailed design of the ILC, explaining how the performance of  each
component
is supported by prototyping and, in most cases, by operational
experience.

The detectors are designed to meet the challenges of high precision.
Taking advantage of the more benign environment of $\ee$ colliders,
they are designed for performance on charged-particle tracking,
heavy-flavor identification, and hadron calorimetry that improve on
existing detectors by large factors.  These are essential capabilities
to confidently obtain the high-quality measurements that we seek.

The physics program begins with a stage at 250~GeV in the center of
mass.  At this stage, a Higgs boson are produced together with a $Z$
at 110~GeV lab energy that serves to tag the event.  This allows
unambiguous,
model-independent measurements of the total cross section for Higgs
boson production, the branching ratios for Higgs boson  decays.  It also gives
tool for searches for exotic Higgs boson decays, including decays to
invisible or partially visible final states.

Measurements at the 250~GeV stage will improve current
measurements of $W$ boson couplings and SM quark and lepton couplings
by large factors beyond what is possible at the HL-LHC.

The simplicity of $\ee$ pair production allows the full set of
electroweak and Higgs processes at the ILC to be combined in a global
fit based on an Effective Field Theory description of modifications of
the Standard Model.   This framework is essentially model-independent
with respect to new physics.  Within this framework, the 250~GeV stage
of the ILC will measure the $Hbb$ couplings to 1\%, the
$HWW$ and $HZZ$ to 0.7\%, and all other important Higgs boson
couplings to levels close to 1\%.   These are the levels of precision
required to access new physics beyond the reach of direct searches at HL-LHC.

The first-stage ILC is constructed to be readily upgraded in energy.
The accelerator and detectors are designed for operation up to a
center of mass energy of 1~TeV.   The technology, detector
performance, and physics for the 500~GeV stage has been described in
detail.  All of the measurements discussed in the previous paragraphs
benefit,
with the uncertainties in Higgs boson couplings decreasing by a factor
of 2.    The 500~GeV stage offers a program exploring 
the couplings of the top quark and thus a second, independent,
opportunity to probe for new physics through precision measurement.
It also offers the opportunity to search for pair-production of
elusive particles produced in electroweak interactions that are
challenging  to discover at the LHC. 

The opportunities that the ILC gives to discover new physics are
robust, and the ILC measurements are improvable as the accelerator
moves from one energy stage to the next.

We have compared the projected ILC performance on Higgs boson
couplings to those put forward for other colliders.  The ILC will
provide a signficant---and necessary---step in precision beyond the
HL-LHC.    A number of other proposals for $\ee$ Higgs factories are
now under discussion.  We have shown that none expects a performance
significantly superior to the ILC even at its 250~GeV stage.   Also,
no other proposal has been designed and costed at the level of a
formal proposal.   Only ILC is on the table now.

 Finally, the ILC laboratory will provide a base for future proposals
 of $\ee$ and $\gamma\gamma$ colliders based on advanced high-gradient
 acceleration.   The ILC laboratory thus can expect a long lifetime,
 beyond our current horizon, in which it will continue to explore the
 frontier of fundamental physics.

Come join us!   It is time to make the International Linear Collider a reality.
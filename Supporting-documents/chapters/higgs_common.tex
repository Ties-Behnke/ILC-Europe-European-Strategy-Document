%[pre-selection: isolated lepton finder or isolated lepton veto; overlay removal; jet clustering; 
%final-selection; to be filled]
The full simulation analysis at the event selection level can be described 
in two steps: {\it pre-selection} and {\it final-selection}. At the pre-selection step
each signal event is characterized according to its final states at parton level by
numbers of isolate leptons (meaning electron or muon unless otherwise stated), 
isolated taus, isolated photons and jets, and nature of missing momentum.
Here isolated particle is meant to be not coming from a jet. The procedures
for the pre-selection are typically as follows:
\begin{itemize}
%\begin{enumerate}[i]
\item {\it isolated lepton finder}, which will try to reconstruct the isolated leptons
in each event. The main algorithms are implemented in the processor called
IsolatedLeptonTagging~\cite{}, based on a multivariate method [as introduced in 
previous section?]. [in case not yet introduced]
It starts with selecting energetic electron/muon (momentum $P>5$ GeV), 
from the reconstructed particles collection, by requiring the particle has characteristic
energy fractions deposited in each sub-detector, namely $E_{ecal}/E_{tot}$,
$E_{tot}/P$, $E_{yoke}$, where $E_{ecal}$ ($E_{hcal}$) is the energy deposited 
in ECAL (HCAL), $E_{tot}$ is the sum of $E_{ecal}$ and $E_{hcal}$, and
$E_{yoke}$ is the energy deposited in Yoke. For electron, 
it is required that $E_{ecal}/E_{tot}>0.9$ and $0.5<E_{tot}/P<1.3$. For muon,
it is required that $E_{tot}/P<0.3$ and $E_{yoke}>1.2$ GeV. The selected
electron/muon is then further required to have impact parameters consistent with
that from primary vertex. Double cones are then defined around that electron/muon,
and variables such as the energies of the charged and neutral particles within a cone 
are utilized for isolation requirement. The exact criteria for isolation 
are realized by a MVA, trained using true isolated leptons and leptons from jets.
A cut on the MVA output is then required as the last step of isolated lepton finder.
\item {\it isolated tau/photon selection}, using TaFinder and PandoraPFA.
\item {\it overlay removal}, which will try to remove the pile-up beam background
events in every event. An exclusive jet clustering is performed using $k_t$ algorithm
on all the particles except the selected isolated lepton/tau/photon in above step. 
As a result, the particles from beam background events, which usually have very low-$p_t$, 
will be effectively removed by the exclusive jet clustering process in which the 
input parameters such as $R$ and number of required jets are carefully optimized 
for each signal process. Alternative algorithms include anti-$k_t$ and Valencia.
\item {\it jet clustering and flavor tagging}, using LCFIPlus. 
All the particles belonging to the
jets obtained in previous step are then re-clustered into a few jets using another inclusive 
jet clustering algorithm, Durham algorithm~\cite{}.
Each jet is flavor tagged using the reconstructed information of 
its secondary and tertiary vertices. 
\end{itemize}
%\end{enumerate}

At the final-selection step, the reconstructed leptons, taus, photons and jets 
will be first combined to reconstruct $W$, $Z$, $h$ or $top$ according to the signal. 
Then various cuts will be applied to further suppress background events.
Details are explained measurement by measurement in the following. 
Unless stated otherwise, the analysis is done at $\sqrt{s}=250$ GeV,
a nominal integrated luminosity of 250 fb$^{-1}$ is assumed, and
the cuts and results are illustrated with left-handed beam polarization
$e^-_Le^+_R: P(e^-,e^+)=(-0.8,+0.3)$. Additional comments will be given
when $\sqrt{s}$ or right-handed beam polarization 
$e^-_Re^+_L: P(e^-,e^+)=(+0.8,-0.3)$ has a significant impact on the results.
The results are straightforwardly extrapolated into that for the running scenario
introduced in Sec.~\ref{sec:runscenario} and are then used as input for the 
Higgs coupling determination by a global fit introduced in Sec.~\ref{sec:physics}.
